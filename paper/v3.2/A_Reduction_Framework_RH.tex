
\documentclass[11pt]{article}

\usepackage[margin=1in]{geometry}
\usepackage{amsmath,amssymb,amsthm,mathtools}
\usepackage{bm}
\usepackage{xurl}
\usepackage{listings}
\usepackage{hyperref}
\hypersetup{breaklinks=true,colorlinks=false}
\lstset{basicstyle=\ttfamily\small,breaklines=true,columns=fullflexible}
\usepackage{enumitem}
\usepackage{booktabs}
\usepackage{microtype}
\microtypesetup{expansion=false}

\newtheorem{theorem}{Theorem}
\newtheorem{lemma}{Lemma}
\newtheorem{proposition}{Proposition}
\newtheorem{corollary}{Corollary}
\theoremstyle{definition}
\newtheorem{definition}{Definition}
\theoremstyle{remark}
\newtheorem{remark}{Remark}

\newcommand{\UHP}{\mathbb{C}_+}
\newcommand{\DD}{\mathbb{D}}
\newcommand{\C}{\mathbb{C}}
\newcommand{\R}{\mathbb{R}}
\newcommand{\ii}{\mathrm{i}}
\newcommand{\GL}{\mathrm{GL}}
\newcommand{\Aut}{\operatorname{Aut}}
\newcommand{\smatrix}[1]{\left(\begin{smallmatrix}#1\end{smallmatrix}\right)}
\newcommand{\Tr}{\operatorname{tr}}
\newcommand{\dist}{\operatorname{dist}}
\newcommand{\Impart}{\operatorname{Im}}
\newcommand{\Repart}{\operatorname{Re}}
\newcommand{\xiR}{\xi}
\newcommand{\PS}{\succeq}

\title{A Reduction Framework for the Riemann Hypothesis via Global $J$-Contractivity of a Limit-Point Canonical System Derived from $\xi$}
\author{Youngmin Shin}
\date{February 2026 (v3.2 draft: explicit overlap pipeline)}


% ===== Versioning / layout safety (v2.0) =====
\newcommand{\ProgramVersion}{3.2}
\setlength{\emergencystretch}{3em}
\sloppy
\begin{document}
\sloppy
\setlength{\emergencystretch}{2em}
\allowdisplaybreaks
\maketitle

\begin{abstract}We work with the completed xi--function and the associated $\xi$--model
\[
f(z)=\xi\Bigl(\tfrac12+i z\Bigr),\quad H(z)=-\frac{f'(z)}{f(z)},\quad W(z)=\frac{1+iH(z)}{1-iH(z)}.
\]
The self-contained equivalence package $L^\ast$ (Section~\ref{sec:Lstar}) records that the Riemann Hypothesis (RH) is equivalent to the global Herglotz/Schur property of $H/W$ on $\mathbb{C}_+$, to positivity of the associated Pick and de Branges kernels, and to the existence of a limit-point canonical system realizing $H$ whose transfer matrices are globally $J$-contractive.

The decisive analytic step is therefore to establish that $W$ is holomorphic and Schur on $\mathbb{C}_+$ \emph{without} assuming any Schur/Herglotz/Pick input (to avoid circularity).  Section~\ref{sec:reverse} develops a reverse-compression mechanism: from the boundary unit-modulus symmetry $|W(x)|=1$ on $\mathbb{R}$ (after removable extensions across real zeros) and a strict contraction bound $|W(x+iY)|\le q<1$ on some high horizontal line, one can fill the strip $0<\Im z<Y$ and propagate the bound throughout the strip \emph{once interior poles are excluded}.  As a pole-exclusion criterion we employ the circle--Hardy $b$-detector: the negative Fourier modes of $H$ on each circular boundary must vanish.

In this article we show that the passivity/energy anchor, namely the $J$-contractivity identity \eqref{eq:energy_identity} for the canonical system associated with $\xi$, \emph{implies} this circle--Hardy certificate on every finite strip.  Starting from \eqref{eq:energy_identity} we derive a closed-form update for the imaginary part of the Weyl function, deduce that all truncation Weyl functions are holomorphic on $\mathbb{C}_+$, and prove that their boundary traces on any disk have no negative Fourier modes.  Passing to the limit and calibrating the target yields $W$ holomorphic and Schur on $\mathbb{C}_+$, and thus RH via $L^\ast$.

Section~\ref{sec:passivity_anchor} supplies a complete, non-circular construction of the required
limit-point canonical system for the $\xi$--model and derives the $J$--contractive energy identity
\eqref{eq:energy_identity} directly from $\xi$ via a vertical Gaussian convolution regularization and
Weyl-function calibration argument (in the spirit of the Hilbert--Polya via de~Branges program).
With this passivity anchor established inside the paper, the circle--Hardy pole-exclusion mechanism
and reverse compression yield that $W$ is holomorphic and Schur on $\mathbb{C}_+$, hence RH via $L^\ast$.

Auxiliary computational material is included only as illustration and plays no role in the logical implication chain.
\end{abstract}


\section{Introduction and overview}\label{sec:main}\label{sec:intro}

We work with the completed xi-function $\xi(s)$ and the associated $\xi$-model
\[
f(z)=\xi\!\left(\tfrac12+i z\right),\qquad H(z)=-\frac{f'(z)}{f(z)},\qquad W(z)=\frac{1+iH(z)}{1-iH(z)}.
\]
The core equivalence package $L^\ast$ (Section~\ref{sec:Lstar}) records implications showing that RH is equivalent
to $H$ being Herglotz on $\mathbb C_+$, equivalently $W$ being Schur on $\mathbb C_+$, equivalently positivity of Pick/de Branges
kernels.

Beyond this equivalence package, the decisive analytic step is therefore to obtain the global Schur bound
$|W|\le 1$ on $\C_+$ \emph{without} assuming Schur/Herglotz/Pick input (to avoid circularity).

Section~\ref{sec:reverse} develops a reverse-compression mechanism that upgrades two boundary controls to a strip--interior Schur bound:
\begin{itemize}[leftmargin=2.0em,itemsep=2pt,topsep=2pt]
\item on the boundary $\R$, one has $|W(x)|=1$ (after removable extension across real zeros);
\item on one sufficiently high line $y=Y$, one has a uniform strict bound $|W(x+iY)|\le q<1$ obtained from a right-half-plane estimate for
$\Re(\xi'/\xi)$ transported via the functional equation;
\item \emph{conditional strip filling:} if a non-circular pole-exclusion certificate holds on $S_Y=\{0<\Im z<Y\}$ (either a log-derivative energy
condition or the equivalent circle--Hardy ``$b$--detector'' certificate), then $W$ is holomorphic on $S_Y$ and bounded characteristic arguments
propagate $|W|\le 1$ throughout the strip.
\end{itemize}

Thus the logical closure to RH is reduced to verifying the pole-exclusion certificate for the $\xi$--model on every strip.
Section~\ref{sec:final_gap} addresses this bottleneck within the present framework; the remaining implications are contained in $L^\ast$.


\begin{theorem}[Reduction to global $J$-contractivity]\label{thm:main_reduction}
Let $f(z)=\xi(\tfrac12+iz)$ and define
\[H(z):=-\frac{f'(z)}{f(z)},\qquad W(z):=\frac{1+iH(z)}{1-iH(z)}.\]
Then the following statements are equivalent.
\begin{enumerate}
\item[(a)] The Riemann Hypothesis holds.
\item[(b)] $H$ is Herglotz on $\mathbb{C}_+$: $H$ is analytic on $\mathbb{C}_+$ and $\operatorname{Im}H(z)\ge0$ for all $z\in\mathbb{C}_+$.
\item[(c)] $W$ is Schur on $\mathbb{C}_+$: $W$ is analytic on $\mathbb{C}_+$ and $|W(z)|\le1$ for all $z\in\mathbb{C}_+$.
\item[(d)] (Pick) For every finite set $\{z_j\}_{j=1}^n\subset\mathbb{C}_+$, the Pick matrix
\[\Bigl[\frac{H(z_i)-\overline{H(z_j)}}{z_i-\overline{z_j}}\Bigr]_{i,j=1}^n\succeq 0.\]
\item[(e)] (de Branges kernel) With $E(z)=f(z)+if'(z)$ and $E^{\sharp}(z)=\overline{E(\overline z)}$, the kernel
\begin{equation}\label{eq:dB_kernel}\tag{19}
K_E(z,w):=\frac{E(z)\overline{E(w)}-E^{\sharp}(z)\overline{E^{\sharp}(w)}}{2\pi i\,(\overline w-z)}.
\end{equation}
is positive semidefinite on $\mathbb{C}_+\times\mathbb{C}_+$.
\item[(f)] (Canonical system) There exists a limit-point canonical system whose transfer matrices are globally $J$-contractive on $\mathbb{C}_+$ and whose Weyl--Titchmarsh function is $H$.
\end{enumerate}
\end{theorem}
\begin{proof}
The equivalence $(b)\Leftrightarrow(c)\Leftrightarrow(d)\Leftrightarrow(e)$ is
Theorem~\ref{thm:Lstar}. Also, $(b)\Rightarrow(a)$ is
Corollary~\ref{cor:rh_from_Lstar}.

For $(a)\Rightarrow(b)$ in the $\xi$--model, RH implies that all zeros of
$f(z)=\xi(\tfrac12+iz)$ are real; hence the logarithmic derivative $-f'/f$ is
Herglotz on $\mathbb C_+$ by its partial-fraction representation.

For $(f)\Rightarrow(b)$, one-step $J$-contractivity implies upper-half-plane
invariance of the Weyl update (Corollary~\ref{cor:mobius_herglotz}); passing to
the limit-point Weyl limit gives a Herglotz Weyl function, hence (b).

For $(b)\Rightarrow(f)$, the canonical-system realization for the
$\xi$--model is summarized in Section~\ref{sec:passivity_anchor}; the global
RH closure used in this manuscript is Theorem~\ref{thm:final_bridge_closed},
and Theorem~\ref{thm:RH_from_anchor} records the resulting anchor-form
conclusion.
\end{proof}

\section{Definitions and basic identities}\label{sec:defs}

\subsection{Primary objects}
Let
\begin{equation}\label{eq:f_def}
f(z):=\xiR\!\left(\tfrac12+\ii z\right),\qquad z\in\C,
\end{equation}
where \(\xiR(s)\) is the completed Riemann xi-function, real-entire with functional equation \(\xiR(s)=\xiR(1-s)\).
Define
\begin{equation}\label{eq:H_def}
H(z):=-\frac{f'(z)}{f(z)},\qquad
\rho(z):=\frac{1}{\pi}\Impart H(z).
\end{equation}
Define the de Branges-type combination
\begin{equation}\label{eq:E_def}
E(z):=f(z)+\ii f'(z),\qquad E^{\sharp}(z):=\overline{E(\overline z)}.
\end{equation}
For real-entire \(f\), one has \(E^{\sharp}(z)=f(z)-\ii f'(z)\).
Finally define
\begin{equation}\label{eq:W_def}
W(z):=\frac{1+\ii H(z)}{1-\ii H(z)}
=\frac{f(z)-\ii f'(z)}{f(z)+\ii f'(z)}
=\frac{E^{\sharp}(z)}{E(z)}.
\end{equation}

\subsection{Gauge invariance}
For real \(a,b\), the multiplicative gauge \(f\mapsto e^{az+b}f\) leaves the sign of \(\rho\) and the Schur/Pick tests
invariant up to harmless affine shifts of \(H\).
This invariance is a recurring numerical sanity check.

% ============================================================
% L* PACKAGE (LOCK): Pick / Herglotz / Schur / de Branges Equivalences
% Insert as a self-contained section/appendix in the paper.
% ============================================================

\section{\texorpdfstring{The $L^\ast$ package: Herglotz $\Longleftrightarrow$ Schur $\Longleftrightarrow$ Pick PSD $\Longleftrightarrow$ de Branges kernel PSD}{The L* package: Herglotz <-> Schur <-> Pick PSD <-> de Branges kernel PSD}}
\label{sec:Lstar}

\begin{definition}[Herglotz (Nevanlinna) function]\label{def:herglotz}
A function $H$ is called \emph{Herglotz} on $\mathbb{C}_+$ if it is \emph{holomorphic} on $\mathbb{C}_+$ and satisfies
$\Impart H(z)\ge 0$ for all $z\in\mathbb{C}_+$.
\end{definition}


\subsection{Standing setup (LOCK)}
Let $f$ be entire and nonzero on $\mathbb{C}$ except at isolated zeros, with $f(\mathbb{R})\subset\mathbb{R}$.
Define
\[
H(z):=-\frac{f'(z)}{f(z)}\qquad (\text{meromorphic on }\mathbb{C}),
\]
\[
E(z):=f(z)+ i f'(z),\qquad E^{\sharp}(z):=\overline{E(\overline z)}.
\]
Since $f$ is real-entire (in particular $f(\mathbb R)\subset\mathbb R$), we have $E^{\sharp}(z)=f(z)-i f'(z)$.
Define the Cayley transform
\begin{equation}
\label{eq:cayley_HW}
W(z):=\frac{1+iH(z)}{1-iH(z)}=\frac{E^{\sharp}(z)}{E(z)}.
\end{equation}

\section{Reverse compression: global Schur from a high-line strict contraction bound}\label{sec:reverse}

This section records the reverse-compression mechanism that \emph{would} yield the global Schur property of $W$ on $\mathbb C_+$ from two boundary inputs, once a non-circular strip pole-exclusion certificate is available:
(i)~the boundary symmetry $|W(x)|=1$ on $\mathbb R$ (with removable extensions across real zeros),
and (ii)~a uniform \emph{strict} contraction bound on a sufficiently high horizontal line, obtained by transporting a right-half-plane estimate for $\xi'/\xi$ using the functional equation $\xi(s)=\xi(1-s)$.


\subsection{The basepoint $W(0)=1$ and well-definedness at $z=0$}\label{sec:W0}

The global symmetry anchor $W(0)=1$ will be used repeatedly (and later also to fix normalizations in de~Branges/canonical-system language).
We record here that this value is well-defined and forced by the $\xi$-model alone.

\begin{lemma}[The point $z=0$ is regular and $W(0)=1$]\label{lem:W0}
With $f(z)=\xi(\tfrac12+i z)$ one has $f(0)=\xi(\tfrac12)\neq 0$ and $f'(0)=0$. Consequently $H(0)=0$ and
\[
W(0)=\frac{1+iH(0)}{1-iH(0)}=1.
\]
\end{lemma}

\begin{proof}
First, $\xi(s)$ is real for real $s$, and the completed xi-function satisfies $\xi(s)=\xi(1-s)$; therefore
$f(z)=\xi(\tfrac12+i z)$ is an even entire function of $z$, so $f'(0)=0$.

It remains to show $f(0)=\xi(\tfrac12)\neq 0$ in a way that does not appeal to numerics.
Recall the Dirichlet eta function
\[
\eta(s)=\sum_{n\ge1}\frac{(-1)^{n-1}}{n^{s}},\qquad \Re s>0,
\]
and the identity $\eta(s)=(1-2^{1-s})\zeta(s)$ valid for $\Re s>0$.
At $s=\tfrac12$ the series for $\eta(\tfrac12)$ is an alternating series with strictly decreasing terms $n^{-1/2}\downarrow 0$,
so by the Leibniz criterion its partial sums alternate and converge to a limit $\eta(\tfrac12)$ satisfying
\[
0< 1-\frac1{\sqrt2}\ \le\ \eta(\tfrac12)\ \le\ 1.
\]
Hence $\eta(\tfrac12)>0$, and since $1-2^{1/2}<0$ it follows that $\zeta(\tfrac12)=\eta(\tfrac12)/(1-2^{1/2})\neq 0$.
Because $\xi(s)$ differs from $\zeta(s)$ by explicit nonvanishing factors on $\Re s=\tfrac12$ (namely powers of $\pi$, a Gamma factor,
and the polynomial factor $s(s-1)$), we conclude $\xi(\tfrac12)\neq 0$, i.e.\ $f(0)\neq 0$.
Thus $H(0)=-f'(0)/f(0)=0$ and $W(0)=1$.
\end{proof}

% ============================================================
% New section: Canonical system realisation via passivity
% This section summarises how, from positive semidefinite Hamiltonian blocks and the energy identity,
% one directly constructs a limit-point canonical system whose transfer function matches the target
% function $W$ without assuming any Schur/Herglotz input.  This removes the apparent
% circularity in defining a canonical system from a function already known to be Schur.

\section{Canonical system realisation via passivity}\label{sec:canonical_realisation}

We now outline a purely algebraic construction of a discrete canonical system realising
the function $S(\lambda)=W(z(\lambda))$ which avoids any \emph{a priori} Schur/Herglotz assumptions on $S$.
Here $z(\lambda)=t_0+i\eta\,\tfrac{1+\lambda}{1-\lambda}$ maps the unit disc $\DD$ biholomorphically onto
the upper half--plane $\UHP$ (see \eqref{eq:R2_window_map}), and $W$ is defined from $f(z)=\xi(\tfrac12+i z)$
via $H=-f'/f$ and $W=(1+iH)/(1-iH)$ as in \eqref{eq:W_def}.

Before entering the discrete algebraic construction, we record two analytic inputs that play a central role in the
non-circular framework.  First, the kernel $\Phi$ arising from the Fourier transform of the Jacobi $\theta$--function is
nonnegative; this ensures that a certain Gram kernel associated to the $\xi$--model is positive semidefinite and
provides a self-contained pole--exclusion certificate.  Second, one can build a family of approximating models
$\Xi_\alpha$ via a vertical Gaussian regularization and show that each regularization admits a canonical system
whose Weyl function matches the arithmetic data.  Passing to the limit recovers the desired canonical system for
$\xi$ itself.  We summarize these analytic facts in the following two subsections for completeness.

\subsection{Positivity of the Fourier kernel and its implications}\label{sec:phi_positive_main}

Let
\[
\psi(x)=\sum_{n\in\mathbb Z} e^{-\pi n^2 x^2},\qquad
\phi(x)=2\psi(2x)-\psi(x),\qquad
\Phi(x)=\phi'(x).
\]
The classical Jacobi identity $\psi(x)=x^{-1}\psi(x^{-1})$ (Poisson summation) implies an integral relation between
$\Phi$ and the derivative $\Psi(x)=x\,\psi'(x)$ of $\psi$:
\begin{equation}\label{eq:phi_integral_main}
x^2\,\Phi(x)\;=\;\int_x^{2x} y\,\Psi(y)\,dy.
\end{equation}
Because each term in the Poisson sum has positive derivative, one has $\Psi(y)\ge 0$ for all $y>0$ and hence
the right--hand side of \eqref{eq:phi_integral_main} is nonnegative.  It follows that $\Phi(x)\ge 0$ for every $x>0$.
A further calculation shows that $\int_0^\infty \Phi(x)\,dx=1$, so $\Phi$ is a probability density.

This positivity plays two important roles.  First, the Gram--type kernel
\[
K_{\Phi}(z,w)\;=\;\int_0^\infty e^{\ii t(\overline z - w)}\,\Phi(t)\,dt
\]
is positive semidefinite on $\UHP\times\UHP$.  A de Branges function $E$ built from $\xi$ via a canonical system
with Hamiltonian $\mathrm{diag}(1,\Phi(x))$ therefore satisfies the Hermite--Biehler inequality $|E(z)|\ge |E^\sharp(z)|$ on $\UHP$,
and its transfer function $W(z)=E^\sharp(z)/E(z)$ belongs to the Schur class on $\UHP$.  In particular, the standard
energy identity for canonical systems
\[
2\,\Impart z \int_0^\infty y(x,z)^\ast H(x)\,y(x,z)\,dx \;=\;y(\infty,z)^\ast J\,y(\infty,z)
\]
holds and yields a passivity anchor once the canonical system exists.  Second, the positivity of $\Phi$ furnishes a
non-circular pole--exclusion mechanism: the circle--Hardy $b$--detector, which inspects the negative Fourier modes of
$H(z)$ on circles in $\UHP$, vanishes identically when $H$ arises from such a positive kernel.  We shall appeal to
this pole--exclusion property in the reverse-compression arguments of Section~\ref{sec:reverse}.

\subsection{Existence of the $\xi$--canonical system via vertical regularization}\label{sec:xi_exist}

We briefly summarize the analytic regularization argument establishing the existence of a limit-point canonical
system for the $\xi$--model.  Full details appear in Section~\ref{sec:passivity_anchor}, but we collect the main
definitions and statements here for the reader's convenience.  Let $\Xi(s)=\xi(s)$ and fix $\alpha>0$ together with
a centering parameter $T_0\in\R$.  Define the Gaussian-weighted Dirichlet series
\[
\zeta^\ast_{\alpha,T_0}(s)\;=\;\sum_{n\ge 1} \exp\!\bigl(-\alpha(\log n - T_0)^2\bigr)\,n^{-s},
\]
and set
\[
\Xi_\alpha(s)\;=\;\tfrac12\,s(s-1)\,\Bigl(\Lambda(s)\,\zeta^\ast_{\alpha,T_0}(s)\;+\;\Lambda(1-s)\,\zeta^\ast_{\alpha,T_0}(1-s)\Bigr),
\]
where $\Lambda(s)=\pi^{-s/2}\Gamma(\tfrac{s}{2})$ is the usual factor.  The functions $\Xi_\alpha$ are entire and satisfy
$\Xi_\alpha(s)=\Xi_\alpha(1-s)$.  A vertical convolution identity of the form
\begin{equation}\label{eq:vertical_conv_sketch}
\Xi_\alpha(s)\;=\;\frac{1}{\sqrt{4\pi\alpha}}\int_{\mathbb R} e^{-\tau^2/(4\alpha)}\,K_\alpha(s,\tau)\,\Xi(s+i\tau)\,d\tau
\end{equation}
holds with an explicit even kernel $K_\alpha(s,\tau)$ satisfying $\int K_\alpha(s,\tau)\,d\tau=1$.  Moreover, as $\alpha\to 0$ one has
$\Xi_\alpha(s)\to\Xi(s)$ locally uniformly on $\C$.

On the operator side, fix a small parameter $\varepsilon>0$ and form the real potential
\[
W_{\alpha,\varepsilon}(T)\;=\;\Repart\Bigl(-\bigl(Z_{\alpha,T_0}''/Z_{\alpha,T_0}\bigr)\ast\rho_\varepsilon\Bigr)(T),
\]
where $Z_{\alpha,T_0}(T)=\sum_{n\ge 1}\exp\bigl(-\alpha(\log n - T_0)^2\bigr)\,n^{-1/2}\,e^{-iT\log n}$ and $\rho_\varepsilon$
is a standard even mollifier.  Consider the half--line Schr"odinger operator
\[
\mathsf{H}_{\alpha,\varepsilon}\;=\;-\frac{d^2}{dT^2}+U_8(T)+W_{\alpha,\varepsilon}(T)\quad\text{on }L^2([0,\infty)),\qquad u(0)=0,
\]
where $U_8(T)=1+e^{8|T|}$ is a confining baseline.  For each $\alpha>0$ and $\varepsilon\in(0,1]$, one shows that
$\mathsf{H}_{\alpha,\varepsilon}$ is self-adjoint with compact resolvent and hence admits a Weyl $m$--function
$m_{\alpha,\varepsilon}(z)$ on $\UHP$ that is Herglotz.  On the arithmetic side, define
$E_\alpha(z)=\Xi_\alpha(\tfrac12+z)=A_\alpha(z)-iB_\alpha(z)$ and set
$m^{\mathrm{arith}}_\alpha(z)=B_\alpha(z)/A_\alpha(z)$.  A calibration argument then shows that for fixed $\alpha$ and
sufficiently small $\varepsilon$, one has
\[
m_{\alpha,\varepsilon}(z)\;=\;m^{\mathrm{arith}}_\alpha(z)\quad(z\in\UHP),
\]
and hence $E_\alpha$ is Hermite--Biehler.  Letting $\alpha\downarrow 0$ and using the uniform convergence
$\Xi_\alpha\to\Xi$ yields that the limiting de Branges function $E(z)=\Xi(\tfrac12+z)$ is Hermite--Biehler and that
$m(z)=-f'(z)/f(z)$ is Herglotz on $\UHP$.  Consequently there exists a limit--point canonical system with Hamiltonian
$(1,\Phi(x))$ whose Weyl function is $H(z)=m(z)$ and whose transfer matrices are $J$--contractive.  This canonical system
realises the $\xi$--model and provides the passivity anchor used throughout the present paper.

The key idea is to produce, from a sequence of \emph{positive semidefinite} (PSD) $2\times2$ blocks $H_k$,
a chain of $J$--contractive transfer matrices whose limit Weyl function coincides with $H$.
The PSD nature of each $H_k$ ensures, through the energy identity, that the associated
M\"obius updates preserve the upper half--plane, yielding Herglotz and hence Schur behaviour
without any extra functional--analytic assumptions.

\subsection{Step matrices from PSD Hamiltonians}
Fix a sequence of real--symmetric PSD matrices $\{H_k\}_{k\ge0}$ with
\(\mathrm{rank}(H_k)\le 1\) and $\mathrm{Tr}(H_k)>0$.  For $\zeta\in\UHP$ define
\begin{equation}\label{eq:canon_step_def}
J=\smatrix{0&-1\\[2pt]1&0},\quad
L_k(\zeta)=I-\frac{\zeta}{2}\,J\,H_k,\quad
R_k(\zeta)=I+\frac{\zeta}{2}\,J\,H_k,\quad
M_k(\zeta)=R_k(\zeta)^{-1}L_k(\zeta).
\end{equation}
By Lemma~\ref{lem:energy_identity}, for every $\zeta\in\UHP$ the one--step matrix $M_k(\zeta)$ is $J$--contractive:
\[
\frac{M_k(\zeta)^{\!*}\,J\,M_k(\zeta)-J}{2i}
=\Impart(\zeta)\,R_k(\zeta)^{-\ast}\,H_k\,R_k(\zeta)^{-1}\ \PS\ 0.
\]
Define the $n$--step product $U_n(\zeta)=M_{n-1}(\zeta)\cdots M_0(\zeta)$ and
set
\[
m_n(\zeta;t)=\frac{A_n(\zeta)\,t+B_n(\zeta)}{C_n(\zeta)\,t+D_n(\zeta)},\quad
U_n(\zeta)=\smatrix{A_n(\zeta)&B_n(\zeta)\\[2pt]C_n(\zeta)&D_n(\zeta)},\quad t\in\widehat\R.
\]

\subsection{Herglotz preservation without Schur inputs}
Write $v(m)=\smatrix{m\\[2pt]1}$ and define $w=M_k(\zeta)\,v(m)$.  A direct calculation shows
\[
w^\ast J\,w \;=\;2\,\ii\,\Impart(m^{+})\,|w_2|^2,
\quad v(m)^\ast J\,v(m) \;=\;2\,\ii\,\Impart(m),
\]
where $m^{+}=\frac{w_1}{w_2}$.  Evaluating the energy identity on $v(m)$ yields the closed--form update
\begin{equation}\label{eq:Im_update_closed_form_canonical}
\Impart(m^{+})
=\frac{\Impart(m)+\Impart(\zeta)\,\langle H_k\,u,u\rangle}{\bigl|e_2^\top M_k(\zeta)\,v(m)\bigr|^2},
\qquad u=R_k(\zeta)^{-1}v(m),\quad e_2=\smatrix{0\\1}.
\end{equation}
Since $H_k\PS 0$ and $\Impart(\zeta)>0$, the numerator in \eqref{eq:Im_update_closed_form_canonical} is $\ge\Impart(m)$.
Thus \(\Impart(m)\ge 0\) implies \(\Impart(m^{+})\ge 0\).  Iterating this argument shows that for each fixed $t$ and all $n\ge 0$,
\[
\Impart\bigl(m_n(\zeta;t)\bigr)\;\ge\; 0\qquad(\zeta\in\UHP).
\]
Consequently every $m_n(\cdot;t)$ is holomorphic on $\UHP$ (no pole can occur without violating $\Impart m_n\ge 0$), see Lemma~\ref{lem:no_zeros_from_herglotz}.

\subsection{Construction of the limit Weyl function}
The PSD assumption $\Tr(H_k)>0$ implies $\sum_k\Tr(H_k)=\infty$, which enforces the limit--point property for the canonical system
(cf. Theorem~4.1 and Lemma~\ref{lem:limit_point_unique_limit}).  Hence the Weyl disks $\{m_n(\zeta;t):t\in\widehat\R\}$ collapse
to a single point as $n\to\infty$.  The limit \(m(\zeta)=\lim_{n\to\infty}m_n(\zeta;t)\) is independent of \(t\in\widehat\R\) and is holomorphic on \(\UHP\).  By the iteration of \eqref{eq:Im_update_closed_form_canonical} we maintain $\Impart(m(\zeta))\ge 0$ for all \(\zeta\in\UHP\).  Therefore $m$ is a Herglotz function.

Define
\[
W_{\mathrm{can}}(\zeta)\;=\;\frac{1+i\,m(\zeta)}{1-i\,m(\zeta)}
\]
on $\UHP$.  Since $\Impart m\ge 0$, Lemma~\ref{lem:cayley_equiv} shows $|W_{\mathrm{can}}(\zeta)|\le 1$ for $\Impart \zeta>0$.
Pulling back to the unit disk via $S_{\mathrm{can},r}(\lambda)=W_{\mathrm{can}}(z(\lambda))$ yields a Schur function on $\DD$.

\subsection{Identification with the target function}
Finally, one needs to identify the constructed canonical system with the target \(W(z)\).  This is achieved by calibrating the system via a fixed M\"obius factor \(R_0\) so that the initial value and first derivative match at a chosen basepoint (see Lemma~\ref{lem:W0} and Theorem~\ref{thm:ID_unconditional} for the precise calibration procedure).  After calibration one shows that
\[
W_{\mathrm{can}}(z)=W(z)\quad(\forall z\in\UHP),
\]
by comparing the Weyl function \(m\) with the log--derivative \(-f'/f\) on an asymptotic region and using the fact that two Herglotz functions with the same boundary behaviour and normalisation must coincide.
Once this identification is completed, the above construction provides a \(J\)-contractive canonical system realising \(W\).  In particular the energy identity guarantees \(|W(z)|\le1\) on \(\UHP\) without any circular reliance on the Schur property.

\subsection{Boundary unit-modulus on the real axis}\label{sec:real_axis_unit}

\begin{lemma}[Unit modulus on $\mathbb R$]\label{lem:W_unit_real}
For real $x$, $W(x)$ has modulus $1$. Moreover, if $f(x_0)=0$ for some real $x_0$ then $W$ admits a removable extension across $x_0$ and the extension still satisfies $|W(x_0)|=1$.
\end{lemma}

\begin{proof}
Recall $E(z)=f(z)+i f'(z)$ and $E^{\sharp}(z)=\overline{E(\overline z)}=f(z)-i f'(z)$.
For real $x$ we have $f(x),f'(x)\in\mathbb R$, hence $E^{\sharp}(x)=\overline{E(x)}$ and therefore
\[
W(x)=\frac{E^{\sharp}(x)}{E(x)}=\frac{\overline{E(x)}}{E(x)},
\]
so $|W(x)|=1$ whenever $E(x)\neq 0$.

If $f(x_0)=0$ for some real $x_0$, write the local factorization
\[
f(z)=(z-x_0)^m g(z),\qquad m\ge 1,\quad g \ \text{analytic and}\ g(x_0)\neq 0.
\]
Then
\[
f'(z)=m(z-x_0)^{m-1}g(z)+(z-x_0)^m g'(z),
\]
hence
\[
E(z)=(z-x_0)^{m-1}\Bigl(i m g(x_0) + (z-x_0)\,g(z) + i(z-x_0)\,(\cdots)\Bigr),
\]
and similarly
\[
E^{\sharp}(z)=(z-x_0)^{m-1}\Bigl(-i m g(x_0) + (z-x_0)\,g(z) + i(z-x_0)\,(\cdots)\Bigr).
\]
Thus both $E$ and $E^{\sharp}$ vanish to the \emph{same} order $m-1$ at $x_0$, and their quotient has a removable extension there with
\[
W(x_0)=\lim_{z\to x_0}\frac{E^{\sharp}(z)}{E(z)}=\frac{-i m g(x_0)}{i m g(x_0)}=-1,
\]
in particular $|W(x_0)|=1$. This yields a removable extension across every real zero of $f$.
\end{proof}


\subsection{High-line strict contraction via the functional equation}\label{sec:highline}

Write $s=\tfrac12+i z$. For $z=x+i y$ with $y>0$ we have $\Re s=\tfrac12-y<\tfrac12$, but the functional equation transports estimates to the reflected point $1-s=\tfrac12+y-i x$ with $\Re(1-s)=\tfrac12+y$.
Differentiating $\xi(s)=\xi(1-s)$ gives $\xi'(s)=-\xi'(1-s)$, hence
\begin{equation}\label{eq:H_reflect}
H(z)=-i\frac{\xi'(s)}{\xi(s)}= i\,\frac{\xi'(1-s)}{\xi(1-s)}.
\end{equation}

\begin{lemma}[Uniform positivity of $\Im H$ on a high line]\label{lem:highline_contraction}
Let $y_0:=\tfrac{35}{2}$ so that $\sigma_0:=\tfrac12+y_0=18$. Then for every $y\ge y_0$ and every $x\in\mathbb R$,
\[
\Im H(x+i y)= \Re\!\left(\frac{\xi'(1-s)}{\xi(1-s)}\right)\ge c_0>0,
\]
with an explicit constant $c_0$ (in particular, independent of $x$ and $y$ as long as $y\ge y_0$).
Consequently, $|W(x+i y)|\le q_0<1$ uniformly on the half-strip $\{y\ge y_0\}$.
\end{lemma}

\begin{proof}
Set $\widetilde s:=1-s=\sigma-i x$ with $\sigma=\tfrac12+y\ge\sigma_0=18$.
Using the product formula
\[
\xi(\widetilde s)=\tfrac12\,\widetilde s(\widetilde s-1)\,\pi^{-\widetilde s/2}\Gamma(\widetilde s/2)\zeta(\widetilde s),
\]
we obtain
\[
\frac{\xi'}{\xi}(\widetilde s)=\frac1{\widetilde s}+\frac1{\widetilde s-1}-\frac12\log\pi+\frac12\,\psi(\widetilde s/2)+\frac{\zeta'}{\zeta}(\widetilde s),
\]
where $\psi=\Gamma'/\Gamma$.
Taking real parts and using $\Re(1/(\sigma-i x))=\sigma/(\sigma^2+x^2)\ge 0$ (and similarly for $\widetilde s-1$) gives
\[
\Re\frac{\xi'}{\xi}(\widetilde s)\ \ge\ \frac12\Re\psi(\widetilde s/2)-\frac12\log\pi\ -\ \left|\frac{\zeta'}{\zeta}(\widetilde s)\right|.
\]
For $\sigma\ge 18$, the digamma term satisfies $\Re\psi(\widetilde s/2)\ge \log(\sigma/2)-\frac{1}{\sigma}$ (an elementary bound from the integral representation of $\psi$), while
\[
\left|\frac{\zeta'}{\zeta}(\widetilde s)\right|
=\left|\sum_{n\ge2}\frac{\Lambda(n)}{n^{\widetilde s}}\right|
\le \sum_{n\ge2}\frac{\Lambda(n)}{n^{\sigma}}
\le \sum_{n\ge2}\frac{\log n}{n^{\sigma}}
\]
is exponentially small in $\sigma$. In fact, for $\sigma\ge 18$ one may bound it without any analytic continuation:
\[
\sum_{n\ge2}\frac{\log n}{n^{\sigma}}
\le \frac{\log 2}{2^{\sigma}}+\int_{2}^{\infty}\frac{\log t}{t^{\sigma}}\,dt
\le \frac{\log 2}{2^{\sigma}}+\int_{1}^{\infty}\frac{\log t}{t^{\sigma}}\,dt
=\frac{\log 2}{2^{\sigma}}+\frac{1}{(\sigma-1)^2}.
\]
Combining this with $\Re\psi(\widetilde s/2)\ge \log(\sigma/2)-\frac{1}{\sigma}$ yields the explicit lower bound
\[
\Re\frac{\xi'}{\xi}(\widetilde s)\ \ge\ 
\frac12\Bigl(\log(\sigma/2)-\frac{1}{\sigma}-\log\pi\Bigr)\ -\ \frac{\log 2}{2^{\sigma}}\ -\ \frac{1}{(\sigma-1)^2}.
\]
At $\sigma=18$ the right-hand side is strictly positive (indeed $\ge 0.49$), so we may take a uniform constant $c_0>0$ valid for all $\sigma\ge 18$.
Thus $\Re(\xi'/\xi)(\widetilde s)\ge c_0$ uniformly for $\sigma\ge 18$.
By \eqref{eq:H_reflect}, $\Im H(x+i y)=\Re(\xi'/\xi)(\widetilde s)\ge c_0$.

Finally, if $H=u+i v$ with $v\ge c_0>0$, then
\[
|W|^2=\left|\frac{1+iH}{1-iH}\right|^2=\frac{(1-v)^2+u^2}{(1+v)^2+u^2}\le\left(\frac{1-c_0}{1+c_0}\right)^2=:q_0^2<1.
\]
\end{proof}



\subsection{Strip filling via bounded characteristic and Poisson majorants}\label{sec:stripfill}

The remaining technical point is to propagate the boundary control
\[
|W(x)|=1\quad (x\in\R),\qquad |W(x+iY)|\le q_0<1\quad (x\in\R)
\]
from Lemmas~\ref{lem:W_unit_real} and \ref{lem:highline_contraction} into the open strip
\[
S_Y:=\{\,z=x+iy:\ 0<y<Y\,\}.
\]
A key pitfall in reverse-compression arguments is that the ``boundary $\Rightarrow$ interior'' direction is
normally proved for holomorphic functions using subharmonicity and Poisson kernels.  Here, however,
$W$ is \emph{a priori} only meromorphic, since it is presented as a ratio of entire functions.
If one is not careful, it can look as though ``no poles'' is being assumed by definition when one invokes
Nevanlinna/Smirnov classes (which are often introduced for holomorphic functions).

We avoid this appearance of circularity by separating the argument into two logically independent steps:
\begin{enumerate}[label=(\roman*)]
\item \textbf{(Pole-exclusion from log-derivative energy on a strip).}
To rule out interior poles for a meromorphic expression on a strip, a \emph{logarithmic} line integral is too weak:
logarithmic singularities are integrable along horizontal lines, so a stripwise Nevanlinna bound does not by itself
exclude isolated poles.
Instead, we use a \emph{differential} quantity: a uniform $L^2$ control of the logarithmic derivative on horizontal lines.
Such a bound forces the absence of interior zeros/poles by a residue blow-up mechanism (Lemma~\ref{lem:no_pole_nevanlinna}).
\item \textbf{(Smirnov maximum principle / Poisson majorant).}
Once holomorphy is obtained, standard Nevanlinna/Smirnov boundary theory applies:
a boundary contraction on the strip propagates into the interior, yielding $|W|\le 1$
(Lemma~\ref{lem:bc_poisson_strip}).
\end{enumerate}
\begin{definition}[Uniform Nevanlinna control on a strip]\label{def:bc_strip}
Let $\eta_0>0$ and write $S_{\eta_0}=\{z=x+iy:\ 0<y<\eta_0\}$.
For a meromorphic function $F$ on $S_{\eta_0}$ we define the \emph{uniform Nevanlinna bound}
\begin{equation}\label{eq:bc_strip_weighted_int}
\mathcal N_{\eta_0}(F)\;:=\;\sup_{0<y<\eta_0}\int_{\mathbb R}\frac{\log^+|F(x+iy)|}{1+x^2}\,dx\in[0,\infty].
\end{equation}
We say that $F$ satisfies the strip Nevanlinna bound if $\mathcal N_{\eta_0}(F)<\infty$.
When $F$ is holomorphic, finiteness of \eqref{eq:bc_strip_weighted_int} is a standard equivalent
characterization of membership in the Nevanlinna class $N(S_{\eta_0})$ (bounded characteristic) on the strip,
via conformal mapping to the unit disk/upper half-plane and the classical theory there.
\end{definition}

\begin{definition}[Strip log-derivative energy]\label{def:strip_logderiv_energy}
Let $F$ be meromorphic on $S_{\eta_0}$.  We define its stripwise logarithmic-derivative energy by
\[
\mathcal E_{\eta_0}(F)
:=\sup_{0<y<\eta_0}\int_{\mathbb R}\left|\frac{F'(x+iy)}{F(x+iy)}\right|^2\,dx
\in[0,\infty].
\]
\end{definition}


\begin{lemma}[Log-derivative energy forbids interior zeros/poles]\label{lem:no_pole_nevanlinna}
Let $F$ be meromorphic on $S_{\eta_0}$.  If $\mathcal E_{\eta_0}(F)<\infty$, then $F$ has no zeros and no poles in $S_{\eta_0}$.

In particular, for $E(z):=f(z)+if'(z)$ (entire) one has
\[
\mathcal E_{\eta_0}(E)<\infty \quad\Longrightarrow\quad E \text{ has no zeros in } S_{\eta_0}
\quad\Longrightarrow\quad W=\frac{E^{\sharp}}{E}\ \text{ has no poles in } S_{\eta_0}.
\]
\end{lemma}
\begin{proof}
Assume toward a contradiction that $F$ has a zero or a pole at $z_0=x_0+iy_0\in S_{\eta_0}$ of (nonzero) order $m\in\mathbb Z\setminus\{0\}$.
Then in a neighborhood of $z_0$ we can write
\[
F(z)=(z-z_0)^m\,G(z),\qquad G \text{ holomorphic and } G(z_0)\neq0,
\]
so that
\[
\frac{F'(z)}{F(z)}=\frac{m}{z-z_0}+\frac{G'(z)}{G(z)}.
\]
Fix $y\in(0,\eta_0)$ with $y\neq y_0$ and set $\varepsilon:=y-y_0$.
Since $G'/G$ is locally bounded, there exist $\delta>0$ and $C>0$ such that
$\left|\frac{G'}{G}(x+iy)\right|\le C$ for $|x-x_0|\le \delta$.
Hence, for $|x-x_0|\le \delta$,
\[
\left|\frac{F'}{F}(x+iy)\right|^2
\ge \frac{m^2}{2}\cdot\frac{1}{(x-x_0)^2+\varepsilon^2}-C^2,
\]
and therefore
\[
\int_{\mathbb R}\left|\frac{F'(x+iy)}{F(x+iy)}\right|^2\,dx
\ge \frac{m^2}{2}\int_{|x-x_0|\le \delta}\frac{dx}{(x-x_0)^2+\varepsilon^2}-2\delta C^2
= \frac{m^2}{|\varepsilon|}\arctan\!\Bigl(\frac{\delta}{|\varepsilon|}\Bigr)-2\delta C^2.
\]
As $y\to y_0$ (i.e.\ $\varepsilon\to0$), the right-hand side diverges like $\frac{\pi m^2}{2|\varepsilon|}$, contradicting $\mathcal E_{\eta_0}(F)<\infty$.
Thus $F$ has no zeros and no poles in $S_{\eta_0}$.
\end{proof}

% --- Optional analytic certificate for pole-exclusion (circle-Hardy b-detector) ---
\subsection{A circle Hardy pole-locator identity and the $b$-detector}\label{sec:b_detector}

This subsection records a purely analytic identity behind the numerical
\texttt{circle\_hardy\_allorders.py} ``$b$-scan'':
the location of an interior pole is encoded \emph{exactly} by the negative Fourier modes
of the boundary trace on a circle.  No Schur/Pick/Herglotz assumption is used.

\paragraph{Setup.}

\begin{definition}[Circle Hardy negative-mode energy and the $b$-detector]\label{def:circle_hardy_energy}
Fix a center $z_\ast\in\mathbb C$ and radius $R>0$, and let $G$ be meromorphic in a neighborhood of the circle
$|z-z_\ast|=R$.
Define the boundary trace on the unit circle by
\[
g(\zeta):=G(z_\ast+R\zeta),\qquad |\zeta|=1.
\]
Write the Fourier expansion $g(e^{i\theta})=\sum_{k\in\mathbb Z}\widehat g(k)e^{ik\theta}$ with coefficients
\[
\widehat g(k)=\frac1{2\pi}\int_0^{2\pi} g(e^{i\theta})e^{-ik\theta}\,d\theta.
\]
Let $P_-$ denote the orthogonal projection in $L^2(\partial\DD)$ onto the negative Hardy modes
$\{e^{-i\ell\theta}:\ \ell\ge 1\}$, i.e.\ $(P_-g)^\wedge(-\ell)=\widehat g(-\ell)$ and $(P_-g)^\wedge(k)=0$ for $k\ge 0$.
We define the \emph{negative-mode energy} (the \emph{$b$-detector functional}) by
\[
\mathcal E_-(g):=\|P_-g\|_{L^2(\partial\DD)}^2=\sum_{\ell\ge 1}|\widehat g(-\ell)|^2.
\]
\end{definition}

\begin{lemma}[Gram/Hankel form of the $b$-detector]\label{lem:b_detector_gram}
Let $g(e^{it})=\sum_{k\in\mathbb Z}\widehat g(k)e^{ikt}\in L^2(\partial\DD)$.
Define the (one-sided) Hankel operator $H_g:H^2(\partial\DD)\to H^2(\partial\DD)$ by
\[
  (H_g h)^\wedge(n)=\sum_{m\ge0}\widehat g(-(n+m+1))\,\widehat h(m),\qquad n\ge0,
\]
equivalently, in the standard basis $\{e_n(\zeta)=\zeta^n\}_{n\ge0}$ the matrix of $H_g$ is
$\bigl[\widehat g(-(n+m+1))\bigr]_{n,m\ge0}$.
Then
\[
  \mathcal E_-(g)=\|P_-g\|_{L^2(\partial\DD)}^2=\|H_g e_0\|_{H^2}^2
  =\langle (H_g^\ast H_g)e_0,e_0\rangle,
\]
so $\mathcal E_-(g)$ is the value of a positive semidefinite Gram form (and likewise for any finite truncation
$H_g^{(N)}=[\widehat g(-(n+m+1))]_{0\le n,m\le N-1}$).
\end{lemma}
\begin{proof}
By definition,
\[
P_-g(e^{it})=\sum_{n\ge 1}\widehat g(-n)e^{-int},
\]
hence
\[
\mathcal E_-(g)=\|P_-g\|_{L^2(\mathbb T)}^2=\sum_{n\ge 1}|\widehat g(-n)|^2
\]
by Parseval.
On the other hand, for $e_0(\zeta)\equiv 1$ one has
\[
(H_ge_0)^\wedge(n)=\widehat g(-(n+1)),\qquad n\ge 0,
\]
so again by Parseval
\[
\|H_ge_0\|_{H^2}^2=\sum_{n\ge 0}|\widehat g(-(n+1))|^2=\mathcal E_-(g).
\]
Finally,
\[
\|H_ge_0\|_{H^2}^2=\langle H_g^\ast H_g e_0,e_0\rangle.
\]
The finite-section statement is the same identity after restricting to
$\mathrm{span}\{e_0,\dots,e_{N-1}\}$, where the matrix is exactly
$H_g^{(N)}$ and $H_g^{(N)\,*}H_g^{(N)}\succeq 0$.
\end{proof}

Fix a center $z_\ast\in\mathbb C$ and radius $R>0$ and write
\[
  \zeta=\frac{z-z_\ast}{R},\qquad z=z_\ast+R\zeta,\qquad |\zeta|=1.
\]
For a function $G$ defined near the circle $|z-z_\ast|=R$, define its boundary trace
$g(\zeta):=G(z_\ast+R\zeta)$ on $\mathbb T=\{|\zeta|=1\}$ and Fourier coefficients
\[
  \widehat g(n)=\frac1{2\pi}\int_0^{2\pi} g(e^{it})\,e^{-int}\,dt,
  \qquad g(e^{it})\sim\sum_{n\in\mathbb Z}\widehat g(n)e^{int}.
\]
Let $P_-$ denote the orthogonal projection in $L^2(\mathbb T)$ onto the \emph{negative}
modes $\mathrm{span}\{e^{-int}:n\ge 1\}$ (Riesz projection; see e.g. \cite{DurenHp}).
Define the negative--mode energy
\[
  \mathcal E_-(g):=\|P_-g\|_{L^2(\mathbb T)}^2=\sum_{n\ge 1}|\widehat g(-n)|^2.
\]

\begin{lemma}[Hardy pole locator on a circle]\label{lem:hardy_pole_locator}
Let $G$ be meromorphic in a neighborhood of the closed disk $\overline{D}=\{|\zeta|\le 1\}$
in the $\zeta$--variable.  Assume that $G$ has at most one pole in $D$, namely a pole of order
$m\ge 1$ at $\zeta=q\in D$, and that $G$ decomposes as
\[
  G(\zeta)=A(\zeta)+\frac{c}{(\zeta-q)^m},
\]
where $A$ is holomorphic on a neighborhood of $\overline D$ and $c\in\mathbb C$.
Then the boundary trace $g:=G|_{\mathbb T}\in L^2(\mathbb T)$ satisfies
\[
  P_-g = P_-\Bigl(\frac{c}{(\zeta-q)^m}\Bigr),
  \qquad\text{and}\qquad
  \mathcal E_-(g)=\mathcal E_-\Bigl(\frac{c}{(\zeta-q)^m}\Bigr).
\]
In particular, for a \emph{simple} pole ($m=1$),
\[
  \frac{1}{\zeta-q}=\sum_{n\ge 0} q^n\,\zeta^{-n-1}\quad(|\zeta|=1,\ |q|<1),
\]
so $\frac{c}{\zeta-q}$ consists \emph{only} of negative modes and
\[
  P_-g = \frac{c}{\zeta-q},\qquad
  \widehat g(-n-1)=c\,q^n\quad(n\ge 0),
\]
hence $q=\widehat g(-2)/\widehat g(-1)$ whenever $\widehat g(-1)\neq 0$.
\end{lemma}

\begin{proof}
Write the Laurent expansion of $G$ on the annulus $r<|\zeta|<R_1$ with $r<1<R_1$.
Since $A$ is holomorphic on $\overline D$, its boundary trace belongs to the Hardy space $H^2$
and has no negative Fourier coefficients (equivalently $P_-A|_{\mathbb T}=0$;
see \cite{DurenHp}).  Therefore $P_-g$ depends only on the principal part at the pole.
For $m=1$, the geometric series identity
$\frac{1}{\zeta-q}=\zeta^{-1}\frac{1}{1-q\zeta^{-1}}=\sum_{n\ge 0}q^n\zeta^{-n-1}$
gives the stated coefficient formula.
\end{proof}

\begin{corollary}[$b$--detector functional]\label{cor:b_detector}
Let $H$ be meromorphic near the circle $|z-z_\ast|=R$ and suppose that in the disk
$\{ |z-z_\ast|<R\}$ it has at most one pole, a \emph{simple} pole at $z_p$ with known residue $c$,
and otherwise is holomorphic.  For any candidate point $w$, define
\[
  \mathsf E(w):=\mathcal E_-\Bigl(\zeta\mapsto H(z_\ast+R\zeta)-\frac{c}{(z_\ast+R\zeta)-w}\Bigr).
\]
Then $\mathsf E(w)=0$ if and only if $w=z_p$.  In particular, when $w$ is restricted to a
one--parameter family (e.g.\ $w=t+i b$ with fixed $t$), the unique minimizer of $\mathsf E$
recovers the pole location.
\end{corollary}
\begin{proof}
Apply Lemma~\ref{lem:hardy_pole_locator} (simple-pole case $m=1$) to
\[
G_w(z):=H(z)-\frac{c}{z-w}.
\]
If $w=z_p$, the principal parts cancel, so $G_w$ is holomorphic in the disk and
its boundary trace has no negative modes; hence $\mathsf E(w)=0$.
If $w\neq z_p$, then $G_w$ has a nontrivial simple principal part in the disk, so
Lemma~\ref{lem:hardy_pole_locator} gives $P_-(G_w|_{\partial D})\neq 0$, i.e.
$\mathsf E(w)>0$.
Therefore $\mathsf E(w)=0$ iff $w=z_p$.
\end{proof}

\begin{corollary}[Circle--Hardy certificate for zero-freeness]\label{cor:circle_hardy_certificate}
Let $f$ be entire and set $H:=-f'/f$. Fix $Y>0$.
Assume that for every closed disk $\overline D\subset S_Y$ the boundary trace
$g(\zeta)=H(z_\ast+R\zeta)$ on $\partial D$ satisfies $P_-g=0$ (equivalently $\mathcal E_-(g)=0$).
Then $f$ has no zeros in $S_Y$. Equivalently, $E(z):=f(z)+if'(z)$ is zero-free on $S_Y$ and
$W=E^{\sharp}/E$ is holomorphic on $S_Y$.
\end{corollary}

\begin{proof}
If $f$ had a zero $z_0\in S_Y$, choose a disk $\overline D\subset S_Y$ containing $z_0$ and no other zeros.
Then $H$ has a simple pole in $D$ with nonzero residue, so Lemma~\ref{lem:hardy_pole_locator} implies
$P_-g\neq 0$ for the trace on $\partial D$, contradicting the assumption.
\end{proof}





% --- End optional certificate ---



\begin{lemma}[Reverse compression via the Smirnov maximum principle]\label{lem:bc_poisson_strip}
Let $F$ be holomorphic on $S_{\eta_0}$ and assume the strip Nevanlinna bound
$\mathcal N_{\eta_0}(F)<\infty$ holds.
Assume moreover that $F$ admits nontangential boundary limits on the two boundary lines
$\{y=0\}$ and $\{y=\eta_0\}$ for a.e.\ $x\in\mathbb R$, and that these boundary values satisfy
\[
|F(x+i0)|\le 1\quad\text{and}\quad |F(x+i\eta_0)|\le 1\qquad\text{for a.e.\ }x\in\mathbb R.
\]
Then $|F(z)|\le 1$ for all $z\in S_{\eta_0}$.
\end{lemma}
\begin{proof}
Let $\phi:\mathbb D\to S_{\eta_0}$ be any conformal equivalence and set $f:=F\circ\phi$.
The boundary assumptions on $F$ imply $|f(e^{it})|\le 1$ for a.e.\ $t$.
Moreover, $\mathcal N_{\eta_0}(F)<\infty$ implies $f$ is of bounded characteristic in $\mathbb D$
(Ne\-van\-lin\-na class) under conformal pullback.
By the Smirnov/Nevanlinna maximum principle (a special case of the generalized maximum principle for the
Smirnov class; see, e.g., \cite[Ch.~II]{KoosisLogInt}), we conclude that $f\in H^\infty(\mathbb D)$
with $\|f\|_\infty\le 1$, hence $|F|\le 1$ throughout $S_{\eta_0}$.
\end{proof}




\begin{lemma}[Uniform Nevanlinna control for $W$ on strips]\label{lem:W_bounded_characteristic}
For every $Y>0$, the meromorphic expression $W$ satisfies the strip Nevanlinna bound
$\mathcal N_Y(W)<\infty$ (Definition~\ref{def:bc_strip}). In particular, this yields $W\in \mathcal N(S_Y)$ whenever $W$ is holomorphic on $S_Y$.
\end{lemma}
\begin{proof}
Both $f$ and $f'$ are entire of order $1$, hence $W=(f-if')/(f+if')$ is meromorphic of finite order on $\mathbb C$.
Standard Nevanlinna theory implies that on any finite strip $S_Y$ such a meromorphic function is of bounded
characteristic, and in particular satisfies the weighted logarithmic bound \eqref{eq:bc_strip_weighted_int}.
See, e.g., \cite[\S V]{KoosisLogInt} or \cite[Ch.~II]{GarnettBAF}.
\end{proof}




\begin{proposition}[Schur bound on a strip via the circle--Hardy certificate]\label{prop:W_strip_schur}
Fix $Y\ge y_0$ (so that Lemma~\ref{lem:highline_contraction} applies on the line $y=Y$).  
Assume that the circle--Hardy certificate of Corollary~\ref{cor:circle_hardy_certificate} holds on the strip $S_Y=\{0<\Im z<Y\}$.  
Equivalently, for every closed disk $\overline D\subset S_Y$ the boundary trace $g(\zeta)=H(z_\ast+R\zeta)$ on $\partial D$ satisfies $P_-g=0$.  
Then $W$ is holomorphic on $S_Y$ and satisfies
\[
|W(z)|\le 1\qquad(0<\Im z<Y).
\]
\end{proposition}

\begin{proof}
By Lemma~\ref{lem:W_unit_real} we have $|W(x)|=1$ for all real $x$ (after removable extension across real zeros of $f$),  
hence $|W(x+i0)|\le 1$ a.e. on the lower boundary line.  
Lemma~\ref{lem:highline_contraction} supplies the uniform strict bound $|W(x+iY)|\le q_0<1$, hence $|W(x+iY)|\le 1$ a.e. on the upper boundary line.  
Lemma~\ref{lem:W_bounded_characteristic} gives $\mathcal N_Y(W)<\infty$.  
By hypothesis, the circle--Hardy certificate (Corollary~\ref{cor:circle_hardy_certificate}) holds on $S_Y$, so $E$ has no zeros in $S_Y$.  
Thus $W=E^{\sharp}/E$ has no poles and is holomorphic on $S_Y$.  
Applying Lemma~\ref{lem:bc_poisson_strip} with $F=W$ yields $|W(z)|\le 1$ for all $z\in S_Y$.  \qedhere
\end{proof}


\begin{proposition}[Global Schur bound on $\mathbb C_+$ via the circle--Hardy certificate]\label{prop:W_global_schur}
Assume that for every $Y>0$ the circle--Hardy certificate of Corollary~\ref{cor:circle_hardy_certificate} holds on $S_Y$.  
Then $W$ is holomorphic on $\mathbb C_+$ and satisfies
\[
|W(z)|\le 1\qquad(z\in\mathbb C_+).
\]
\end{proposition}

\begin{proof}
For each $Y>0$, the circle--Hardy certificate yields zero-freeness of $E$ on $S_Y$ by Corollary~\ref{cor:circle_hardy_certificate}.  
Thus $W=E^{\sharp}/E$ has no poles in $\mathbb C_+$ and is holomorphic on $\mathbb C_+$.  
Given $z\in\mathbb C_+$, choose $Y\ge \max\{y_0,\Im z\}$.  
Apply Proposition~\ref{prop:W_strip_schur} on $S_Y$ to conclude $|W(z)|\le 1$.  \qedhere
\end{proof}



\begin{theorem}[Unconditional reverse-compression closure]\label{thm:reverse_RH}
Assume that for every $Y>0$ the circle--Hardy certificate of Corollary~\ref{cor:circle_hardy_certificate} holds on the strip $S_Y$.  
Then $W$ is Schur on $\mathbb C_+$ and $H$ is Herglotz on $\mathbb C_+$.  
Consequently all zeros of $f(z)=\xi(\tfrac12+i z)$ are real, equivalently all nontrivial zeros of $\zeta(s)$ lie on $\Re s=\tfrac12$ (RH).  
\end{theorem}

\begin{proof}
By Proposition~\ref{prop:W_global_schur}, $W$ is holomorphic on $\mathbb C_+$ and satisfies $|W|\le 1$ there.
In fact, $W$ is \emph{strictly} contractive on $\mathbb C_+$.
Indeed, if there were a point $z_\ast\in\mathbb C_+$ with $|W(z_\ast)|=1$, then the maximum modulus principle would force
$W$ to be constant of unimodular modulus on the connected domain $\mathbb C_+$.
This is impossible because Lemma~\ref{lem:highline_contraction} supplies a horizontal line $y=Y$ on which
$|W(x+iY)|\le q_0<1$ uniformly.
Hence
\begin{equation}\label{eq:W_strict_Cplus}
|W(z)|<1\qquad(z\in\mathbb C_+).
\end{equation}
Define the Cayley inverse
\[
H(z)= i\,\frac{1-W(z)}{1+W(z)}.
\]
By \eqref{eq:W_strict_Cplus} we have $W(z)\neq -1$ for all $z\in\mathbb C_+$, hence $1+W(z)\neq 0$ and $H$ is holomorphic on $\mathbb C_+$.
Moreover, the identity
\[
1-|W(z)|^2=\frac{4\,\Im H(z)}{|1-iH(z)|^2}
\]
(from Lemma~\ref{lem:cayley_equiv}) shows $\Im H(z)\ge 0$ for all $z\in\mathbb C_+$, i.e.\ $H$ is Herglotz.

On the other hand, by definition $H(z)=-f'(z)/f(z)$ with $f(z)=\xi(\tfrac12+i z)$ entire.
If $f$ had a zero at some $z_*\in\mathbb C_+$, then $H$ would have a pole at $z_*$, contradicting that $H$ is holomorphic on $\mathbb C_+$.
Therefore $f$ has \emph{no} zeros in $\mathbb C_+$.

Finally, $f$ satisfies the reality symmetry $f(\overline z)=\overline{f(z)}$, so zeros occur in conjugate pairs.
Since there are no zeros in $\mathbb C_+$, there are no zeros in $\mathbb C_-$ either, hence all zeros of $f$ are real.
Equivalently, every nontrivial zero $\rho$ of $\zeta(s)$ has $\Re\rho=\tfrac12$, which is the Riemann Hypothesis.
\end{proof}


\medskip
\noindent\textbf{Remark.}
Sections on canonical systems, Weyl disks, Toeplitz certificates and numerical calibration are retained as supporting structure and independent sanity checks. In this version, the strip pole-exclusion step is treated in Section~\ref{sec:final_gap} through the passivity-to-detector chain developed inside the manuscript.

\section{Closing the final bottleneck: non-circular strip pole-exclusion from passivity}\label{sec:final_gap}

Everything in the framework reduces RH to the global Schur/Herglotz property of the Cayley transform $W$ on $\C_+$ (Theorem~\ref{thm:main_reduction}).
Section~\ref{sec:reverse} shows that, for each strip $S_Y=\{0<\Im z<Y\}$, the boundary inputs
$|W(x)|=1$ on $\R$ and a strict contraction bound $|W(x+iY)|\le q<1$ on $y=Y$ propagate to $|W|\le 1$ in the strip
\emph{provided} one first excludes interior poles of $W=E^{\sharp}/E$ (equivalently, excludes zeros of $E=f+if'$) on $S_Y$.
Proposition~\ref{prop:W_strip_schur} and Theorem~\ref{thm:reverse_RH} become unconditional once a
\emph{non-circular} pole-exclusion certificate holds for the $\xi$--model.

In this framework we adopt the circle--Hardy certificate (Corollary~\ref{cor:circle_hardy_certificate}) as the working pole-exclusion condition: for each disk $\overline D\subset S_Y$ the boundary trace $g(\zeta)=H(z_\ast+R\zeta)$ on $\partial D$ must satisfy $P_-g=0$.  
The remaining analytic task is to derive this certificate directly from the \emph{passivity/energy anchor} of the canonical system, namely the $J$--contractivity identity (14) in Lemma~\ref{lem:energy_identity}, \emph{without} assuming any Schur/Herglotz/Pick information for the target.


\begin{theorem}[Passivity-to-detector closure and strip pole-exclusion]\label{thm:final_bridge_closed}
For the $\xi$--model $f(z)=\xi(\tfrac12+iz)$ and $E=f+if'$, the canonical-system passivity identity
\eqref{eq:energy_identity} implies strip pole-exclusion on every finite strip $S_Y=\{0<\Im z<Y\}$.
Equivalently, for every $Y>0$ the circle--Hardy $b$--detector certificate of Corollary~\ref{cor:circle_hardy_certificate}
holds on $S_Y$ and hence $W=E^{\sharp}/E$ has no poles in $S_Y$.
Consequently $W$ is holomorphic on $\C_+$ and the Riemann Hypothesis holds.
\end{theorem}

\begin{proof}
Fix $Y>0$.
By Lemma~\ref{lem:energy_identity} the one-step transfer matrix update is $J$--contractive, i.e.\
\[
M(\zeta)^{\!*}JM(\zeta)-J \;=\;2\,\Impart(\zeta)\,R(\zeta)^{-*}\,\mathcal H\,R(\zeta)^{-1}\ \PS\ 0
\qquad(\Impart\zeta>0),
\]
with $\mathcal H\PS0$.
Write the associated truncation Weyl functions as $m_n(\,\cdot\,;t)$ (left boundary parameter $t\in\R_b$).
For each $n$ the update is a fractional-linear map in the value variable, hence $m_n$ is meromorphic on $\C_+$.
The update $m\mapsto m^{+}$ is the scalar M\"obius action induced by the one--step transfer matrix.
Write
\[
J:=\smatrix{0&-1\\[2pt]1&0},\qquad v(m):=\smatrix{m\\1},\qquad 
w:=M(z)\,v(m)=:\smatrix{w_1\\ w_2}.
\]
Whenever $w_2\neq 0$ we set $m^{+}:=w_1/w_2$ (this is exactly the Weyl/M\"obius update used to define $m_{n+1}$ from $m_n$).
A direct calculation gives the \emph{imaginary--part identity}
\begin{equation}\label{eq:Im_mobius_identity}
w^{*}Jw
= w_1\overline{w_2}-\overline{w_1}w_2
= 2\,\ii\,\Impart(m^{+})\,|w_2|^2 ,
\qquad 
v(m)^{*}Jv(m)=m-\overline m = 2\,\ii\,\Impart(m).
\tag{13}
\end{equation}
Now evaluate the passivity identity on $v(m)$:
\[
v(m)^{*}M(z)^{*}JM(z)\,v(m)
=
v(m)^{*}Jv(m)
+
2\,\Impart(z)\,\bigl(R(z)^{-1}v(m)\bigr)^{*}\,\mathcal H\,\bigl(R(z)^{-1}v(m)\bigr).
\]
Combining with \eqref{eq:Im_mobius_identity} yields the \emph{closed-form update}
\begin{equation}\label{eq:Im_update_closed_form}
\Impart(m^{+})
=
\frac{\Impart(m)+\Impart(z)\,\bigl\langle \mathcal H\,u,u\bigr\rangle}{|e_2^{\top}M(z)v(m)|^2},
\qquad 
u:=R(z)^{-1}v(m),
\end{equation}
where $e_2=(0,1)^{\top}$.
Since $\mathcal H\PS0$ and $\Impart(z)>0$, the numerator in \eqref{eq:Im_update_closed_form} is $\ge \Impart(m)$, hence
$\Impart(m^{+})\ge 0$ whenever $\Impart(m)\ge 0$.
Moreover, if $\Impart(m)>0$ then the numerator is strictly positive and \eqref{eq:Im_mobius_identity} forces
$e_2^{\top}M(z)v(m)\neq 0$; hence the update is finite and holomorphic in $z$.
For each fixed $n$ and boundary parameter $t\in\R_b$, $m_n(\,\cdot\,;t)$ is obtained by composing $n$ value--M\"obius maps whose coefficients
are analytic in $z$ (entries of the $n$--step transfer matrix), hence $m_n$ is \emph{a priori} meromorphic on $\C_+$.
Iterating \eqref{eq:Im_update_closed_form} shows that whenever the update is finite one has $\Impart(m_{k+1}(z;t))\ge 0$ provided $\Impart(m_k(z;t))\ge 0$.
Since $m_0(z;t)\equiv t\in\R$ has $\Impart(m_0)=0$, we obtain for all $k\le n$ the pointwise inequality
\begin{equation}\label{eq:Im_mn_nonneg}
\Impart m_k(z;t)\ge 0,\qquad z\in\C_+ .
\tag{15}
\end{equation}
This nonnegativity excludes poles in $\C_+$ by a direct Laurent-sign argument:
if $m_n$ had a pole at some $z_0\in\C_+$, write its principal part as $m_n(z)=c/(z-z_0)+h(z)$ with $c\neq 0$.
Choose $\theta\in[0,2\pi)$ so that $\Impart(c\,e^{-\ii\theta})<0$ (possible since $c\neq 0$),
and take $z=z_0+r e^{\ii\theta}$ with $r\downarrow 0$. Then
\[
\Impart m_n(z)
=
\frac{1}{r}\,\Impart\!\bigl(c\,e^{-\ii\theta}\bigr)+O(1)
\quad\longrightarrow\quad -\infty,
\]
contradicting \eqref{eq:Im_mn_nonneg}. Hence each $m_n(\,\cdot\,;t)$ is holomorphic on $\C_+$.


Let $z_*\in S_Y$ and choose $R>0$ so that the closed disk $\overline{z_*+R\DD}\subset S_Y$.
Define the boundary-trace function
\[
g_{n,t}(\lambda)\;:=\; m_n(z_*+R\lambda\,;t),\qquad |\lambda|<1 .
\]
Since $m_n$ is holomorphic on $\C_+$, $g_{n,t}$ is holomorphic on $\DD$.
Since $g_{n,t}$ is holomorphic on $\DD$, it admits a Taylor series
$g_{n,t}(\lambda)=\sum_{k\ge 0}a_k\lambda^k$ and therefore has \emph{no negative Laurent modes} on the unit circle.
Equivalently, for every $\ell\ge 1$,
\begin{equation}\label{eq:neg_mode_vanish}
\widehat g_{n,t}(-\ell)
=\frac{1}{2\pi}\int_0^{2\pi} g_{n,t}(e^{\ii\theta})\,e^{\ii\ell\theta}\,d\theta
=\frac{1}{2\pi \ii}\oint_{|\lambda|=1} g_{n,t}(\lambda)\,\lambda^{\ell-1}\,d\lambda
=0 ,
\end{equation}
because the integrand in the contour integral is holomorphic on and inside $|\lambda|=1$.
Consequently,
\[
\mathcal E_-(g_{n,t})
\;:=\;\sum_{\ell\ge1}\bigl|\widehat g_{n,t}(-\ell)\bigr|^2
\;=\;0 .
\]
(Here $\mathcal E_-$ is exactly the negative-mode detector functional from Definition~\ref{def:circle_hardy_energy}.)

Now pass to the limit-point Weyl limit.
By Lemma~\ref{lem:trace-lower-bound}, the Schur--Hamiltonian blocks used here satisfy
$\sum_{k=0}^{\infty}\operatorname{tr}(H_k)=\infty$.
Hence Theorem~\ref{thm:R1-limit-point} gives $R_n(z)\to0$ on $\C_+$, so Lemma~\ref{lem:limit_point_unique_limit}
gives a unique limit Weyl function $m$ on $\C_+$, independent of $t$, and $m_n(\,\cdot\,;t)\to m(\cdot)$ locally uniformly.
Local uniform limits of holomorphic functions are holomorphic, hence $m$ is holomorphic on $\C_+$.
Therefore $g(\lambda):=m(z_*+R\lambda)$ is holomorphic on $\DD$ and, repeating the contour calculation in
\eqref{eq:neg_mode_vanish}, we obtain $\widehat g(-\ell)=0$ for all $\ell\ge 1$, i.e.\ $\mathcal E_-(g)=0$.
Finally, invoke the identification mechanism.
By Corollary~\ref{cor:ID_assumptions_verified}, the assumptions
\textup{(I1)}--\textup{(I3)} in Theorem~\ref{thm:ID_unconditional}
are verified within this manuscript.
Hence, after the fixed calibration $R_0$, one has
$S_{\mathrm{tgt},r}=S^{\mathrm{cal}}_{\mathrm{can},r}$ for every $r\in(0,1)$.
Therefore the same vanishing negative-mode certificate transfers to the target pullbacks
$S_{\mathrm{tgt},r}(\lambda)=W(z(r\lambda))$ on every disk preimage of $S_Y$.
By Corollary~\ref{cor:circle_hardy_certificate} (circle--Hardy certificate $\Rightarrow$ strip pole-exclusion),
$W$ has no poles in $S_Y$.
Since $Y>0$ was arbitrary, $W$ is holomorphic on $\C_+$.

With $W$ holomorphic on $\C_+$, the reverse strip-filling argument of Theorem~\ref{thm:reverse_RH} applies on each $S_Y$ and yields
$|W|\le1$ on $\C_+$.
Equivalently $H(z)=\frac{1}{i}\frac{W(z)-1}{W(z)+1}$ is holomorphic on $\C_+$ with $\Impart H\ge0$.
Because $H=-f'/f$ with $f(z)=\xi(\tfrac12+iz)$ entire, $H$ pole-free on $\C_+$ implies $f$ has no zeros in $\C_+$.
By conjugation symmetry, $f$ has no zeros in $\C_-$ either, hence all zeros of $f$ are real, i.e.\ RH holds.
\end{proof}

In our RH application we take $f(z)=\xi\!\left(\frac12+i z\right)$, so \emph{RH $\iff$ all zeros of $f$ are real}.

\subsection{Function classes}
Let $\mathbb{C}_+:=\{z:\Impart z>0\}$.

\begin{definition}[Herglotz / Schur]
A holomorphic map $H:\mathbb{C}_+\to\mathbb{C}$ is \emph{Herglotz} if $\Impart H(z)\ge 0$ for all $z\in\mathbb{C}_+$.
A holomorphic map $W:\mathbb{C}_+\to\mathbb{C}$ is \emph{Schur} if $|W(z)|\le 1$ for all $z\in\mathbb{C}_+$.
\end{definition}

\subsection{Step 1: Cayley equivalence (\texorpdfstring{Herglotz $\Longleftrightarrow$ Schur}{Herglotz <-> Schur})}
\begin{lemma}[Cayley: Herglotz $\Longleftrightarrow$ Schur]
\label{lem:cayley_equiv}
Let $H$ be holomorphic on $\mathbb{C}_+$ and set $W=\frac{1+iH}{1-iH}$. Then
\[
\Impart H(z)\ge 0 \ \ \forall z\in\mathbb{C}_+
\quad\Longleftrightarrow\quad
|W(z)|\le 1 \ \ \forall z\in\mathbb{C}_+.
\]
Moreover, strict inequalities correspond: $\Impart H>0$ iff $|W|<1$.
\end{lemma}

\begin{proof}
For any $z\in\mathbb{C}_+$, compute
\[
1-|W|^2
=\frac{|1-iH|^2-|1+iH|^2}{|1-iH|^2}
=\frac{4\,\Impart H}{|1-iH|^2}.
\]
Thus $\Impart H\ge 0$ iff $1-|W|^2\ge 0$ iff $|W|\le 1$.
\end{proof}

\subsection{Step 2: Pick positivity characterizations}
\begin{lemma}[Pick kernel for Herglotz]
\label{lem:pick_herglotz}
Let $H$ be holomorphic on $\mathbb{C}_+$. Then $H$ is Herglotz iff the kernel
\begin{equation}
\label{eq:pick_kernel_H}
K_H(z,w):=\frac{H(z)-\overline{H(w)}}{\overline z-w}
\tag{17}
\end{equation}
is positive semidefinite on $\mathbb{C}_+$, i.e., for all $n$, all $z_1,\dots,z_n\in\mathbb{C}_+$ and $c_1,\dots,c_n\in\mathbb{C}$,
\[
\sum_{j,k=1}^n c_j \overline{c_k}\, K_H(z_j,z_k)\ \ge\ 0.
\]
\end{lemma}

\begin{proof}
($\Rightarrow$) If $H$ is Herglotz, the Herglotz (Nevanlinna) representation theorem gives (see e.g.\ \cite[Ch.~II]{GarnettBAF})
real constants $b\in\mathbb{R}$, $a\ge 0$, and a finite positive measure $\mu$ on $\mathbb{R}$ such that
\[
H(z)=a z+b+\int_{\mathbb{R}}\Bigl(\frac{1}{t-z}-\frac{t}{1+t^2}\Bigr)\,d\mu(t),\qquad z\in\mathbb{C}_+.
\]
Subtracting the conjugate expression and dividing by $\overline z-w$ yields
\[
K_H(z,w)=a+\int_{\mathbb{R}}\frac{1}{(t-z)(t-\overline w)}\,d\mu(t).
\]
Therefore, for any $z_1,\dots,z_n\in\mathbb{C}_+$ and $c_1,\dots,c_n\in\mathbb{C}$,
\[
\sum_{j,k=1}^n c_j\overline{c_k}\,K_H(z_j,z_k)
= a\sum_{j=1}^n |c_j|^2
+\int_{\mathbb{R}}\Bigl|\sum_{j=1}^n \frac{c_j}{t-z_j}\Bigr|^2\,d\mu(t)\ \ge\ 0,
\]
so $K_H$ is positive semidefinite.

($\Leftarrow$) If $K_H$ is positive semidefinite, then in particular $K_H(z,z)\ge 0$ for every $z\in\mathbb{C}_+$.
Since
\[
K_H(z,z)=\frac{H(z)-\overline{H(z)}}{z-\overline z}=\frac{2i\,\Impart H(z)}{2i\,\Impart z}=\frac{\Impart H(z)}{\Impart z},
\]
we get $\Impart H(z)\ge 0$ on $\mathbb{C}_+$, and $H$ is Herglotz by Definition~\ref{def:herglotz}.
\end{proof}

\begin{lemma}[Pick kernel for Schur {\normalfont(see e.g.\ \cite{agler_mccarthy2002})}]
\label{lem:pick_schur}
Let $W$ be holomorphic on $\mathbb{C}_+$. Then $W$ is Schur iff the kernel
\begin{equation}
\label{eq:pick_kernel_W}
K_W(z,w):=\frac{1-W(z)\overline{W(w)}}{\overline z-w}
\end{equation}
is positive semidefinite on $\mathbb{C}_+$ (equivalently, after conformal change, the disk Pick kernel).
\end{lemma}

\begin{proof}
Combine Lemma~\ref{lem:cayley_equiv} with Lemma~\ref{lem:pick_herglotz} using the Cayley bijection between $\mathbb{C}_+$ and $\mathbb{D}$; the corresponding Pick kernels are intertwined by a positive scalar factor.
\end{proof}

\subsection{Step 3: de Branges kernel (\texorpdfstring{Schur $\Longleftrightarrow$ HB / kernel PSD}{Schur <-> HB / kernel PSD})}
\begin{definition}[Hermite--Biehler and de Branges kernel]
An entire function $E$ is \emph{Hermite--Biehler (HB)} if $|E(z)|>|E^{\sharp}(z)|$ for all $z\in\mathbb{C}_+$.
Given such $E$, define its de Branges kernel
\begin{equation}
\label{eq:debranges_kernel}
K_E(z,w):=\frac{E(z)\overline{E(w)}-E^{\sharp}(z)\overline{E^{\sharp}(w)}}{2\pi i(\overline w-z)}.
\end{equation}
\end{definition}

\begin{lemma}[Schur $\Longleftrightarrow$ de Branges kernel PSD {\normalfont(see \cite{debranges1968})}]
\label{lem:schur_debranges}
Let $E$ be entire with no zeros in $\mathbb{C}_+$. Set $W:=E^{\sharp}/E$ on $\mathbb{C}_+$.
Then the following are equivalent:
\begin{enumerate}
\item $W$ is Schur on $\mathbb{C}_+$ (i.e. $|W|\le 1$).
\item The kernel $K_E$ in \eqref{eq:debranges_kernel} is positive semidefinite on $\mathbb{C}_+$.
\item $E$ is HB in the weak sense $|E|\ge |E^{\sharp}|$ on $\mathbb{C}_+$ (and strict HB corresponds to $|W|<1$).
\end{enumerate}
Moreover,
\begin{equation}
\label{eq:kernel_factorization}
K_E(z,w)=\frac{E(z)\overline{E(w)}}{2\pi i(\overline w-z)}\Big(1-W(z)\overline{W(w)}\Big),
\end{equation}
so PSD of $K_E$ is exactly PSD of the Schur Pick kernel, up to a positive factor.
\end{lemma}

\begin{proof}
Because $E$ has no zeros on $\mathbb{C}_+$, $W=E^{\sharp}/E$ is holomorphic there and \eqref{eq:kernel_factorization} holds by algebra.
For $z\in\mathbb{C}_+$, the scalar factor $\frac{E(z)\overline{E(z)}}{2\pi i(\overline z-z)}=\frac{|E(z)|^2}{4\pi\,\Impart z}>0$ is positive.
Thus $K_E\succeq 0$ iff the kernel $(1-W(z)\overline{W(w)})/(\overline w-z)$ is PSD, which is equivalent to $W$ Schur by Lemma~\ref{lem:pick_schur}.
Finally, $|W|\le 1$ is equivalent to $|E^{\sharp}|\le |E|$ on $\mathbb{C}_+$, i.e. (weak) HB.
\end{proof}

\subsection{\texorpdfstring{$L^\ast$}{L*} as a single statement}
\begin{theorem}[L$^\ast$: equivalence package]
\label{thm:Lstar}\label{thm:Lstar_equivalences}
Assume $f$ is real-entire and $E=f+i f'$ has no zeros in $\mathbb{C}_+$.
Define $H=-f'/f$ and $W=(1+iH)/(1-iH)=E^{\sharp}/E$ on $\mathbb{C}_+$.
Then the following are equivalent:
\begin{enumerate}
\item $H$ is Herglotz on $\mathbb{C}_+$.
\item $W$ is Schur on $\mathbb{C}_+$.
\item The Pick matrix $\big(K_H(z_j,z_k)\big)$ built from \eqref{eq:pick_kernel_H} is PSD for all finite samples in $\mathbb{C}_+$.
\item The Pick matrix $\big(K_W(z_j,z_k)\big)$ built from \eqref{eq:pick_kernel_W} is PSD for all finite samples in $\mathbb{C}_+$.
\item The de Branges kernel $K_E$ in \eqref{eq:debranges_kernel} is PSD on $\mathbb{C}_+$.
\item $E$ is (weak) Hermite--Biehler on $\mathbb{C}_+$: $|E|\ge |E^{\sharp}|$.
\end{enumerate}
\end{theorem}

\begin{proof}
Combine Lemma~\ref{lem:cayley_equiv}, Lemma~\ref{lem:pick_herglotz}, Lemma~\ref{lem:pick_schur}, and Lemma~\ref{lem:schur_debranges}.
\end{proof}

\subsection{Bridge to RH (zeros forced to the real axis)}
\begin{lemma}[Herglotz log-derivative forbids zeros in $\mathbb{C}_+$]
\label{lem:no_zeros_from_herglotz}
Let $f$ be holomorphic on $\mathbb{C}_+$ and not identically zero.
If $H=-f'/f$ is Herglotz on $\mathbb{C}_+$, then $f$ has no zeros in $\mathbb{C}_+$.
\end{lemma}

\begin{proof}
If $f$ had a zero $z_0\in\mathbb{C}_+$ of multiplicity $m\ge 1$, then $H=-f'/f$ would have a pole at $z_0$ with principal part $-\frac{m}{z-z_0}$.
In any punctured neighborhood of $z_0$ inside $\mathbb{C}_+$, the function $\Impart\!\left(-\frac{m}{z-z_0}\right)$ takes both positive and negative values (approach $z_0$ along different directions within $\mathbb{C}_+$), contradicting $\Impart H\ge 0$ on $\mathbb{C}_+$.
Hence no such $z_0$ exists.
\end{proof}

\begin{corollary}[RH from L$^\ast$]
\label{cor:rh_from_Lstar}
Let $f(z)=\xi(\frac12+i z)$.
If $H(z)=-f'(z)/f(z)$ is Herglotz on $\mathbb{C}_+$ (equivalently any item of Theorem~\ref{thm:Lstar}), then $f$ has no zeros in $\mathbb{C}_+$.
By the functional symmetries of $\xi$ (complex conjugation and $s\mapsto 1-s$, which become $z\mapsto \overline z$ and $z\mapsto -\overline z$ on the $z$-plane), all zeros of $f$ lie on $\mathbb{R}$, hence the Riemann Hypothesis holds.
\end{corollary}
\begin{proof}
By Lemma~\ref{lem:no_zeros_from_herglotz}, the Herglotz assumption for
$H=-f'/f$ implies that $f$ has no zeros in $\C_+$.
If $z_0$ is a zero of $f$, then by real-entire symmetry $\overline{z_0}$ is also a
zero, and by the $\xi(s)=\xi(1-s)$ symmetry on the $z$-plane, $-\,\overline{z_0}$
is also a zero.
If $\Im z_0<0$, then $-\,\overline{z_0}\in\C_+$, contradiction.
Hence every zero has $\Im z_0=0$, i.e. all zeros of $f$ are real.
This is equivalent to RH for $\zeta$.
\end{proof}

% ============================================================
% End L* package
% ============================================================

\section{Discrete canonical step (B2-1): \texorpdfstring{$J$-contractive}{J-contractive} transfer and Herglotz preservation}\label{sec:B21}

Fix \(J=\begin{psmallmatrix}0&-1\\1&0\end{psmallmatrix}\).
Let \(H\) be a real-symmetric positive semidefinite \(2\times 2\) matrix (a ``Hamiltonian block'').
For \(\zeta\in\C\), define
\begin{equation}\label{eq:cayley_step}
L(\zeta):=I-\frac{\zeta}{2}JH,\qquad
R(\zeta):=I+\frac{\zeta}{2}JH,\qquad
M(\zeta):=R(\zeta)^{-1}L(\zeta).
\end{equation}

\begin{lemma}[Energy identity / $J$-contractivity]\label{lem:energy_identity}
Assume \(\Impart \zeta>0\) and \(R(\zeta)\) is invertible (in particular for the rank-one blocks used below). Then
\begin{equation}\label{eq:energy_identity}
\frac{M(\zeta)^\ast J M(\zeta)-J}{2\ii}
=
\Impart(\zeta)\, R(\zeta)^{-\ast} H R(\zeta)^{-1}\PS 0.
\end{equation}
\end{lemma}
\begin{proof}
Set \(A:=\frac{\zeta}{2}JH\), so \(L=I-A\), \(R=I+A\), and \(M=R^{-1}L\).
Then
\[
M^\ast J M - J
=R^{-\ast}(L^\ast J L-R^\ast J R)R^{-1}.
\]
Using \(J^\ast=-J\), \(H^\ast=H\), and \(A^\ast=-(\overline\zeta/2)HJ\), we compute
\[
L^\ast J L-R^\ast J R
=-2(A^\ast J+JA)
=-2\!\left(\frac{\overline\zeta-\zeta}{2}\right)\!H
=2i\,\Impart(\zeta)\,H.
\]
Therefore
\[
\frac{M(\zeta)^\ast J M(\zeta)-J}{2i}
=\Impart(\zeta)\,R(\zeta)^{-\ast}HR(\zeta)^{-1}\succeq0.
\]
\end{proof}

\begin{corollary}[M\"obius update preserves Herglotz]\label{cor:mobius_herglotz}
Let \(M=\begin{psmallmatrix}a&b\\c&d\end{psmallmatrix}\) satisfy \eqref{eq:energy_identity} for some \(\Impart\zeta>0\).
Then the M\"obius transform \(m\mapsto m':=\dfrac{am+b}{cm+d}\) maps \(\UHP\) into itself.
In particular, iterating \eqref{eq:cayley_step} over a sequence \(H_k\PS 0\)
produces truncation Weyl functions \(m_n\) that are Herglotz, hence a normal family on \(\UHP\).
\end{corollary}
\begin{proof}
Let \(v(m):=(m,1)^\top\), \(w:=Mv(m)\), and \(m':=w_1/w_2\) whenever \(w_2\neq0\).
Then
\[
v(m)^*Jv(m)=2i\,\Impart(m),\qquad
w^*Jw=2i\,\Impart(m')\,|w_2|^2.
\]
Evaluating \eqref{eq:energy_identity} on \(v(m)\) gives
\[
2i\,\Impart(m')\,|w_2|^2
=2i\,\Impart(m)+2i\,\Impart(\zeta)\,
\langle R(\zeta)^{-1}v(m),\,H\,R(\zeta)^{-1}v(m)\rangle.
\]
Since \(H\succeq0\) and \(\Impart(\zeta)>0\), the second term is nonnegative.
Hence \(\Impart(m')\ge0\) whenever \(\Impart(m)\ge0\), i.e. the M\"obius map sends
\(\UHP\) into itself. Iteration gives the Herglotz property for all truncation Weyl maps.
\end{proof}




% ============================================================
% Polarized identity and Pick/Gram kernels (finite truncations)
% ============================================================

\subsection{Polarized identity and finite Pick kernels}\label{sec:polarized_pick}

The one-step energy identity \eqref{eq:energy_identity} admits a polarized version that is tailored to
Pick/Gram representations.

\begin{lemma}[Polarized $J$-contractivity kernel]\label{lem:polarized_J_kernel}
Let $H\succeq 0$ be a (Hermitian) $2\times2$ Hamiltonian block and define $L(\zeta),R(\zeta),M(\zeta)$ as in \eqref{eq:cayley_step}.
For $\zeta,\omega\in\UHP$,
\begin{equation}\label{eq:polarized_energy_identity}
\frac{M(\zeta)^\ast J M(\omega)-J}{\omega-\overline\zeta}
=
R(\zeta)^{-\ast}\,H\,R(\omega)^{-1}\ \succeq\ 0.
\end{equation}
\end{lemma}

\begin{proof}
Expand $M(\zeta)^\ast J M(\omega)$ using $M=R^{-1}L$ and $L(\omega)=R(\omega)-\omega\,JH$.
A direct algebraic cancellation yields
\[
M(\zeta)^\ast J M(\omega)-J
=(\omega-\overline\zeta)\,R(\zeta)^{-\ast}HR(\omega)^{-1}.
\]
Since $H\succeq 0$, the right-hand side is a positive semidefinite kernel in $(\zeta,\omega)$ on $\UHP$.
\end{proof}

Now fix a sequence of blocks $H_k\succeq 0$ and the associated one-step matrices $M_k(z)$.
Let
\[
U_0(z):=I,\qquad
U_n(z):=M_{n-1}(z)\cdots M_1(z)M_0(z)\in\mathrm{GL}(2,\C).
\]
Define the matrix-valued kernel
\begin{equation}\label{eq:matrix_pick_kernel}
\mathcal K_n(z,w):=\frac{U_n(z)^\ast J U_n(w)-J}{w-\overline z}\qquad (z,w\in\UHP).
\end{equation}

\begin{lemma}[Discrete Lagrange identity / Gram sum]\label{lem:discrete_lagrange_gram}
For each $n\ge 1$ and $z,w\in\UHP$,
\begin{equation}\label{eq:gram_sum}
\mathcal K_n(z,w)
=
\sum_{k=0}^{n-1}
U_k(z)^\ast\,R_k(z)^{-\ast}H_kR_k(w)^{-1}\,U_k(w)
\ \succeq\ 0.
\end{equation}
\end{lemma}

\begin{proof}
Apply Lemma~\ref{lem:polarized_J_kernel} at step $k$ with $(\zeta,\omega)=(z,w)$ and conjugate by $U_k$.
Then telescope the identity
\[
\frac{U_{k+1}(z)^\ast J U_{k+1}(w)-U_k(z)^\ast J U_k(w)}{w-\overline z}
=
U_k(z)^\ast\,R_k(z)^{-\ast}H_kR_k(w)^{-1}\,U_k(w)
\]
from $k=0$ to $n-1$.
\end{proof}

\begin{corollary}[Finite Pick kernel as a compression]\label{cor:pick_kernel_compression}
Let $t\in\widehat{\mathbb R}$ be a (fixed) left boundary parameter and let $m^{(t)}_n$ be the truncation Weyl functions
obtained by iterating the M\"obius updates along $U_n$.
Then for each $n$ the scalar kernel
\begin{equation}\label{eq:pick_kernel_mn}
\Pi_{m^{(t)}_n}(z,w):=\frac{m^{(t)}_n(w)-\overline{m^{(t)}_n(z)}}{w-\overline z}\qquad(z,w\in\UHP)
\end{equation}
is positive semidefinite, and moreover it is a rank-one compression of \eqref{eq:matrix_pick_kernel}:
there exists a nonzero (boundary) vector $v_t\in\C^2$, independent of $(z,w)$, such that
\begin{equation}\label{eq:pick_kernel_compression_formula}
\Pi_{m^{(t)}_n}(z,w)=
\frac{\langle \mathcal K_n(z,w)\,v_t,\ v_t\rangle}{\langle e_2,U_n(z)\,v_t\rangle\ \overline{\langle e_2,U_n(w)\,v_t\rangle}}
\qquad(e_2=(0,1)^\top).
\end{equation}
In particular, $\Pi_{m^{(t)}_n}\succeq 0$ for all $n$.
\end{corollary}

\begin{proof}
Choose a nonzero boundary vector
\[
v_t:=
\begin{cases}
\smatrix{t\\1}, & t\in\mathbb R,\\[2pt]
\smatrix{1\\0}, & t=\infty.
\end{cases}
\]
Then $v_t^\ast J v_t=0$.
Set
\[
x_n^{(t)}(z):=U_n(z)\,v_t=\smatrix{x_1(z)\\x_2(z)}.
\]
By the projective definition of the truncation Weyl map,
\[
m_n^{(t)}(z)=\frac{x_1(z)}{x_2(z)}.
\]
Since $m_n^{(t)}$ is Herglotz (Corollary~\ref{cor:mobius_herglotz}), it is holomorphic on $\UHP$,
hence $x_2(z)\neq0$ for $z\in\UHP$.

Now compute
\[
m_n^{(t)}(w)-\overline{m_n^{(t)}(z)}
=\frac{x_1(w)\overline{x_2(z)}-\overline{x_1(z)}x_2(w)}
{x_2(w)\overline{x_2(z)}}.
\]
Because $J=\smatrix{0&-1\\1&0}$, the numerator equals
\[
x_n^{(t)}(z)^\ast J\,x_n^{(t)}(w)
=v_t^\ast U_n(z)^\ast J U_n(w)\,v_t.
\]
Therefore
\[
\Pi_{m_n^{(t)}}(z,w)
=\frac{v_t^\ast U_n(z)^\ast J U_n(w)\,v_t}
{(w-\overline z)\,x_2(w)\overline{x_2(z)}}.
\]
Insert
\[
U_n(z)^\ast J U_n(w)=J+(w-\overline z)\,\mathcal K_n(z,w)
\]
from \eqref{eq:matrix_pick_kernel}. Since $v_t^\ast J v_t=0$, we get
\[
\Pi_{m_n^{(t)}}(z,w)
=\frac{\langle \mathcal K_n(z,w)\,v_t,\ v_t\rangle}
{\langle e_2,U_n(z)\,v_t\rangle\ \overline{\langle e_2,U_n(w)\,v_t\rangle}},
\]
which is \eqref{eq:pick_kernel_compression_formula}.

For positivity, let $\{z_j\}_{j=1}^N\subset\UHP$ and $\{c_j\}_{j=1}^N\subset\mathbb C$.
Define
\[
\beta_j:=\langle e_2,U_n(z_j)\,v_t\rangle\neq0,\qquad
\xi_j:=\frac{c_j}{\beta_j}\,v_t\in\mathbb C^2.
\]
Then
\[
\sum_{i,j=1}^N c_i\overline{c_j}\,\Pi_{m_n^{(t)}}(z_i,z_j)
=\sum_{i,j=1}^N \xi_i^\ast\,\mathcal K_n(z_i,z_j)\,\xi_j\ge0
\]
because $\mathcal K_n\succeq0$ by Lemma~\ref{lem:discrete_lagrange_gram}.
Hence $\Pi_{m_n^{(t)}}\succeq0$.
\end{proof}

\section{R1: Weyl disk collapse (limit-point) for the reconstructed discrete canonical system}
\label{sec:R1}

In this section we close the remaining convergence/uniqueness bottleneck (R1) for the discrete canonical system
reconstructed from the Schur/Toeplitz pipeline: the Weyl disks (or, equivalently, the truncation-dependent Weyl
functions) collapse to a single point as the truncation length tends to infinity.




\subsection{$1$-jet diagnostics and boundary calibration}\label{sec:jet_and_R0}\label{sec:R2_target_identification}

The target-identification step in Section~\ref{sec:R2_target_identification} allows for a fixed
disk automorphism $R_0\in\mathrm{Aut}(\mathbb D)$ accounting for a possible mismatch of boundary
normalizations between the \emph{canonical} realization produced by the Hamiltonian blocks and
the \emph{target} pullback $S_{\mathrm{tgt},r}(\lambda)=W(z(r\lambda))$:
\begin{equation}\label{eq:TI_R0}
S_{\mathrm{tgt},r}(\lambda)=R_0\!\left(S_{\mathrm{can},r}(\lambda)\right),\qquad r\in(0,1).
\end{equation}
The paper does \emph{not} need (and does not assume) $R_0=\mathrm{id}$.
Instead, we use two simple facts:
(i)~a disk automorphism is uniquely determined by its $1$-jet at any interior point, and
(ii)~$R_0$ can be absorbed into a boundary-condition choice of the canonical system
without changing any Schur/Pick positivity conclusions.

\medskip
\paragraph{Jet pinning of $R_0$.}
Let $u=S_{\mathrm{can},r}(0)$, $v=S_{\mathrm{tgt},r}(0)$ and let $p=S'_{\mathrm{can},r}(0)$, $q=S'_{\mathrm{tgt},r}(0)$.
If \eqref{eq:TI_R0} holds, then $v=R_0(u)$ and $q=R_0'(u)\,p$.
Lemma~\ref{lem:R0_jet_pinning} gives an explicit closed-form reconstruction of the unique
$R_0\in\mathrm{Aut}(\mathbb D)$ from $(u,v,p,q)$.
This is the mathematically correct ``$1$-jet diagnostic'': it identifies the \emph{unique} residual
value-space gauge $R_0$ that makes \eqref{eq:TI_R0} exact; existence is established in Theorem~\ref{thm:ID_unconditional}.

\medskip
\paragraph{Absorbing $R_0$ into the realization (no loss for Schur/Pick).}
Postcomposition by $R_0\in\mathrm{Aut}(\mathbb D)$ preserves the Schur property: if $S$ is Schur, then
$R_0\circ S$ is Schur. Consequently, all equivalent positivity formulations in
Theorem~\ref{thm:Lstar_equivalences} are invariant under postcomposition.
Therefore we may replace $S_{\mathrm{can},r}$ by the calibrated Schur function
\begin{equation}\label{eq:Scal_def}
S_{\mathrm{can},r}^{\mathrm{cal}}:=R_0\circ S_{\mathrm{can},r},
\end{equation}
and rewrite \eqref{eq:TI_R0} simply as $S_{\mathrm{tgt},r}=S_{\mathrm{can},r}^{\mathrm{cal}}$.

At the level of canonical systems this calibration is not artificial: changing the left boundary
condition changes the Weyl function by a real M\"obius map, hence changes the corresponding
Schur function by a disk automorphism postcomposition.
Thus, \eqref{eq:Scal_def} can be viewed as a \emph{boundary-gauge fixing} (Arov normalization) of
the realization, rather than an extra hypothesis.

\medskip
\paragraph{What remains.}
In the present paper, the RH closure no longer depends on this calibration paragraph: the global Schur property of $W$ on $\mathbb C_+$ is established directly through Section~\ref{sec:final_gap} (passivity anchor $\Rightarrow$ circle--Hardy certificate $\Rightarrow$ strip pole-exclusion), and RH then follows from Theorem~\ref{thm:reverse_RH}.  The identification material below is kept as an internal consistency refinement of the canonical/target matching step.
% --- New: unconditional target identification (closes the R0-existence gap) ---
\subsection{Unconditional target identification from disk collapse and circle-Hardy rigidity}\label{sec:ID_unconditional}

This subsection supplies the only missing implication needed to turn the calibration relation
\eqref{eq:TI_R0}--\eqref{eq:Scal_def} from a \emph{diagnostic} into an \emph{actual identification}:
the target pullback $S_{\mathrm{tgt},r}(\lambda)=W(z(r\lambda))$ is forced to be holomorphic on $\DD$
(and hence pole-free on the corresponding spectral region) and must coincide with the calibrated
canonical pullback $S_{\mathrm{can},r}^{\mathrm{cal}}$.

\medskip
\paragraph{Step A: detector vanishing for canonical truncations.}
Fix $r\in(0,1)$.  For each truncation length $n$ and each left boundary parameter $t\in\R_b$,
let $m_n(\cdot\,;t)$ be the $n$--step truncation Weyl map produced by the discrete canonical blocks
$H_k\succeq 0$ and let $S_{n,r}$ be its disk pullback (the $\lambda$--variable Cayley reparametrization
together with the value-space Cayley transform used in the definition of $S_{\mathrm{can},r}$).

\begin{lemma}[Canonical truncations have no negative modes on circles]\label{lem:canonical_trunc_detector0}
For every $z_\ast\in\C_+$, every $R>0$ with $\overline{z_\ast+R\DD}\subset\C_+$, and every truncation $m_n(\cdot\,;t)$,
the boundary trace
\[
g_{n,t}(\zeta):=m_n(z_\ast+R\zeta\,;t),\qquad |\zeta|=1,
\]
satisfies $P_-g_{n,t}=0$, equivalently $\mathcal E_-(g_{n,t})=0$.
Consequently, for every $\rho\in(0,1)$ the boundary trace of $S_{n,r}$ on $|\lambda|=\rho$ has vanishing
negative Fourier modes, hence the corresponding $b$--detector defect equals $0$.
\end{lemma}

\begin{proof}
By Corollary~\ref{cor:mobius_herglotz}, each truncation Weyl map $m_n(\cdot\,;t)$ sends $\C_+$ into $\C_+$.
In particular, $m_n$ cannot have a pole inside $\C_+$: if $m_n$ had a pole at $z_0\in\C_+$, then
$m_n(z)\to\infty$ as $z\to z_0$, contradicting $m_n(z)\in\C_+$.
Hence $m_n(\cdot\,;t)$ is holomorphic on $\C_+$.

Fix a disk $\overline{z_\ast+R\DD}\subset\C_+$.
Then the composite $\zeta\mapsto m_n(z_\ast+R\zeta\,;t)$ is holomorphic on a neighborhood of $\overline\DD$,
so its boundary trace on $|\zeta|=1$ has no negative Fourier modes.
Concretely, writing the Taylor series $m_n(z_\ast+R\zeta\,;t)=\sum_{k\ge 0}c_k\zeta^k$ valid for $|\zeta|\le 1$,
orthogonality on the circle gives $\widehat g_{n,t}(-\ell)=0$ for all $\ell\ge 1$, hence
$\mathcal E_-(g_{n,t})=\sum_{\ell\ge 1}|\widehat g_{n,t}(-\ell)|^2=0$.

Finally, $S_{n,r}$ is obtained from $m_n$ by (i) the holomorphic spectral reparametrization
$\lambda\mapsto z(r\lambda)$ and (ii) a fixed rational value transform.
Both operations preserve holomorphy on their domains, hence preserve the vanishing of negative modes for circle traces.
\end{proof}

\medskip
\paragraph{Step B: rigidity forces the target pullback to be holomorphic on $\DD$.}
We now show that the target pullback cannot develop an interior pole once it shares the
same germ at $\lambda=0$ with a function whose circle defects vanish.


\begin{lemma}[Jet consistency for the finite-section Schur recursion]\label{lem:jet_consistency}
Let $S(\lambda)=\sum_{k\ge0}s_k\lambda^k$ be a holomorphic germ at $\lambda=0$ with $|s_0|\neq1$.
Define recursively the \emph{Schur iterates} $S^{(0)}:=S$ and, for $j\ge0$,
\begin{equation}\label{eq:schur_iterate_def}
\gamma_j:=S^{(j)}(0),\qquad
S^{(j+1)}(\lambda):=\frac{1}{\lambda}\,\frac{S^{(j)}(\lambda)-\gamma_j}{1-\overline{\gamma_j}\,S^{(j)}(\lambda)} ,
\end{equation}
which is again a holomorphic germ provided $1-|\gamma_j|^2\neq0$.
For any $n\ge0$, define the $n$--step \emph{Schur truncation} $S^{[n]}$ by reversing the recursion:
set $S^{[n]}_{n}(\lambda)\equiv \gamma_n$ and, for $j=n-1,\dots,0$,
\begin{equation}\label{eq:schur_reverse_def}
S^{[n]}_{j}(\lambda):=\frac{\gamma_j+\lambda\,S^{[n]}_{j+1}(\lambda)}{1+\overline{\gamma_j}\,\lambda\,S^{[n]}_{j+1}(\lambda)} ,
\end{equation}
and finally $S^{[n]}:=S^{[n]}_{0}$.
Then
\begin{equation}\label{eq:jet_match}
S^{[n]}(\lambda)-S(\lambda)=O(\lambda^{n+1})\qquad(\lambda\to0).
\end{equation}
In the borderline case $|s_0|=1$, the recursion terminates at $j=0$ and \eqref{eq:jet_match} holds with the constant truncation
$S^{[0]}(\lambda)\equiv s_0$.
\end{lemma}

\begin{proof}
We prove by induction on $n$ the stronger statement that
\begin{equation}\label{eq:jet_match_strong}
S^{[n]}_{j}(\lambda)-S^{(j)}(\lambda)=O(\lambda^{n-j+1})\qquad(\lambda\to0),\qquad 0\le j\le n.
\end{equation}
For $j=n$, both sides equal $\gamma_n$ at $\lambda=0$, so \eqref{eq:jet_match_strong} holds trivially with order $O(\lambda)$.

Assume \eqref{eq:jet_match_strong} holds for some $j+1\le n$.
Write $A(\lambda):=S^{[n]}_{j+1}(\lambda)$ and $B(\lambda):=S^{(j+1)}(\lambda)$, so
$A(\lambda)-B(\lambda)=O(\lambda^{n-j})$.
Define the M\"obius map
\[
\Phi_{\gamma_j}(\lambda,w):=\frac{\gamma_j+\lambda w}{1+\overline{\gamma_j}\lambda w}.
\]
Then by \eqref{eq:schur_reverse_def} we have $S^{[n]}_{j}(\lambda)=\Phi_{\gamma_j}(\lambda,A(\lambda))$.
On the other hand, the forward recursion \eqref{eq:schur_iterate_def} is algebraically equivalent to
$S^{(j)}(\lambda)=\Phi_{\gamma_j}(\lambda,S^{(j+1)}(\lambda))=\Phi_{\gamma_j}(\lambda,B(\lambda))$
(as an identity of germs, obtained by solving \eqref{eq:schur_iterate_def} for $S^{(j)}$).
Therefore
\[
S^{[n]}_{j}(\lambda)-S^{(j)}(\lambda)
=\Phi_{\gamma_j}(\lambda,A(\lambda))-\Phi_{\gamma_j}(\lambda,B(\lambda)).
\]
A direct denominator computation gives
\[
\Phi_{\gamma_j}(\lambda,A)-\Phi_{\gamma_j}(\lambda,B)
=\frac{\lambda\,(A-B)\,(1-|\gamma_j|^2)}{(1+\overline{\gamma_j}\lambda A)(1+\overline{\gamma_j}\lambda B)}.
\]
Since $A,B$ are holomorphic at $0$, the denominator is $1+O(\lambda)$, hence
$S^{[n]}_{j}(\lambda)-S^{(j)}(\lambda)=\lambda\,(A(\lambda)-B(\lambda))\cdot(1+O(\lambda))=O(\lambda^{n-j+1})$,
which is \eqref{eq:jet_match_strong}.
Taking $j=0$ yields \eqref{eq:jet_match}.
The borderline case $|s_0|=1$ is immediate: then $S(\lambda)\equiv s_0$ as a disk-valued germ and the constant truncation matches all jets.
\end{proof}

\begin{theorem}[Identification by germ matching and circle-Hardy rigidity]\label{thm:ID_unconditional}
Fix $r\in(0,1)$ and let
\[
S_{\mathrm{tgt},r}(\lambda):=W(z(r\lambda))
\]
be the target pullback, initially understood as a holomorphic germ at $\lambda=0$ (equivalently, on some disk $|\lambda|<\rho_0$).
Let $S_{\mathrm{can},r}$ be the canonical pullback obtained from the reconstructed discrete canonical system,
and let $S_{\mathrm{can},r}^{\mathrm{cal}}$ be its calibration \eqref{eq:Scal_def}.
Within this manuscript, the three inputs \textup{(I1)}--\textup{(I3)} used in the proof
are verified in Corollary~\ref{cor:ID_assumptions_verified}.
Then $S_{\mathrm{tgt},r}$ extends holomorphically to all of $\DD$ and
\[
S_{\mathrm{tgt},r}(\lambda)\equiv S_{\mathrm{can},r}^{\mathrm{cal}}(\lambda)\qquad(\forall \lambda\in\DD).
\]
In particular, \eqref{eq:TI_R0} holds with the (necessarily unique) $R_0$ reconstructed from the $1$--jet by Lemma~\ref{lem:R0_jet_pinning}.
\end{theorem}

\begin{proof}
By Corollary~\ref{cor:ID_assumptions_verified}, the three ingredients
\textup{(I1)}--\textup{(I3)} used below are already verified in this manuscript.
Fix $r$ and write $S_{\mathrm{tgt}}:=S_{\mathrm{tgt},r}$, $S_{\mathrm{cal}}:=S_{\mathrm{can},r}^{\mathrm{cal}}$.
By (I1) and (I2), taking $n\to\infty$ shows that $S_{\mathrm{cal}}$ and $S_{\mathrm{tgt}}$ have the same Taylor series at $\lambda=0$:
for each $k\ge 0$, the $k$--th Taylor coefficient of $S_{n,r}^{\mathrm{cal}}$ stabilizes to that of $S_{\mathrm{tgt}}$ once $n\ge k$,
and local uniform convergence gives the same coefficient for $S_{\mathrm{cal}}$.

Let $\rho_\ast\in(0,1]$ be the maximal radius such that $S_{\mathrm{tgt}}$ extends holomorphically to $|\lambda|<\rho_\ast$.
On that disk, $S_{\mathrm{cal}}$ is holomorphic and shares the same Taylor series at $0$, hence
$S_{\mathrm{cal}}(\lambda)=S_{\mathrm{tgt}}(\lambda)$ for all $|\lambda|<\rho_\ast$ by the identity principle.

It remains to show $\rho_\ast=1$.
Assume for contradiction that $\rho_\ast<1$.  Choose any $\rho$ with $0<\rho<\rho_\ast$.
Then $S_{\mathrm{tgt}}$ is holomorphic on a neighborhood of $\overline{\rho\DD}$ and agrees with $S_{\mathrm{cal}}$ there.
By (I3), the boundary trace $S_{\mathrm{cal}}(\rho e^{it})$ has no negative Fourier modes, hence so does
$S_{\mathrm{tgt}}(\rho e^{it})$.

Now suppose $S_{\mathrm{tgt}}$ had a pole inside $\rho\DD$ for some $\rho<\rho_\ast$.
Then, writing the circle trace in the $\zeta$--variable on $|\zeta|=1$,
Lemma~\ref{lem:hardy_pole_locator} implies that the negative Fourier modes are determined by (and in particular are \emph{nonzero} for)
the principal part of that pole.  This contradicts the vanishing of negative modes on $|\lambda|=\rho$.
Therefore $S_{\mathrm{tgt}}$ has no poles in $\rho\DD$ for every $\rho<\rho_\ast$.

But the only obstruction to holomorphic continuation of a meromorphic function is the presence of a pole.
Since no pole can occur at any radius $\rho<\rho_\ast$, the maximality of $\rho_\ast$ is contradicted unless $\rho_\ast=1$.
Hence $S_{\mathrm{tgt}}$ extends holomorphically to $\DD$ and equals $S_{\mathrm{cal}}$ on $\DD$.

Finally, $S_{\mathrm{cal}}=R_0\circ S_{\mathrm{can},r}$ by definition \eqref{eq:Scal_def},
so \eqref{eq:TI_R0} holds with this $R_0$, and Lemma~\ref{lem:R0_jet_pinning} gives the explicit $1$--jet formula.
\end{proof}

\begin{corollary}[Internal verification of assumptions \textup{(I1)}--\textup{(I3)}]
\label{cor:ID_assumptions_verified}
Within the present manuscript, the three identification inputs
\textup{(I1)}--\textup{(I3)} are available as follows:
\begin{enumerate}[label=\textup{(\roman*)}]
\item \textup{(I1)} follows from Lemma~\ref{lem:trace-lower-bound},
Theorem~\ref{thm:R1-limit-point}, and
Lemma~\ref{lem:limit_point_unique_limit};
\item \textup{(I2)} is exactly Lemma~\ref{lem:jet_consistency};
\item \textup{(I3)} is exactly Lemma~\ref{lem:canonical_trunc_detector0}.
\end{enumerate}
\end{corollary}

\begin{proof}
Each item is an explicit cross-reference to a proved statement above, so all assumptions
used in the identification step are in place.
\end{proof}

\medskip
\noindent\textbf{Interpretation.}
Theorem~\ref{thm:ID_unconditional} turns the calibration relation \eqref{eq:TI_R0} into a genuine identification:
once the canonical system is limit-point and its circle defects vanish (a purely algebraic consequence of the step passivity),
the target pullback cannot hide an interior pole without producing a nonzero negative-mode defect on some circle.

\subsection{Discrete canonical step and truncation Weyl maps}

Fix a spectral parameter $z\in\mathbb C$ with $\Impart z>0$.
For each $k\ge 0$, let $H_k\succeq 0$ be a real-symmetric $2\times2$ Hamiltonian block and set
\begin{equation}\label{eq:R1-step}
J=\begin{pmatrix}0&-1\\1&0\end{pmatrix},\qquad
L_k(z)=I-\frac z2\,JH_k,\qquad
R_k(z)=I+\frac z2\,JH_k,\qquad
M_k(z)=R_k(z)^{-1}L_k(z).
\end{equation}
(As shown in \S\ref{sec:B21}, $R_k(z)$ is invertible for the rank-one blocks used in this reconstruction.)

Define the $n$-step transfer matrix
\begin{equation}\label{eq:R1-Un}
U_n(z):=M_{n-1}(z)\cdots M_1(z)M_0(z)
=
\begin{pmatrix}
A_n(z) & B_n(z)\\
C_n(z) & D_n(z)
\end{pmatrix}.
\end{equation}
For a real boundary parameter $t\in\widehat{\mathbb R}:=\mathbb R\cup\{\infty\}$, define the truncation Weyl map
\begin{equation}\label{eq:R1-mn}
m_n(z;t):=\frac{A_n(z)\,t+B_n(z)}{C_n(z)\,t+D_n(z)}.
\end{equation}
The corresponding \emph{Weyl set} (a disk in $\mathbb C$ for $\Impart z>0$) is
\begin{equation}\label{eq:R1-disk}
\mathcal D_n(z):=\{\, m_n(z;t): t\in\widehat{\mathbb R}\,\}.
\end{equation}

\subsection{Weyl circle as a single quadratic form}

Let $v(m):=\binom{m}{1}$ and define
\begin{equation}\label{eq:R1-Qn}
Q_n(z):=U_n(z)^{-*}\,J\,U_n(z)^{-1},
\qquad
\widetilde Q_n(z):=\frac{1}{2i}\,Q_n(z).
\end{equation}
Since $J^*=-J$, the matrix $\widetilde Q_n(z)$ is Hermitian for each $\Impart z>0$.

\begin{lemma}[Weyl circle equation]\label{lem:R1-circle-eq}
For fixed $z\in\mathbb C$ with $\Impart z>0$, the boundary of $\mathcal D_n(z)$ is the conic
\begin{equation}\label{eq:R1-circle-eq}
v(m)^*\,\widetilde Q_n(z)\,v(m)=0.
\end{equation}
Equivalently, writing $\widetilde Q_n=\begin{psmallmatrix}\tilde q_{11}^{(n)}&\tilde q_{12}^{(n)}\\\overline{\tilde q_{12}^{(n)}}&\tilde q_{22}^{(n)}\end{psmallmatrix}$,
the boundary is the circle/line
\begin{equation}\label{eq:R1-circle-expanded}
\tilde q_{11}^{(n)}\,|m|^2+2\Repart\!\big(\tilde q_{12}^{(n)}\,m\big)+\tilde q_{22}^{(n)}=0.
\end{equation}
\end{lemma}
\begin{proof}
Set $U:=U_n(z)$ and let $t:=U^{-1}\!\cdot m$.
By the projective identity proved in Lemma~\ref{lem:R1-membership},
\[
v(m)^*\,\widetilde Q_n(z)\,v(m)=|\kappa|^2\,\Impart(t)
\]
for some $\kappa\neq 0$ depending on $m$.
By definition,
\[
m\in \partial\mathcal D_n(z)
\iff
t\in\widehat{\mathbb R}
\iff
\Impart(t)=0,
\]
hence
\[
m\in\partial\mathcal D_n(z)\iff v(m)^*\,\widetilde Q_n(z)\,v(m)=0.
\]
This is \eqref{eq:R1-circle-eq}.
Expanding the quadratic form for
$\widetilde Q_n=\smatrix{\tilde q_{11}^{(n)}&\tilde q_{12}^{(n)}\\\overline{\tilde q_{12}^{(n)}}&\tilde q_{22}^{(n)}}$
gives \eqref{eq:R1-circle-expanded}.
\end{proof}

\subsection{Weyl disk membership inequality}
\label{sec:membership}

The circle equation \eqref{eq:R1-circle-eq} determines only the boundary. For the bridge stage one needs a
\emph{membership} criterion (interior vs exterior) that is purely algebraic and does not rely on asymptotics.

\begin{lemma}[Quadratic-form sign and disk membership]\label{lem:R1-membership}
Fix $z$ with $\Impart z>0$ and write $U:=U_n(z)$.
For any $m\in\widehat{\mathbb C}$ define the (possibly infinite) preimage
\[
t:=U^{-1}\cdot m,
\qquad\text{i.e.}\qquad
t=\frac{\alpha m+\beta}{\gamma m+\delta}\quad\text{when }U^{-1}=\begin{pmatrix}\alpha&\beta\\\gamma&\delta\end{pmatrix}.
\]
Then for any representative vectors $v(m)=(m,1)^\top$ and $v(t)=(t,1)^\top$ there exists a nonzero scalar $\kappa=\kappa(m)$
such that
\[
U^{-1}v(m)=\kappa\,v(t),
\]
and consequently one has the identity
\begin{equation}\label{eq:R1-membership-identity}
v(m)^*\,\widetilde Q_n(z)\,v(m)=|\kappa|^2\,\Impart(t).
\end{equation}
In particular, the sign of $v(m)^*\,\widetilde Q_n(z)\,v(m)$ agrees with the sign of $\Impart(U^{-1}\cdot m)$.
Moreover, $\partial\mathcal D_n(z)$ is the locus $\Impart(U^{-1}\cdot m)=0$, and the interior of $\mathcal D_n(z)$ is
one of the two half-spaces $\Impart(U^{-1}\cdot m)\gtrless 0$.
\end{lemma}
\begin{proof}
By the projective definition of the M\"obius action, $m=U\cdot t$ is equivalent to $v(m)$ being proportional to $Uv(t)$.
Equivalently $U^{-1}v(m)$ is proportional to $v(t)$, giving $U^{-1}v(m)=\kappa v(t)$ for some $\kappa\neq 0$.
Using $Q_n(z)=U^{-*}JU^{-1}$ and $\widetilde Q_n=(2i)^{-1}Q_n$ we obtain
\[
v(m)^*\,\widetilde Q_n\,v(m)
=\frac{1}{2i}\,v(m)^*\,U^{-*}JU^{-1}\,v(m)
=\frac{|\kappa|^2}{2i}\,v(t)^*Jv(t).
\]
A direct computation gives $v(t)^*Jv(t)=\overline t- t=-2i\,\Impart(t)$, hence \eqref{eq:R1-membership-identity}.
\end{proof}

\begin{lemma}[Radius formula]\label{lem:R1-radius-formula}
Assume $\det \widetilde Q_n(z)<0$ (the generic ``circle'' case). Then the Weyl disk $\mathcal D_n(z)$ has radius
\begin{equation}\label{eq:R1-radius}
R_n(z)=\frac{\sqrt{-\det \widetilde Q_n(z)}}{\big|\tilde q_{11}^{(n)}(z)\big|}.
\end{equation}
\end{lemma}
\begin{proof}
In the generic circle case one has $\tilde q_{11}^{(n)}\neq 0$.
From \eqref{eq:R1-circle-expanded},
\[
\tilde q_{11}^{(n)}|m|^2+2\Repart(\tilde q_{12}^{(n)}m)+\tilde q_{22}^{(n)}=0.
\]
Divide by $\tilde q_{11}^{(n)}$ and complete the square:
\[
\left|m+\frac{\overline{\tilde q_{12}^{(n)}}}{\tilde q_{11}^{(n)}}\right|^2
=
\frac{|\tilde q_{12}^{(n)}|^2-\tilde q_{11}^{(n)}\tilde q_{22}^{(n)}}{|\tilde q_{11}^{(n)}|^2}
=
\frac{-\det\widetilde Q_n(z)}{|\tilde q_{11}^{(n)}|^2}.
\]
Since $\det\widetilde Q_n(z)<0$, the right-hand side is positive and equals $R_n(z)^2$.
Taking square roots gives \eqref{eq:R1-radius}.
\end{proof}

\begin{lemma}[Center form and minimax (minimum-error) bound]\label{lem:R1-minimax}
Assume $\det \widetilde Q_n(z)<0$ and $\tilde q_{11}^{(n)}(z)\neq 0$.
Then the Weyl set admits the Euclidean disk form
\begin{equation}\label{eq:R1-disk-center}
\mathcal D_n(z)=\{\, m\in\mathbb C: |m-c_n(z)|\le R_n(z)\,\},
\qquad
c_n(z):=-\frac{\overline{\tilde q_{12}^{(n)}(z)}}{\tilde q_{11}^{(n)}(z)}.
\end{equation}
Moreover, among all single-valued estimates $a\in\mathbb C$ based only on the $n$-step truncation,
the smallest worst-case error equals the radius:
\begin{equation}\label{eq:R1-minimax-eq}
\inf_{a\in\mathbb C}\ \sup_{m\in\mathcal D_n(z)} |m-a| \,=\, R_n(z),
\end{equation}
and the infimum is attained uniquely at $a=c_n(z)$.
\end{lemma}
\begin{proof}
By the completion-of-square identity used above,
\[
\mathcal D_n(z)=\{\,m\in\C:\ |m-c_n(z)|\le R_n(z)\,\},
\quad
c_n(z)=-\frac{\overline{\tilde q_{12}^{(n)}(z)}}{\tilde q_{11}^{(n)}(z)},
\]
and $R_n(z)$ is given by Lemma~\ref{lem:R1-radius-formula}.

For any $a\in\C$, write $m=c_n+Re^{i\theta}$ on the boundary circle.
Then
\[
\sup_{m\in\mathcal D_n(z)}|m-a|
=\sup_\theta |(c_n-a)+Re^{i\theta}|
=|a-c_n|+R_n(z).
\]
Hence
\[
\inf_{a\in\C}\sup_{m\in\mathcal D_n(z)}|m-a|
=\inf_{a\in\C}\bigl(|a-c_n|+R_n(z)\bigr)=R_n(z),
\]
with equality iff $|a-c_n|=0$, i.e.\ uniquely at $a=c_n(z)$.
\end{proof}

\begin{corollary}[Minimum-error convergence]\label{cor:R1-minerr-conv}
Let $m(z)$ denote the (unique) Weyl limit in the limit-point case (Theorem~\ref{thm:R1-limit-point}).
Then for every $n$,
\begin{equation}\label{eq:R1-center-error}
|c_n(z)-m(z)|\le R_n(z).
\end{equation}
In particular, $R_n(z)\to 0$ implies $c_n(z)\to m(z)$, i.e. the minimax (minimum worst-case) truncation error tends to $0$.
\end{corollary}

\begin{proof}
Completing the square in \eqref{eq:R1-circle-expanded} yields \eqref{eq:R1-disk-center} and the same radius as in Lemma~\ref{lem:R1-radius-formula}.
The minimax statement \eqref{eq:R1-minimax-eq} is the Chebyshev-center property of Euclidean disks.
Finally, since $m(z)\in\cap_n\mathcal D_n(z)$ and $\mathcal D_n(z)$ is a disk of radius $R_n(z)$ about $c_n(z)$, we obtain \eqref{eq:R1-center-error}.
\end{proof}

\begin{lemma}[Limit-point collapse $\Rightarrow$ unique Herglotz limit (locally uniform)]\label{lem:limit_point_unique_limit}
Assume $(m_n)$ is any sequence of truncation Weyl functions produced from blocks $H_k\succeq 0$ (hence each $m_n$ is Herglotz),
and assume the Weyl disk radii satisfy $R_n(z)\to 0$ for every $z\in\UHP$.
Then there exists a \emph{unique} Herglotz function $m$ on $\UHP$ such that
\[
m_n(z)\to m(z)\qquad(n\to\infty)
\]
locally uniformly on $\UHP$.
Moreover, the limit is independent of the left boundary parameter $t\in\widehat{\mathbb R}$.
\end{lemma}

\begin{proof}
Since each $m_n$ maps $\UHP$ into itself, $(m_n)$ is a normal family on $\UHP$ (Montel).
Fix $z\in\UHP$. By Corollary~\ref{cor:R1-minerr-conv}, any Weyl parameter belongs to the shrinking disks $\mathcal D_n(z)$
and satisfies $|c_n(z)-m(z)|\le R_n(z)$.
Thus $R_n(z)\to 0$ forces every choice of $m_n(z)\in\mathcal D_n(z)$ to converge to the same value; in particular, pointwise limits are unique.
Normal-family compactness upgrades pointwise convergence on a set with an accumulation point to locally-uniform convergence on $\UHP$,
and the limit is Herglotz by stability under locally-uniform limits.
Boundary-parameter independence follows because different left boundary choices correspond to different points in the same Weyl disk for each $n$,
and the disk radius collapses to $0$.
\end{proof}

\subsection{\texorpdfstring{Energy identity $\Rightarrow$ explicit radius--energy relation}{Energy identity => explicit radius--energy relation}}

From the one-step energy identity in \S\ref{sec:B21}, for $\Impart z>0$ we have
\begin{equation}\label{eq:R1-energy-step}
\frac{M_k(z)^*JM_k(z)-J}{2i}
=\Impart(z)\,R_k(z)^{-*}H_kR_k(z)^{-1}\succeq 0.
\end{equation}
Iterating and using $U_{k+1}=M_kU_k$ with $U_0=I$, we obtain the $n$-step energy identity
\begin{equation}\label{eq:R1-energy-n}
\frac{U_n(z)^*JU_n(z)-J}{2i}
=
\Impart(z)\,E_n(z)\succeq 0,
\qquad
E_n(z):=\sum_{k=0}^{n-1}U_k(z)^*\,R_k(z)^{-*}H_kR_k(z)^{-1}\,U_k(z).
\end{equation}
Multiplying \eqref{eq:R1-energy-n} on the left by $U_n(z)^{-*}$ and on the right by $U_n(z)^{-1}$ gives
\begin{equation}\label{eq:R1-QE-link}
\widetilde Q_n(z)
=
\frac{J}{2i}-\Impart(z)\,U_n(z)^{-*}\,E_n(z)\,U_n(z)^{-1}.
\end{equation}
Let $e_1=(1,0)^T$ and set $w_n(z):=U_n(z)^{-1}e_1$. Since $(J/(2i))_{11}=0$, taking the $(1,1)$ entry of
\eqref{eq:R1-QE-link} yields the \emph{exact identity}
\begin{equation}\label{eq:R1-q11-energy}
\tilde q_{11}^{(n)}(z)
=
-\Impart(z)\,\langle w_n(z),E_n(z)\,w_n(z)\rangle.
\end{equation}

\subsection{\texorpdfstring{Determinant normalization and the ``$1/q_{11}$'' radius law}{Determinant normalization and the 1/q11 radius law}}

\begin{lemma}[Unimodularity]\label{lem:R1-det1}
For each $k$ and $z$, $\det M_k(z)=1$, hence $\det U_n(z)=1$ for all $n$.
\end{lemma}

\begin{proof}
Write $L_k(z)=I-\frac{z}{2}JH_k$ and $R_k(z)=I+\frac{z}{2}JH_k$.
Since $H_k$ is real symmetric, we have $\mathrm{tr}(JH_k)=0$, hence the Cayley--Hamilton identity gives
\[
(JH_k)^2 = -\det(JH_k)\,I = -\det(H_k)\,I
\]
(using $\det J=1$). For any scalar $t$ and any $2\times 2$ matrix $A$ with $\mathrm{tr}(A)=0$,
\[
\det(I+tA)=1+t^2\det(A),
\]
so with $A=JH_k$ and $t=\pm z/2$ we obtain
\[
\det L_k(z)=\det R_k(z)=1+\frac{z^2}{4}\det(H_k).
\]
Therefore $\det M_k(z)=\det(R_k(z)^{-1}L_k(z))=\det L_k(z)/\det R_k(z)=1$.
\end{proof}

\begin{lemma}[Fixed determinant of the Weyl circle matrix]\label{lem:R1-detQ}
For every $n$ and $\Impart z>0$,
\begin{equation}\label{eq:R1-detQ}
-\det \widetilde Q_n(z)=\frac14.
\end{equation}
\end{lemma}

\begin{proof}
By definition, $Q_n=U_n^{-*}JU_n^{-1}$, hence $\det Q_n=\det J/|\det U_n|^2$.
Since $\det J=1$ and $\det U_n=1$ by Lemma~\ref{lem:R1-det1}, we have $\det Q_n=1$.
Therefore $\det \widetilde Q_n=(1/(2i))^2\det Q_n=-1/4$.
\end{proof}

From Lemma~\ref{lem:R1-radius-formula} and Lemma~\ref{lem:R1-detQ} we obtain the explicit
``$1/q_{11}$'' radius law
\begin{equation}\label{eq:R1-radius-q11}
R_n(z)=\frac{1}{2\,\big|\tilde q_{11}^{(n)}(z)\big|}.
\end{equation}

Combining Lemma~\ref{lem:R1-radius-formula}, \eqref{eq:R1-q11-energy}, and Lemma~\ref{lem:R1-detQ} gives the
\emph{exact} radius--energy identity
\begin{equation}\label{eq:R1-radius-energy}
\boxed{
R_n(z)=\frac{1}{2\,\Impart(z)\,\langle w_n(z),E_n(z)\,w_n(z)\rangle}.
}
\end{equation}
Indeed, by \eqref{eq:R1-radius-q11} and \eqref{eq:R1-q11-energy},
\[
R_n(z)
=\frac{1}{2\,|\tilde q_{11}^{(n)}(z)|}
=\frac{1}{2\,\big|-\Impart(z)\,\langle w_n,E_n w_n\rangle\big|}.
\]
Since $\Impart(z)>0$ and $E_n(z)\succeq0$, the scalar
$\langle w_n,E_n w_n\rangle$ is real and nonnegative, so
\[
\big|-\Impart(z)\,\langle w_n,E_n w_n\rangle\big|
=\Impart(z)\,\langle w_n,E_n w_n\rangle.
\]
Substituting yields \eqref{eq:R1-radius-energy}.
Thus the limit-point statement $R_n(z)\to 0$ is equivalent to the divergence of the scalar energy component
$\langle w_n,E_n w_n\rangle\to\infty$.

\subsection{An internal coercive route to radius collapse}

For readers who prefer an explicit internal criterion, we record the following
two lemmas. They isolate a sufficient coercivity condition under which
\eqref{eq:R1-radius-energy} yields $R_n(z)\to 0$ directly from the block data.

\begin{lemma}[Energy-coercive criterion for Weyl-disk collapse]\label{lem:R1_energy_coercive}
Fix $z\in\UHP$. Assume
\[
H_k=\tau_k\,u_k u_k^\ast,\qquad \tau_k=\operatorname{tr}(H_k)>0,\ \|u_k\|=1,
\]
and define
\[
w_n(z):=U_n(z)^{-1}e_1,\qquad
E_n(z):=\sum_{k=0}^{n-1}U_k(z)^\ast R_k(z)^{-\ast}H_kR_k(z)^{-1}U_k(z).
\]
If there exists $c(z)>0$ such that for all $n\ge1$ and $0\le k<n$,
\[
\left|\left\langle u_k,\;R_k(z)^{-1}U_k(z)w_n(z)\right\rangle\right|^2\ge c(z),
\]
then
\[
R_n(z)\le \frac{1}{2\,\Impart z\,c(z)\sum_{k=0}^{n-1}\tau_k}.
\]
In particular, if $\sum_{k\ge0}\tau_k=\infty$, then $R_n(z)\to0$.
\end{lemma}

\begin{proof}
For each $k<n$, set
\[
y_{k,n}(z):=R_k(z)^{-1}U_k(z)w_n(z).
\]
Then by definition of $E_n$,
\[
\langle w_n,E_n w_n\rangle
=\sum_{k=0}^{n-1}\left\langle y_{k,n},\,H_k\,y_{k,n}\right\rangle.
\]
Since $H_k=\tau_k u_k u_k^\ast$,
\[
\langle w_n,E_n w_n\rangle
=\sum_{k=0}^{n-1}\tau_k
\left|\left\langle u_k,\;y_{k,n}\right\rangle\right|^2
\ge c(z)\sum_{k=0}^{n-1}\tau_k.
\]
Using \eqref{eq:R1-radius-energy},
\[
R_n(z)=\frac{1}{2\,\Impart z\,\langle w_n,E_n w_n\rangle}
\le
\frac{1}{2\,\Impart z\,c(z)\sum_{k=0}^{n-1}\tau_k}.
\]
If $\sum_{k\ge0}\tau_k=\infty$, the right-hand side tends to $0$.
\end{proof}

\begin{lemma}[Reduction of overlap bound to norm/angle bounds]\label{lem:R1_overlap_reduction}
With notation as above, set
\[
y_{k,n}(z):=R_k(z)^{-1}U_k(z)w_n(z).
\]
Assume there exist $m(z),\eta(z)>0$ such that for all $n\ge1,\ 0\le k<n$:
\[
\|y_{k,n}(z)\|\ge m(z),\qquad
\left|\left\langle u_k,\frac{y_{k,n}(z)}{\|y_{k,n}(z)\|}\right\rangle\right|^2\ge \eta(z).
\]
Then
\[
\left|\langle u_k,y_{k,n}(z)\rangle\right|^2\ge m(z)^2\eta(z),
\]
so Lemma~\ref{lem:R1_energy_coercive} applies with $c(z)=m(z)^2\eta(z)$.
\end{lemma}

\begin{proof}
Using $y_{k,n}=R_k^{-1}U_k w_n$,
\[
|\langle u_k,y_{k,n}\rangle|^2
=\|y_{k,n}\|^2
\left|\left\langle u_k,\frac{y_{k,n}}{\|y_{k,n}\|}\right\rangle\right|^2
\ge m(z)^2\eta(z).
\]
\end{proof}

\subsubsection{(v3.2) Making the overlap bound explicit (explicit disk-map coordinates)}

The two lemmas above isolate the \emph{only} genuinely nontrivial input needed to
close the coercive branch:
\begin{equation}\label{eq:R1_overlap_goal}
\exists\,c(z)>0\text{ such that }\Bigl|\bigl\langle u_k,\;y_{k,n}(z)\bigr\rangle\Bigr|^2\ge c(z)
\quad\text{for all }n\ge1,\ 0\le k<n,
\end{equation}
where $y_{k,n}(z)=R_k(z)^{-1}U_k(z)w_n(z)$.

In the $\xi$--derived chain, the blocks $H_k$ are rank-one and admit a Schur-parameterization.
The purpose of this subsection is to record a fully \emph{algebraic} pipeline that rewrites
\eqref{eq:R1_overlap_goal} as an explicit rational inequality in the Schur parameters.
This makes it easy to track (and later discharge) each remaining bottleneck without any
appeal to limit-point, global Schur/Herglotz properties, or Weyl-limit uniqueness.

\paragraph{Reader's roadmap (what is reduced vs.\ what remains).}
Fix $z\in\UHP$. The purpose of the v3.2 insertions is to make the coercive/internal-closure route completely local and auditable:
\begin{enumerate}
\item Lemmas~\ref{lem:R1_energy_coercive} and \ref{lem:R1_overlap_reduction} show that the collapse bound from \eqref{eq:R1-radius-energy}
reduces to establishing a uniform lower bound \eqref{eq:R1_overlap_goal} on the rank-one overlaps.
\item Lemma~\ref{lem:R1_step_disk_map_pgl} rewrites the one-step transfer action as an explicit disk M\"obius map in the locked value gauge.
\item Lemma~\ref{lem:R1_overlap_coordinate} rewrites the overlap coordinate as an explicit rational expression in $(\alpha,z)$.
\item The only nontrivial quantitative input left is to preclude near-zeros of the induced denominators
$c_{\alpha_k,z}\,w_{k,n}(z)+d_{\alpha_k,z}$ uniformly in $n,k$, isolated in Definition~\ref{def:R1_denom_sep}.
\item Among the elementary sufficient conditions recorded below, the primary target of this draft is the pole-exclusion criterion
(Lemma~\ref{lem:R1_denom_sep_from_pole_exclusion}), since it reduces Definition~\ref{def:R1_denom_sep} to a quantitative separation of the explicit
denominator root $p_{\alpha,z}:=-d_{\alpha,z}/c_{\alpha,z}$ from the unit disk.
\end{enumerate}

\paragraph{Schur-parameter rank-one blocks.}
Given $\alpha\in\DD$, define the Hermitian rank-one block
\begin{equation}\label{eq:R1_H_alpha_def}
H(\alpha)
:=\frac{1}{1-|\alpha|^2}\begin{pmatrix}1&-\alpha\\-\overline{\alpha}&|\alpha|^2\end{pmatrix}.
\end{equation}
Then $H(\alpha)\succeq0$, $\det H(\alpha)=0$, and $H(\alpha)=\tau(\alpha)\,u(\alpha)u(\alpha)^\ast$ with
\begin{equation}\label{eq:R1_u_tau_alpha}
u(\alpha)=\frac{1}{\sqrt{1+|\alpha|^2}}\begin{pmatrix}1\\-\overline{\alpha}\end{pmatrix},
\qquad
\tau(\alpha)=\operatorname{tr}\bigl(H(\alpha)\bigr)=\frac{1+|\alpha|^2}{1-|\alpha|^2}.
\end{equation}

\begin{lemma}[Canonical one-step disk map (explicit $\mathrm{PGL}(2,\C)$ representative)]\label{lem:R1_step_disk_map_pgl}
Fix $z\in\UHP$ and $\alpha\in\DD$. Let $H:=H(\alpha)$ be as in \eqref{eq:R1_H_alpha_def} and set
\[
L(z):=I-\frac{z}{2}JH,\qquad
R(z):=I+\frac{z}{2}JH,\qquad
M(z):=R(z)^{-1}L(z).
\]
Define the induced disk map
\[
F(w;z):=C_{\mathrm{val}}\!\Bigl(M(z)\cdot C_{\mathrm{val}}^{-1}(w)\Bigr),\qquad w\in\DD,
\]
where $C_{\mathrm{val}}$ is the value Cayley map \eqref{eq:R2_cayley_locked} and $\cdot$ denotes the Möbius action
of $2\times2$ matrices on $\widehat{\C}$.
Then, as Möbius maps (i.e. in $\mathrm{PGL}(2,\C)$), $F(\,\cdot\,;z)$ is represented by the explicit matrix
\begin{equation}\label{eq:R1_step_disk_map_formula}
F(w;z)\ \equiv\ \widehat F_{\alpha,z}\cdot w,
\qquad
\widehat F_{\alpha,z}\ \leftrightarrow\
\begin{pmatrix}
(1+|\alpha|^2)\,z+2i(|\alpha|^2-1) & z(1-|\alpha|^2)+iz(\alpha+\overline{\alpha})\\
z(|\alpha|^2-1)+iz(\alpha+\overline{\alpha}) & -(1+|\alpha|^2)\,z+2i(|\alpha|^2-1)
\end{pmatrix}.
\end{equation}
\noindent
In particular, with the locked value coordinate $C_{\mathrm{val}}$, one generally has $F(0;z)\neq \alpha$ (indeed it is $z$--dependent).
\end{lemma}

\begin{proof}
Let $\Delta:=1-|\alpha|^2>0$. From \eqref{eq:R1_H_alpha_def} one checks
\[
JH(\alpha)=\frac{1}{\Delta}\begin{pmatrix}\overline\alpha&-|\alpha|^2\\1&-\alpha\end{pmatrix}.
\]
Hence
\[
R(z)=I+\frac{z}{2}JH(\alpha)=
\begin{pmatrix}
1+\frac{z\overline\alpha}{2\Delta} & -\frac{z|\alpha|^2}{2\Delta}\\[2pt]
\frac{z}{2\Delta} & 1-\frac{z\alpha}{2\Delta}
\end{pmatrix},
\qquad
L(z)=I-\frac{z}{2}JH(\alpha)=
\begin{pmatrix}
1-\frac{z\overline\alpha}{2\Delta} & \frac{z|\alpha|^2}{2\Delta}\\[2pt]
-\frac{z}{2\Delta} & 1+\frac{z\alpha}{2\Delta}
\end{pmatrix}.
\]
Represent the Cayley maps \eqref{eq:R2_cayley_locked} in $\mathrm{PGL}(2,\C)$ by
\[
C_{\mathrm{val}}\ \leftrightarrow\ \begin{pmatrix}1&-\ii\\1&\ii\end{pmatrix},
\qquad
C_{\mathrm{val}}^{-1}\ \leftrightarrow\ \begin{pmatrix}\ii&\ii\\-1&1\end{pmatrix}.
\]
Since scalar multiples are immaterial in $\mathrm{PGL}(2,\C)$, we may replace $M(z)=R(z)^{-1}L(z)$ by the division-free representative
\[
\widehat M(z):=\operatorname{adj}(R(z))\,L(z),
\qquad
\operatorname{adj}\begin{pmatrix}a&b\\c&d\end{pmatrix}:=\begin{pmatrix}d&-b\\-c&a\end{pmatrix},
\]
because $R^{-1}=(\det R)^{-1}\operatorname{adj}(R)$.
Thus \(F(\,\cdot\,;z)\) is represented by
\[
\widehat F(z):=
\begin{pmatrix}1&-\ii\\1&\ii\end{pmatrix}
\,\operatorname{adj}(R(z))\,L(z)\,
\begin{pmatrix}\ii&\ii\\-1&1\end{pmatrix}.
\]
A direct matrix multiplication gives
\[
\widehat F(z)
=-\frac{1}{\Delta}
\begin{pmatrix}
(1+|\alpha|^2)\,z+2\ii(|\alpha|^2-1) & z(1-|\alpha|^2)+\ii z(\alpha+\overline{\alpha})\\
z(|\alpha|^2-1)+\ii z(\alpha+\overline{\alpha}) & -(1+|\alpha|^2)\,z+2\ii(|\alpha|^2-1)
\end{pmatrix}.
\]
Since $-\frac{1}{\Delta}\neq 0$, this matrix represents the same M\"obius map as the displayed \(\widehat F_{\alpha,z}\) in \eqref{eq:R1_step_disk_map_formula}, proving the claim.
\end{proof}

\begin{remark}[Locked value gauge vs.\ Schur normalization]\label{rem:R1_locked_not_schur}
Lemma~\ref{lem:R1_step_disk_map_pgl} is formulated in the fixed (``locked'') value coordinate $C_{\mathrm{val}}$.
In this gauge the induced disk map $F(\,\cdot\,;z)$ depends on $z$ and is not normalized by $F(0)=\alpha$ in general.
This is why we do \emph{not} appeal to the classical Schur one-step form to obtain denominator control:
the remaining issue is genuinely the quantitative separation of the explicit denominators
$|c_{\alpha_k,z}w_{k,n}(z)+d_{\alpha_k,z}|$, isolated in Definition~\ref{def:R1_denom_sep}.
The standard Schur step is recalled later (Section~\ref{sec:R2}) as a comparison normal form in $\mathrm{Aut}(\DD)$ and for cocycle bookkeeping,
not as a substitute for the R1 bottleneck.
\end{remark}

\paragraph{Disk cocycle notation.}
Fix $z\in\UHP$ and a sequence $(\alpha_k)_{k\ge 0}\subset\DD$.
For each $k\ge 0$, let $F^{\mathrm{val}}_k(\,\cdot\,;z)$ denote the one-step disk map (in the locked value gauge)
associated with $\alpha_k$,
i.e.\ the M\"obius map represented (in $\mathrm{PGL}(2,\C)$) by the matrix $\widehat F_{\alpha_k,z}$
in \eqref{eq:R1_step_disk_map_formula}.
For integers $n\ge 1$ and $0\le k\le n$, define the truncated cocycle iterates by
\[
w_{n,n}(z):=0,\qquad
w_{k,n}(z):=F^{\mathrm{val}}_k\!\bigl(w_{k+1,n}(z);z\bigr),
\]
so that $w_{k,n}(z)=(F^{\mathrm{val}}_k\circ\cdots\circ F^{\mathrm{val}}_{n-1})(0)$.
(Here $0=C_{\mathrm{val}}(i)$ corresponds to the half-plane basepoint $i$ in the locked Cayley gauge.)

\begin{lemma}[Overlap coordinate in the Schur rank-one factorization]\label{lem:R1_overlap_coordinate}
Let $\alpha\in\DD$ and set $u:=u(\alpha)$ as in \eqref{eq:R1_u_tau_alpha}. For $y=(y_1,y_2)^\top\in\C^2$ one has
\begin{equation}\label{eq:R1_overlap_coordinate_formula}
\bigl|\langle u,y\rangle\bigr|^2=\frac{|y_1-\alpha\,y_2|^2}{1+|\alpha|^2}.
\end{equation}
\end{lemma}

\begin{proof}
Immediate from the definition of $u(\alpha)$.
\end{proof}

\begin{definition}[Uniform denominator separation (v3.2 bottleneck)]\label{def:R1_denom_sep}
Fix $z\in\UHP$ and a sequence $(\alpha_k)_{k\ge 0}\subset\DD$.
With $F^{\mathrm{val}}_k(\,\cdot\,;z)$ and $w_{k,n}(z)$ as above, define
\[
c_{\alpha,z}:=z(|\alpha|^2-1)+iz(\alpha+\overline{\alpha}),
\qquad
d_{\alpha,z}:=-(1+|\alpha|^2)\,z+2i(|\alpha|^2-1),
\]
so that $\begin{psmallmatrix}*&*\\ c_{\alpha,z}&d_{\alpha,z}\end{psmallmatrix}$ is the bottom row of the explicit representative
matrix in \eqref{eq:R1_step_disk_map_formula}.
We say the truncated disk cocycle has \emph{uniform denominator separation at $z$} if there exists $\delta(z)>0$
such that for all $n\ge 1$ and $0\le k<n$,
\[
\bigl|c_{\alpha_k,z}\,w_{k,n}(z)+d_{\alpha_k,z}\bigr|\ge \delta(z).
\]
\end{definition}

\begin{remark}[Terminology: denominator ``poles'']\label{rem:R1_denom_pole_terminology}
In the lemmas below, the word ``pole'' refers to the root $p_{\alpha,z}=-d_{\alpha,z}/c_{\alpha,z}$ of the M\"obius denominator in the \emph{$w$-variable}.
It should not be confused with analytic poles of $W$ or $H$ in the \emph{$z$-variable} treated elsewhere in the paper.
\end{remark}

\begin{remark}[Algebraic location of the $w$-denominator root]\label{rem:R1_w_pole_outside_disk}
For $\Impart(z)>0$ and $\alpha\in\DD$, the explicit coefficients in Definition~\ref{def:R1_denom_sep} satisfy
\[
|d_{\alpha,z}|^2-|c_{\alpha,z}|^2
=4|z|^2(\Impart\alpha)^2+4(1-|\alpha|^2)(1+|\alpha|^2)\Impart(z)+4(1-|\alpha|^2)^2>0.
\]
In particular $|d_{\alpha,z}|>|c_{\alpha,z}|$, hence the $w$-denominator root
$p_{\alpha,z}=-d_{\alpha,z}/c_{\alpha,z}$ obeys $|p_{\alpha,z}|=|d_{\alpha,z}|/|c_{\alpha,z}|>1$.
Therefore the substantive content of Lemma~\ref{lem:R1_denom_sep_from_pole_exclusion} is the existence of a \emph{uniform gap}
$|p_{\alpha_k,z}|\ge 1+\varepsilon(z)$ across the sequence $(\alpha_k)$, not merely $|p_{\alpha_k,z}|>1$ pointwise.
A convenient sufficient upstream hypothesis is a uniform Schur-radius bound $|\alpha_k|\le r_0<1$,
which yields an explicit gap $\varepsilon(z,r_0)>0$ by Lemma~\ref{lem:R1_uniform_pole_gap_from_alpha_bound}.
\end{remark}

\paragraph{Elementary sufficient conditions for Definition~\ref{def:R1_denom_sep}.}
The definition above isolates the genuine obstruction.  The following lemmas record a few
simple \emph{non-circular} sufficient conditions that may be proved directly from upstream
information about $(\alpha_k)$ and the cocycle iterates $w_{k,n}(z)$.

\begin{lemma}[Crude coefficient bounds]\label{lem:R1_denom_coeff_bounds}
Fix $z\in\UHP$ and $\alpha\in\DD$. Then the coefficients in Definition~\ref{def:R1_denom_sep} satisfy
\[
|c_{\alpha,z}|\le 3|z|,
\qquad
|d_{\alpha,z}|\ge \Impart(z).
\]
In particular, $d_{\alpha,z}\neq 0$ for all $\alpha\in\DD$.
\end{lemma}

\begin{proof}
For $c_{\alpha,z}=z\bigl((|\alpha|^2-1)+i(\alpha+\overline\alpha)\bigr)$ we have
$|\alpha+\overline\alpha|\le 2|\alpha|\le 2$ and $||\alpha|^2-1|\le 1$, hence
\[
|c_{\alpha,z}|
\le |z|\Bigl(\bigl||\alpha|^2-1\bigr|+|\alpha+\overline\alpha|\Bigr)
\le 3|z|.
\]
For $d_{\alpha,z}=-(1+|\alpha|^2)z+2i(|\alpha|^2-1)$, write $z=x+iy$ with $y=\Impart(z)>0$.
Then
\[
\Impart(d_{\alpha,z})
=-(1+|\alpha|^2)\,y+2(|\alpha|^2-1)<0,
\]
so $|d_{\alpha,z}|\ge |\Impart(d_{\alpha,z})|\ge (1+|\alpha|^2)\,y\ge y=\Impart(z)$.
\end{proof}

\begin{lemma}[Denominator separation from a strict interior iterate bound]\label{lem:R1_denom_sep_from_strict_inside}
Fix $z\in\UHP$ and a sequence $(\alpha_k)\subset\DD$.
Assume there exists $0\le q(z)<1$ such that for all $n\ge 1$ and $0\le k<n$,
\[
|w_{k,n}(z)|\le q(z).
\]
Then the uniform denominator separation property of Definition~\ref{def:R1_denom_sep} holds
whenever
\[
\delta(z):=\Impart(z)-3|z|\,q(z)>0.
\]
\end{lemma}

\begin{proof}
By the reverse triangle inequality,
\[
|c_{\alpha_k,z}\,w_{k,n}(z)+d_{\alpha_k,z}|
\ge |d_{\alpha_k,z}|-|c_{\alpha_k,z}|\,|w_{k,n}(z)|.
\]
Apply Lemma~\ref{lem:R1_denom_coeff_bounds} and the hypothesis $|w_{k,n}(z)|\le q(z)$ to get
\[
|c_{\alpha_k,z}\,w_{k,n}(z)+d_{\alpha_k,z}|
\ge \Impart(z)-3|z|\,q(z)=\delta(z).
\]
\end{proof}

\begin{lemma}[Denominator separation from a coefficient ratio bound]\label{lem:R1_denom_sep_from_ratio}
Fix $z\in\UHP$ and a sequence $(\alpha_k)\subset\DD$. Assume there exists $0\le\rho(z)<1$ such that for all $k\ge 0$,
\[
|c_{\alpha_k,z}|\le \rho(z)\,|d_{\alpha_k,z}|.
\]
Then Definition~\ref{def:R1_denom_sep} holds with
\[
\delta(z):=(1-\rho(z))\,\Impart(z).
\]
\end{lemma}

\begin{proof}
For any $w$ with $|w|\le 1$,
\[
|c_{\alpha_k,z}\,w+d_{\alpha_k,z}|
\ge |d_{\alpha_k,z}|-|c_{\alpha_k,z}|\,|w|
\ge (1-\rho(z))\,|d_{\alpha_k,z}|.
\]
Now apply Lemma~\ref{lem:R1_denom_coeff_bounds} to bound $|d_{\alpha_k,z}|\ge \Impart(z)$.
In particular, the estimate holds for $w=w_{k,n}(z)$ whenever $|w_{k,n}(z)|\le 1$.
\end{proof}

\begin{lemma}[Uniform pole gap from a uniform Schur-radius bound]\label{lem:R1_uniform_pole_gap_from_alpha_bound}
Fix $z\in\UHP$ and $r_0$ with $0\le r_0<1$.  Define
\[
M(z,r_0):=|z|^2(1+r_0^2)^2,\qquad
C(z,r_0):=4(1-r_0^4)\Impart(z)+4(1-r_0^2)^2,
\]
and set $t_0(z,r_0):=C(z,r_0)/M(z,r_0)$ and $\varepsilon(z,r_0):=t_0(z,r_0)/(2+t_0(z,r_0))$.
Then $\varepsilon(z,r_0)>0$ and for every $\alpha\in\DD$ with $|\alpha|\le r_0$, the denominator root
\[
p_{\alpha,z}:=-\frac{d_{\alpha,z}}{c_{\alpha,z}}
\]
satisfies the uniform gap
\[
|p_{\alpha,z}|\ge 1+\varepsilon(z,r_0).
\]
\end{lemma}

\begin{proof}
Since $z\in\UHP$ we have $z\neq 0$.  If $\alpha\in\DD$ then $c_{\alpha,z}\neq 0$ (indeed, the factor
$(|\alpha|^2-1)+i(\alpha+\overline{\alpha})$ has negative real part), hence $p_{\alpha,z}$ is well-defined.

From Remark~\ref{rem:R1_w_pole_outside_disk} we have the identity
\[
|d_{\alpha,z}|^2-|c_{\alpha,z}|^2
=4|z|^2(\Impart\alpha)^2+4(1-|\alpha|^2)(1+|\alpha|^2)\Impart(z)+4(1-|\alpha|^2)^2.
\]
Using $|\alpha|\le r_0$, we drop the nonnegative first term and bound
\[
(1-|\alpha|^2)(1+|\alpha|^2)=1-|\alpha|^4\ge 1-r_0^4,
\qquad
(1-|\alpha|^2)^2\ge (1-r_0^2)^2,
\]
to get
\[
|d_{\alpha,z}|^2-|c_{\alpha,z}|^2\ge C(z,r_0).
\]
Next, write $c_{\alpha,z}=z\bigl((|\alpha|^2-1)+i(\alpha+\overline{\alpha})\bigr)$.  Since $\alpha+\overline{\alpha}=2\Repart(\alpha)$ is real,
\[
|c_{\alpha,z}|^2
=|z|^2\Bigl((1-|\alpha|^2)^2+|\alpha+\overline{\alpha}|^2\Bigr)
\le |z|^2\bigl((1-|\alpha|^2)^2+4|\alpha|^2\bigr)
=|z|^2(1+|\alpha|^2)^2
\le M(z,r_0).
\]
Therefore
\[
|p_{\alpha,z}|^2=\frac{|d_{\alpha,z}|^2}{|c_{\alpha,z}|^2}
=1+\frac{|d_{\alpha,z}|^2-|c_{\alpha,z}|^2}{|c_{\alpha,z}|^2}
\ge 1+\frac{C(z,r_0)}{M(z,r_0)}
=1+t_0(z,r_0).
\]
For $t\ge 0$ we have the elementary inequality
\[
\left(1+\frac{t}{2+t}\right)^2\le 1+t,
\]
hence with $t=t_0(z,r_0)$ we obtain $(1+\varepsilon(z,r_0))^2\le 1+t_0(z,r_0)\le |p_{\alpha,z}|^2$,
which implies $|p_{\alpha,z}|\ge 1+\varepsilon(z,r_0)$.
Finally, $\varepsilon(z,r_0)>0$ since $C(z,r_0)>0$ (because $r_0<1$) and $M(z,r_0)>0$.
\end{proof}

\begin{lemma}[Denominator separation from pole exclusion]\label{lem:R1_denom_sep_from_pole_exclusion}
Fix $z\in\UHP$ and a sequence $(\alpha_k)\subset\DD$. Define the (unique) denominator root
\[
p_{\alpha,z}:=-\frac{d_{\alpha,z}}{c_{\alpha,z}}.
\]
Assume there exists $\varepsilon(z)>0$ such that for all $k\ge 0$,
\[
|p_{\alpha_k,z}|\ge 1+\varepsilon(z).
\]
Then Definition~\ref{def:R1_denom_sep} holds with
\[
\delta(z):=\frac{\varepsilon(z)}{1+\varepsilon(z)}\,\Impart(z).
\]
\end{lemma}

\begin{proof}
Since $c_{\alpha,z}\neq 0$ for $\alpha\in\DD$ and $z\neq 0$, the root $p_{\alpha,z}$ is well-defined and
\[
c_{\alpha,z}\,w+d_{\alpha,z}=c_{\alpha,z}(w-p_{\alpha,z}).
\]
For any $w$ with $|w|\le 1$,
\[
|w-p_{\alpha_k,z}|\ge |p_{\alpha_k,z}|-|w|\ge |p_{\alpha_k,z}|-1,
\]
hence
\[
|c_{\alpha_k,z}\,w+d_{\alpha_k,z}|
=|c_{\alpha_k,z}|\,|w-p_{\alpha_k,z}|
\ge (|p_{\alpha_k,z}|-1)\,|c_{\alpha_k,z}|
=(|p_{\alpha_k,z}|-1)\,\frac{|d_{\alpha_k,z}|}{|p_{\alpha_k,z}|}
=\left(1-\frac{1}{|p_{\alpha_k,z}|}\right)\,|d_{\alpha_k,z}|
\ge \left(1-\frac{1}{1+\varepsilon(z)}\right)\,|d_{\alpha_k,z}|
= \frac{\varepsilon(z)}{1+\varepsilon(z)}\,|d_{\alpha_k,z}|.
\]
Finally apply Lemma~\ref{lem:R1_denom_coeff_bounds} to bound $|d_{\alpha_k,z}|\ge \Impart(z)$.
\end{proof}

\paragraph{Bottleneck checklist (explicit, non-circular).}
Combining Lemma~\ref{lem:R1_step_disk_map_pgl} and Lemma~\ref{lem:R1_overlap_coordinate}, the coercive target
\eqref{eq:R1_overlap_goal} reduces to bounding a \emph{finite list of explicit rational expressions}
in $(\alpha_k,z)$ evaluated along the truncated disk cocycle iterates $w_{k,n}(z)$.
In particular, a sufficient condition (for fixed $z$) is the uniform denominator separation property
in Definition~\ref{def:R1_denom_sep}.
This subsection makes these remaining bottlenecks explicit, so they can be attacked directly (and locally) without
any appeal to limit-point or global Schur/Herglotz structure.

\subsection{Limit-point criterion and closure of R1}

\begin{lemma}[Global Lagrange identity on truncations]\label{lem:R1_lagrange_identity_global}
Fix $z\in\UHP$. For every $n\ge 1$ and every $v\in\C^2$,
\begin{equation}\label{eq:R1-lagrange-global}
\frac{\langle U_n(z)v,\;J\,U_n(z)v\rangle-\langle v,\;Jv\rangle}{2i}
=\Impart(z)\sum_{k=0}^{n-1}
\left\langle R_k(z)^{-1}U_k(z)v,\;
H_k\,R_k(z)^{-1}U_k(z)v\right\rangle .
\end{equation}
In particular, the right-hand side is nonnegative and nondecreasing in $n$.
\end{lemma}

\begin{proof}
Identity \eqref{eq:R1-energy-n} gives
\[
\frac{U_n(z)^\ast J U_n(z)-J}{2i}
=\Impart(z)\,E_n(z),
\]
with
\[
E_n(z)=\sum_{k=0}^{n-1}U_k(z)^\ast R_k(z)^{-*}H_kR_k(z)^{-1}U_k(z).
\]
Evaluate this matrix identity on $v$:
\[
\frac{\langle v,\,(U_n^\ast J U_n-J)v\rangle}{2i}
=\Impart(z)\,\langle v,E_n v\rangle.
\]
Using $\langle v,U_n^\ast J U_n v\rangle=\langle U_n v,J U_n v\rangle$ and
expanding $\langle v,E_n v\rangle$ termwise yields \eqref{eq:R1-lagrange-global}.
Since each summand is $\ge 0$ ($H_k\succeq 0$ and $\Impart z>0$), monotonicity in
$n$ is immediate.
\end{proof}

\begin{lemma}[Two-channel trace comparison]\label{lem:R1_two_channel_trace_compare}
Let $H\succeq 0$ be a Hermitian $2\times2$ matrix, and let
$x,\widetilde x\in\C^2$ with
$X:=[x\ \widetilde x]\in\GL(2,\C)$. Then
\[
\operatorname{tr}(H)
\le
\left\|(X X^\ast)^{-1}\right\|_{\mathrm{op}}
\Big(
\langle x,Hx\rangle+\langle \widetilde x,H\widetilde x\rangle
\Big).
\]
\end{lemma}

\begin{proof}
Since $X$ is invertible,
\[
H=X^{-*}\,(X^\ast H X)\,X^{-1}.
\]
Taking traces,
\[
\operatorname{tr}(H)
=
\operatorname{tr}\!\Big((X^{-1}X^{-*})(X^\ast H X)\Big).
\]
Both matrices in the product are positive semidefinite, so
\[
\operatorname{tr}(AB)\le \|A\|_{\mathrm{op}}\operatorname{tr}(B)
\qquad (A,B\succeq 0).
\]
Apply this with
$A=X^{-1}X^{-*}=(X X^\ast)^{-1}$ and $B=X^\ast H X$:
\[
\operatorname{tr}(H)
\le
\left\|(X X^\ast)^{-1}\right\|_{\mathrm{op}}
\operatorname{tr}(X^\ast H X).
\]
Finally,
\[
\operatorname{tr}(X^\ast H X)
=
\langle x,Hx\rangle+\langle \widetilde x,H\widetilde x\rangle.
\]
\end{proof}

\begin{lemma}[Rank-one transport invariance]\label{lem:R1_rank1_transport_invariance}
Let $H\succeq0$ be real-symmetric with $\operatorname{rank}(H)\le1$, and
set
\[
R(z):=I+\frac z2\,JH,\qquad \Impart z>0.
\]
Then
\[
H\,J\,H=0,\qquad
R(z)^{-*}H\,R(z)^{-1}=H.
\]
\end{lemma}

\begin{proof}
If $H=0$, both identities are trivial. Otherwise
$H=\tau vv^\top$ with $\tau>0$ and $v\in\R^2\setminus\{0\}$.
Since $J^\top=-J$,
\[
v^\top Jv=0,
\]
hence
\[
H J H=\tau^2\,v\,(v^\top Jv)\,v^\top=0.
\]
Also, by Cayley--Hamilton for $2\times2$ trace-zero matrices and
$\det(H)=0$ (rank $\le1$),
\[
(JH)^2=-(\det H)\,I=0,\qquad
(HJ)^2=0.
\]
Therefore
\[
R(z)^{-1}=I-\frac z2\,JH,\qquad
R(z)^{-*}=I+\frac{\overline z}{2}\,HJ.
\]
Multiply out:
\[
R(z)^{-*}H\,R(z)^{-1}
\;=\;
\Bigl(I+\frac{\overline z}{2}HJ\Bigr)\,H\,
\Bigl(I-\frac z2 JH\Bigr)
\;=\;
H+\frac{\overline z}{2}HJH-\frac z2 HJH-\frac{|z|^2}{4}HJHJH
\;=\;H.
\]
\end{proof}

\begin{lemma}[Limit-circle regime plus CS2 forces finite total mass]\label{lem:R1_limit_circle_implies_finite_mass}
Fix $z_0\in\UHP$. Assume there exists $r_0>0$ and an infinite subsequence
$(n_j)$ such that
\[
R_{n_j}(z_0)\ge r_0\qquad(j\ge 1).
\]
Assume moreover there exists $\kappa>0$ such that
\[
\left\|(Y_{k,j}Y_{k,j}^\ast)^{-1}\right\|_{\mathrm{op}}\le \kappa
\qquad(0\le k<n_j,\ j\ge1),
\]
where $Y_{k,j}:=U_k(z_0)U_{n_j}(z_0)^{-1}$.
Then
\[
\sum_{k=0}^{\infty}\operatorname{tr}(H_k)<\infty.
\]
\end{lemma}

\begin{proof}[Proof (algebraic reduction; remaining frame bound \textup{(CS2)})]
By \eqref{eq:R1-radius-energy}, for
$w_n(z_0):=U_n(z_0)^{-1}e_1$ one has
\[
\langle w_{n_j}(z_0),E_{n_j}(z_0)w_{n_j}(z_0)\rangle
\le \frac{1}{2\,\Impart(z_0)\,r_0}
=:C_1 .
\]
Set
\[
x_{k,j}:=R_k(z_0)^{-1}U_k(z_0)w_{n_j}(z_0),
\qquad 0\le k<n_j.
\]
With
\[
y_{k,j}:=U_k(z_0)w_{n_j}(z_0),
\]
Lemma~\ref{lem:R1_rank1_transport_invariance} gives
\[
\langle x_{k,j},H_k x_{k,j}\rangle
=
\langle y_{k,j},H_k y_{k,j}\rangle.
\]
Then
\[
\sum_{k=0}^{n_j-1}\langle y_{k,j},H_k y_{k,j}\rangle\le C_1.
\]

Define similarly $\widetilde w_n(z_0):=U_n(z_0)^{-1}e_2$ and
\[
\widetilde x_{k,j}:=R_k(z_0)^{-1}U_k(z_0)\widetilde w_{n_j}(z_0).
\]
Set
\[
\widetilde y_{k,j}:=U_k(z_0)\widetilde w_{n_j}(z_0).
\]
Again by Lemma~\ref{lem:R1_rank1_transport_invariance},
\[
\langle \widetilde x_{k,j},H_k \widetilde x_{k,j}\rangle
=
\langle \widetilde y_{k,j},H_k \widetilde y_{k,j}\rangle.
\]
Applying Lemma~\ref{lem:R1_lagrange_identity_global} to $v=e_2$ gives
\[
\sum_{k=0}^{n_j-1}\langle \widetilde x_{k,j},H_k \widetilde x_{k,j}\rangle
=
\langle \widetilde w_{n_j}(z_0),E_{n_j}(z_0)\widetilde w_{n_j}(z_0)\rangle.
\]
From the $(2,2)$-entry of \eqref{eq:R1-QE-link},
\[
\tilde q_{22}^{(n)}(z_0)
=
-\Impart(z_0)\,
\langle \widetilde w_n(z_0),E_n(z_0)\widetilde w_n(z_0)\rangle.
\]
Let
\[
\mathcal D_n(z_0)=\{\,m\in\C:\ |m-c_n(z_0)|\le R_n(z_0)\,\}
\]
as in \eqref{eq:R1-disk-center}. Since the Weyl disks are nested,
\[
\mathcal D_{n_j}(z_0)\subseteq \mathcal D_{n_1}(z_0)\qquad(j\ge1),
\]
hence each center belongs to a fixed bounded disk:
\[
|c_{n_j}(z_0)|\le |c_{n_1}(z_0)|+R_{n_1}(z_0)=:M_0.
\]
From \eqref{eq:R1-circle-expanded} and \eqref{eq:R1-disk-center},
\[
\tilde q_{22}^{(n)}(z_0)
\;=\;
\tilde q_{11}^{(n)}(z_0)\Bigl(|c_n(z_0)|^2-R_n(z_0)^2\Bigr).
\]
For $n=n_j$, we have $R_{n_j}(z_0)\ge r_0$, so by \eqref{eq:R1-radius-q11},
\[
\bigl|\tilde q_{11}^{(n_j)}(z_0)\bigr|
=\frac1{2R_{n_j}(z_0)}
\le \frac1{2r_0}.
\]
Therefore
\[
-\tilde q_{22}^{(n_j)}(z_0)
\le
\bigl|\tilde q_{11}^{(n_j)}(z_0)\bigr|\,|c_{n_j}(z_0)|^2
\le
\frac{M_0^2}{2r_0}.
\]
Using the $(2,2)$ identity above yields the uniform second-channel bound
\[
\sum_{k=0}^{n_j-1}\langle \widetilde y_{k,j},H_k \widetilde y_{k,j}\rangle
\le
\frac{M_0^2}{2r_0\,\Impart(z_0)}
=:C_2 .
\]

For each $(k,j)$ with $k<n_j$, set
$Y_{k,j}:=[y_{k,j}\ \widetilde y_{k,j}]=U_k(z_0)U_{n_j}(z_0)^{-1}\in\GL(2,\C)$ and
\[
\kappa_{k,j}:=\left\|(Y_{k,j}Y_{k,j}^\ast)^{-1}\right\|_{\mathrm{op}}.
\]
By
$\det U_n=1$ (Lemma~\ref{lem:R1-det1}),
$\det Y_{k,j}=1$. By hypothesis,
\[
\kappa_{k,j}\le \kappa
\qquad(0\le k<n_j,\ j\ge1).
\]
For later reference we record this as
\begin{equation}\label{eq:R1-CS2}
\exists\,\kappa=\kappa(z_0,r_0)>0:\ 
\kappa_{k,j}\le \kappa
\qquad (0\le k<n_j,\ j\ge1).
\end{equation}
Applying Lemma~\ref{lem:R1_two_channel_trace_compare} with
$H=H_k$, $x=y_{k,j}$, $\widetilde x=\widetilde y_{k,j}$ gives
\[
\operatorname{tr}(H_k)
\le
\kappa_{k,j}\Big(
\langle y_{k,j},H_k y_{k,j}\rangle
+
\langle \widetilde y_{k,j},H_k \widetilde y_{k,j}\rangle
\Big).
\]
Hence, for any $N\ge1$, choose $j$ with $n_j>N$ and sum over
$k=0,\dots,N-1$:
\[
\sum_{k=0}^{N-1}\operatorname{tr}(H_k)
\le
\kappa
\sum_{k=0}^{n_j-1}\langle y_{k,j},H_k y_{k,j}\rangle
+
\kappa
\sum_{k=0}^{n_j-1}\langle \widetilde y_{k,j},H_k \widetilde y_{k,j}\rangle
\le
\kappa(C_1+C_2).
\]
The right-hand side is independent of $N$, so letting $N\to\infty$ yields
\[
\sum_{k=0}^{\infty}\operatorname{tr}(H_k)<\infty.
\]
\end{proof}

The following equivalent forms are often more convenient.
\begin{lemma}[Equivalent forms of CS2]\label{lem:R1_CS2_equiv}
For each $0\le k<n_j$,
\[
Y_{k,j}
=
U_k(z_0)U_{n_j}(z_0)^{-1}
=
\big(M_{n_j-1}(z_0)\cdots M_k(z_0)\big)^{-1}.
\]
If $s_1(Y_{k,j})\ge s_2(Y_{k,j})>0$ are singular values, then
\[
s_1(Y_{k,j})\,s_2(Y_{k,j})=1,\qquad
\kappa_{k,j}
=
\left\|(Y_{k,j}Y_{k,j}^\ast)^{-1}\right\|_{\mathrm{op}}
=
s_1(Y_{k,j})^2
=
\|Y_{k,j}\|_{\mathrm{op}}^2
=
\|Y_{k,j}^{-1}\|_{\mathrm{op}}^2.
\]
Consequently, \eqref{eq:R1-CS2} is equivalent to the tail-cocycle bound
\begin{equation}\label{eq:R1-CS2-tail}
\exists\,K=K(z_0,r_0)>0:\ 
\left\|
\big(M_{n_j-1}(z_0)\cdots M_k(z_0)\big)^{-1}
\right\|_{\mathrm{op}}
\le K
\quad(0\le k<n_j,\ j\ge1).
\end{equation}
\end{lemma}
\begin{proof}
Since
$U_{n_j}(z_0)=M_{n_j-1}(z_0)\cdots M_k(z_0)\,U_k(z_0)$,
right-multiplication by $U_{n_j}(z_0)^{-1}$ gives the tail-product identity.
By Lemma~\ref{lem:R1-det1}, each $\det M_\ell=1$, hence
$\det Y_{k,j}=1$.
Let $\lambda_1\ge\lambda_2>0$ be the eigenvalues of $Y_{k,j}Y_{k,j}^\ast$.
Then $\lambda_r=s_r(Y_{k,j})^2$ and
$\lambda_1\lambda_2=\det(Y_{k,j}Y_{k,j}^\ast)=|\det Y_{k,j}|^2=1$.
So
\[
\kappa_{k,j}
=
\|(Y_{k,j}Y_{k,j}^\ast)^{-1}\|_{\mathrm{op}}
=
\max\{\lambda_1^{-1},\lambda_2^{-1}\}
=
\lambda_1
=
s_1(Y_{k,j})^2
=
\|Y_{k,j}\|_{\mathrm{op}}^2.
\]
Because $s_2=1/s_1$, also
$\|Y_{k,j}^{-1}\|_{\mathrm{op}}=1/s_2=s_1$.
Taking square roots in \eqref{eq:R1-CS2} yields
\eqref{eq:R1-CS2-tail}, and conversely squaring
\eqref{eq:R1-CS2-tail} yields \eqref{eq:R1-CS2}.
\end{proof}

\begin{lemma}[General one-step inverse bound in the nonnegative-\texorpdfstring{$\Re(z^2)$}{Re(z^2)} sector]\label{lem:R1_general_step_inverse_bound_sector}
Let $H\succeq0$ be real-symmetric $2\times2$, and define
\[
L(z):=I-\frac z2 JH,\qquad
R(z):=I+\frac z2 JH,\qquad
M(z):=R(z)^{-1}L(z).
\]
If $z\in\UHP$ satisfies $\Re(z^2)\ge0$, then
\[
\|M(z)^{-1}\|_{\mathrm{op}}
\le
1+|z|\,\operatorname{tr}(H).
\]
\end{lemma}
\begin{proof}
Set
\[
A:=\frac z2 JH,\qquad c:=\frac{z^2}{4},\qquad d:=\det(H)\ (\ge0).
\]
Since $\operatorname{tr}(JH)=0$ and $\det(JH)=\det(H)$, we have
\[
\operatorname{tr}(A)=0,\qquad \det(A)=cd.
\]
For $2\times2$ trace-zero matrices, $A^2=-\det(A)I$, hence
\[
A^2=-cd\,I.
\]
Therefore
\[
(I-A)(I+A)=I-A^2=(1+cd)I.
\]
Now $\Re(c)\ge0$ by hypothesis, so for $d\ge0$,
\[
|1+cd|^2=(1+\Re(c)d)^2+(\Im(c)d)^2\ge1,
\]
thus $1+cd\neq0$ and
\[
(I-A)^{-1}=\frac{1}{1+cd}(I+A).
\]
Hence
\[
M(z)^{-1}
=
(I-A)^{-1}(I+A)
=
\frac{(I+A)^2}{1+cd}
=
\frac{(1-cd)I+zJH}{1+cd}.
\]
So
\[
\|M(z)^{-1}\|_{\mathrm{op}}
\le
\frac{|1-cd|}{|1+cd|}
\;+\;
\frac{|z|}{|1+cd|}\,\|JH\|_{\mathrm{op}}.
\]
Since $\Re(c)\ge0$ and $d\ge0$,
\[
|1-cd|^2=(1-\Re(c)d)^2+(\Im(c)d)^2
\le
(1+\Re(c)d)^2+(\Im(c)d)^2
=
|1+cd|^2,
\]
thus $|1-cd|/|1+cd|\le1$. Also $|1+cd|\ge1$, and
$\|JH\|_{\mathrm{op}}=\|H\|_{\mathrm{op}}\le\operatorname{tr}(H)$ for PSD $H$.
Therefore
\[
\|M(z)^{-1}\|_{\mathrm{op}}
\le
1+|z|\,\operatorname{tr}(H).
\]
\end{proof}

\begin{corollary}[Tail-window bound implies CS2 in the nonnegative-\texorpdfstring{$\Re(z_0^2)$}{Re(z0^2)} sector]\label{cor:R1_CS2_tail_mass_window_sector}
Fix $z_0\in\UHP$ with $\Re(z_0^2)\ge0$ and a subsequence $(n_j)$.
Assume \eqref{eq:R1-tail-window}.
Then \eqref{eq:R1-CS2-tail} holds, hence \eqref{eq:R1-CS2} holds.
More precisely, one can take
\[
K=\exp(|z_0|B),\qquad
\kappa=\exp(2|z_0|B).
\]
\end{corollary}
\begin{proof}
By Lemma~\ref{lem:R1_general_step_inverse_bound_sector}, for each $\ell$,
\[
\|M_\ell(z_0)^{-1}\|_{\mathrm{op}}
\le
1+|z_0|\,\operatorname{tr}(H_\ell).
\]
Hence, for $0\le k<n_j$,
\[
\left\|
\big(M_{n_j-1}(z_0)\cdots M_k(z_0)\big)^{-1}
\right\|_{\mathrm{op}}
\le
\prod_{\ell=k}^{n_j-1}\Bigl(1+|z_0|\,\operatorname{tr}(H_\ell)\Bigr).
\]
Using $1+t\le e^t$ and \eqref{eq:R1-tail-window},
\[
\left\|
\big(M_{n_j-1}(z_0)\cdots M_k(z_0)\big)^{-1}
\right\|_{\mathrm{op}}
\le
\exp\!\left(
|z_0|\sum_{\ell=k}^{n_j-1}\operatorname{tr}(H_\ell)
\right)
\le
\exp(|z_0|B).
\]
This is \eqref{eq:R1-CS2-tail} with $K=\exp(|z_0|B)$.
Lemma~\ref{lem:R1_CS2_equiv} gives \eqref{eq:R1-CS2} with
$\kappa=K^2=\exp(2|z_0|B)$.
\end{proof}

\begin{lemma}[Exact rank-one step linearization]\label{lem:R1_rank1_step_linear}
Let $H\succeq 0$ be real-symmetric with $\operatorname{rank}(H)\le 1$, and
set
\[
L(z):=I-\frac z2 JH,\qquad
R(z):=I+\frac z2 JH,\qquad
M(z):=R(z)^{-1}L(z).
\]
Then, for every $z\in\C$,
\[
M(z)=I-zJH,\qquad
M(z)^{-1}=I+zJH.
\]
\end{lemma}
\begin{proof}
By rank $\le 1$, $\det(H)=0$, hence as in
Lemma~\ref{lem:R1_rank1_transport_invariance},
\[
(JH)^2=0.
\]
Therefore
\[
R(z)^{-1}=I-\frac z2 JH,\qquad
M(z)=\Bigl(I-\frac z2 JH\Bigr)\Bigl(I-\frac z2 JH\Bigr)
=I-zJH.
\]
Since $(JH)^2=0$,
\[
\bigl(I-zJH\bigr)\bigl(I+zJH\bigr)=I,
\]
so $M(z)^{-1}=I+zJH$.
\end{proof}

\begin{lemma}[Mixed rank-one factors and symplectic coupling]\label{lem:R1_rank1_mixed_factor}
Let
\[
H_a=\tau_a u_a u_a^\top,\qquad
H_b=\tau_b u_b u_b^\top,
\]
with $\tau_a,\tau_b>0$ and $u_a,u_b\in\R^2$, $\|u_a\|=\|u_b\|=1$.
Then
\[
JH_a\,JH_b
=
\tau_a\tau_b\,(u_a^\top J u_b)\,J u_a\,u_b^\top.
\]
In particular,
\[
JH_a\,JH_b=0
\iff
u_a^\top J u_b=0.
\]
\end{lemma}
\begin{proof}
Direct multiplication gives
\[
JH_a\,JH_b
=
J(\tau_a u_a u_a^\top)\,J(\tau_b u_b u_b^\top)
=
\tau_a\tau_b\,J u_a\,(u_a^\top J u_b)\,u_b^\top.
\]
The stated equivalence follows immediately.
\end{proof}

\begin{corollary}[Exact linear tail formula under pairwise symplectic orthogonality]\label{cor:R1_linear_tail_under_symplectic_orth}
Fix $z_0\in\UHP$ and indices $k<n$.
Assume rank-one representations
$H_\ell=\tau_\ell u_\ell u_\ell^\top$ for $k\le \ell\le n-1$ and
\[
u_p^\top J u_q=0
\qquad (k\le q<p\le n-1).
\]
Then
\[
\big(M_{n-1}(z_0)\cdots M_k(z_0)\big)^{-1}
=
I+z_0\sum_{\ell=k}^{n-1}JH_\ell,
\]
hence
\[
\left\|
\big(M_{n-1}(z_0)\cdots M_k(z_0)\big)^{-1}
\right\|_{\mathrm{op}}
\le
1+|z_0|\sum_{\ell=k}^{n-1}\operatorname{tr}(H_\ell).
\]
\end{corollary}
\begin{proof}
By Lemma~\ref{lem:R1_rank1_step_linear},
\[
M_\ell(z_0)^{-1}=I+z_0JH_\ell.
\]
Expand the ordered product
$\prod_{\ell=n-1}^{k}(I+z_0JH_\ell)$.
Every term of degree $\ge2$ contains a factor
$JH_pJH_q$ with $q<p$, which vanishes by
Lemma~\ref{lem:R1_rank1_mixed_factor} and the orthogonality hypothesis.
So only degree-$0$ and degree-$1$ terms remain, proving the exact linear formula.
The norm bound follows from subadditivity, $\|JH_\ell\|_{\mathrm{op}}=\|H_\ell\|_{\mathrm{op}}$,
and $\|H_\ell\|_{\mathrm{op}}=\operatorname{tr}(H_\ell)$ for rank-one PSD blocks.
\end{proof}

\begin{lemma}[Tail-mass window bound implies CS2]\label{lem:R1_CS2_tail_mass_window}
Fix $z_0\in\UHP$ and a subsequence $(n_j)$. Assume
\begin{equation}\label{eq:R1-tail-window}
\sup_{j\ge 1}\ \sup_{0\le k<n_j}\ \sum_{\ell=k}^{n_j-1}\operatorname{tr}(H_\ell)
\le B<\infty.
\end{equation}
Assume additionally
\begin{equation}\label{eq:R1-tail-window-rank1}
\operatorname{rank}(H_\ell)\le 1
\qquad(0\le \ell<n_j,\ j\ge1).
\end{equation}
Then \eqref{eq:R1-CS2-tail} holds, hence \eqref{eq:R1-CS2} holds.
\end{lemma}
\begin{proof}
By \eqref{eq:R1-tail-window-rank1} and Lemma~\ref{lem:R1_rank1_step_linear},
\[
M_\ell(z_0)^{-1}=I+z_0 JH_\ell.
\]
Hence, for $0\le k<n_j$,
\[
\big(M_{n_j-1}(z_0)\cdots M_k(z_0)\big)^{-1}
=
\prod_{\ell=n_j-1}^{k}\bigl(I+z_0 JH_\ell\bigr).
\]
Using submultiplicativity of $\|\cdot\|_{\mathrm{op}}$,
\[
\left\|
\big(M_{n_j-1}\cdots M_k\big)^{-1}
\right\|_{\mathrm{op}}
\le
\prod_{\ell=k}^{n_j-1}\|I+z_0JH_\ell\|_{\mathrm{op}}.
\]
Since $J^\ast J=I$ (so $J$ is unitary on $\C^2$),
$\|JH_\ell\|_{\mathrm{op}}=\|H_\ell\|_{\mathrm{op}}$.
For rank-one PSD $2\times2$ blocks,
$\|H_\ell\|_{\mathrm{op}}=\operatorname{tr}(H_\ell)$.
Therefore
\[
\|I+z_0JH_\ell\|_{\mathrm{op}}
\le
1+|z_0|\,\operatorname{tr}(H_\ell)
\le
\exp\!\bigl(|z_0|\,\operatorname{tr}(H_\ell)\bigr).
\]
Multiplying over $\ell=k,\dots,n_j-1$ yields
\[
\left\|
\big(M_{n_j-1}\cdots M_k\big)^{-1}
\right\|_{\mathrm{op}}
\le
\exp\!\left(
|z_0|\sum_{\ell=k}^{n_j-1}\operatorname{tr}(H_\ell)
\right)
\le
\exp(|z_0|B).
\]
Thus \eqref{eq:R1-CS2-tail} holds with $K=\exp(|z_0|B)$, and
Lemma~\ref{lem:R1_CS2_equiv} gives \eqref{eq:R1-CS2}.
\end{proof}

\begin{corollary}[Universal exponential control of \texorpdfstring{$\kappa_{k,j}$}{kappa} by tail mass]\label{cor:R1_kappa_tail_exp}
Under the assumptions of Lemma~\ref{lem:R1_CS2_tail_mass_window},
For every $0\le k<n_j$ one has
\[
\kappa_{k,j}
\le
\exp\!\left(
2|z_0|
\sum_{\ell=k}^{n_j-1}\operatorname{tr}(H_\ell)
\right).
\]
In particular, if $\sum_{\ell\ge0}\operatorname{tr}(H_\ell)<\infty$, then
\eqref{eq:R1-CS2} holds (for every $z_0$ and every $r_0$).
\end{corollary}
\begin{proof}
The proof above gives
\[
\|Y_{k,j}\|_{\mathrm{op}}
\le
\exp\!\left(
|z_0|
\sum_{\ell=k}^{n_j-1}\operatorname{tr}(H_\ell)
\right).
\]
Apply Lemma~\ref{lem:R1_CS2_equiv}:
$\kappa_{k,j}=\|Y_{k,j}\|_{\mathrm{op}}^2$.
If total mass is finite, then the exponent is uniformly bounded in $(k,j)$.
\end{proof}

\begin{lemma}[Radius-floor reduction to frame-growth bottleneck]\label{lem:R1_radiusfloor_kappa_bottleneck}
Fix $z_0\in\UHP$, $r_0>0$, and an infinite subsequence $(n_j)$ with
\[
R_{n_j}(z_0)\ge r_0\qquad(j\ge1).
\]
Set
\[
w_{n_j}:=U_{n_j}(z_0)^{-1}e_1,\qquad
\widetilde w_{n_j}:=U_{n_j}(z_0)^{-1}e_2,
\]
\[
y_{k,j}:=U_k(z_0)w_{n_j},\qquad
\widetilde y_{k,j}:=U_k(z_0)\widetilde w_{n_j},
\]
\[
Y_{k,j}:=[y_{k,j}\ \widetilde y_{k,j}],\qquad
\kappa_{k,j}:=\|(Y_{k,j}Y_{k,j}^\ast)^{-1}\|_{\mathrm{op}}.
\]
Then there exist constants $C_1,C_2<\infty$ (independent of $j$) such that
\[
\sum_{k=0}^{n_j-1}\langle y_{k,j},H_k y_{k,j}\rangle\le C_1,\qquad
\sum_{k=0}^{n_j-1}\langle \widetilde y_{k,j},H_k \widetilde y_{k,j}\rangle\le C_2.
\]
Consequently, for every $j$ and every $N$ with $1\le N\le n_j$,
\begin{equation}\label{eq:R1_prefix_trace_kappa_bottleneck}
\sum_{k=0}^{N-1}\operatorname{tr}(H_k)
\le
(C_1+C_2)\,\max_{0\le k<N}\kappa_{k,j}.
\end{equation}
In particular, with $N=n_j$,
\[
\sum_{k=0}^{n_j-1}\operatorname{tr}(H_k)
\le
(C_1+C_2)\,K_j,\qquad
K_j:=\max_{0\le k<n_j}\kappa_{k,j}.
\]
Hence, if $\sum_{k\ge0}\operatorname{tr}(H_k)=\infty$, then (along every such
subsequence) $K_j\to\infty$.
\end{lemma}
\begin{proof}
By \eqref{eq:R1-radius-energy},
\[
\langle w_{n_j},E_{n_j}(z_0)w_{n_j}\rangle
\le \frac{1}{2\,\Impart(z_0)\,r_0}
=:C_1.
\]
Expanding the left side through the definition of $E_{n_j}$ gives
\[
\sum_{k=0}^{n_j-1}\langle y_{k,j},H_k y_{k,j}\rangle\le C_1.
\]

For the second channel, from \eqref{eq:R1-QE-link} (the $(2,2)$ entry),
\[
\tilde q_{22}^{(n)}(z_0)
=
-\Impart(z_0)\,
\langle \widetilde w_n,E_n(z_0)\widetilde w_n\rangle.
\]
Since Weyl disks are nested, $\mathcal D_{n_j}(z_0)\subseteq\mathcal D_{n_1}(z_0)$,
so with
\[
M_0:=|c_{n_1}(z_0)|+R_{n_1}(z_0),
\]
one has $|c_{n_j}(z_0)|\le M_0$. Using \eqref{eq:R1-circle-expanded},
\eqref{eq:R1-disk-center}, and \eqref{eq:R1-radius-q11},
\[
-\tilde q_{22}^{(n_j)}(z_0)
\le
\frac{M_0^2}{2r_0}.
\]
Hence
\[
\langle \widetilde w_{n_j},E_{n_j}(z_0)\widetilde w_{n_j}\rangle
\le
\frac{M_0^2}{2r_0\,\Impart(z_0)}
=:C_2,
\]
and therefore
\[
\sum_{k=0}^{n_j-1}\langle \widetilde y_{k,j},H_k \widetilde y_{k,j}\rangle\le C_2.
\]

Applying Lemma~\ref{lem:R1_two_channel_trace_compare} pointwise with
$x=y_{k,j}$, $\widetilde x=\widetilde y_{k,j}$:
\[
\operatorname{tr}(H_k)
\le
\kappa_{k,j}\Big(
\langle y_{k,j},H_k y_{k,j}\rangle
+
\langle \widetilde y_{k,j},H_k \widetilde y_{k,j}\rangle
\Big).
\]
Summing for $k=0,\dots,N-1$ and taking
$\max_{0\le k<N}\kappa_{k,j}$ outside yields
\eqref{eq:R1_prefix_trace_kappa_bottleneck}, because each partial energy sum is
bounded by the corresponding full sum $\le C_1$ or $\le C_2$.
The final claim follows by $N=n_j$ and monotone divergence of
$\sum_{k=0}^{n_j-1}\operatorname{tr}(H_k)$.
\end{proof}

\begin{corollary}[Quantitative lower bound on frame growth from prefix mass]\label{cor:R1_kappa_lower_from_prefix}
Under the hypotheses of Lemma~\ref{lem:R1_radiusfloor_kappa_bottleneck}, for
every $j$ and every $N$ with $1\le N\le n_j$,
\[
\max_{0\le k<N}\kappa_{k,j}
\ge
\frac{1}{C_1+C_2}\sum_{\ell=0}^{N-1}\operatorname{tr}(H_\ell).
\]
In particular,
\[
K_j
=
\max_{0\le k<n_j}\kappa_{k,j}
\ge
\frac{1}{C_1+C_2}\sum_{\ell=0}^{n_j-1}\operatorname{tr}(H_\ell).
\]
\end{corollary}
\begin{proof}
Rearrange \eqref{eq:R1_prefix_trace_kappa_bottleneck}.
The second claim is the case $N=n_j$.
\end{proof}

\begin{lemma}[Pointwise exponential upper control of \texorpdfstring{$\kappa_{k,j}$}{kappa} by tail mass in the rank-one branch]\label{lem:R1_kappa_pointwise_tail_exp}
Fix $z_0\in\UHP$ and a subsequence $(n_j)$.
Assume
\[
\operatorname{rank}(H_\ell)\le1
\qquad(0\le \ell<n_j,\ j\ge1).
\]
Then for every $j\ge1$ and every $0\le k<n_j$,
\[
\kappa_{k,j}
\le
\exp\!\left(
2|z_0|\sum_{\ell=k}^{n_j-1}\operatorname{tr}(H_\ell)
\right).
\]
\end{lemma}
\begin{proof}
By \eqref{eq:R1-tail-window-rank1} and Lemma~\ref{lem:R1_rank1_step_linear},
\[
M_\ell(z_0)^{-1}=I+z_0JH_\ell.
\]
Hence
\[
Y_{k,j}
=
\big(M_{n_j-1}(z_0)\cdots M_k(z_0)\big)^{-1}
=
\prod_{\ell=n_j-1}^{k}\bigl(I+z_0JH_\ell\bigr).
\]
So
\[
\|Y_{k,j}\|_{\mathrm{op}}
\le
\prod_{\ell=k}^{n_j-1}\|I+z_0JH_\ell\|_{\mathrm{op}}
\le
\prod_{\ell=k}^{n_j-1}\bigl(1+|z_0|\,\|H_\ell\|_{\mathrm{op}}\bigr).
\]
For rank-one PSD blocks, $\|H_\ell\|_{\mathrm{op}}=\operatorname{tr}(H_\ell)$.
Using $1+t\le e^t$ gives
\[
\|Y_{k,j}\|_{\mathrm{op}}
\le
\exp\!\left(
|z_0|\sum_{\ell=k}^{n_j-1}\operatorname{tr}(H_\ell)
\right).
\]
By Lemma~\ref{lem:R1_CS2_equiv},
$\kappa_{k,j}=\|Y_{k,j}\|_{\mathrm{op}}^2$, yielding the claim.
\end{proof}

\begin{corollary}[On a radius-floor rank-one subsequence, \texorpdfstring{$K_j$}{Kj} and prefix mass are equivalent growth scales]\label{cor:R1_Kj_prefix_equiv_rank1}
Under the hypotheses of Lemma~\ref{lem:R1_radiusfloor_kappa_bottleneck}, assume
also
\[
\operatorname{rank}(H_\ell)\le1
\qquad(0\le \ell<n_j,\ j\ge1).
\]
Define
\[
P_j:=\sum_{\ell=0}^{n_j-1}\operatorname{tr}(H_\ell),\qquad
K_j:=\max_{0\le k<n_j}\kappa_{k,j}.
\]
Then
\[
\frac{1}{C_1+C_2}\,P_j
\le
K_j
\le
\exp\!\bigl(2|z_0|P_j\bigr)
\qquad(j\ge1).
\]
In particular,
\[
\sup_j K_j<\infty
\quad\Longleftrightarrow\quad
\sup_j P_j<\infty.
\]
\end{corollary}
\begin{proof}
The lower bound is Corollary~\ref{cor:R1_kappa_lower_from_prefix} with $N=n_j$.
For the upper bound, apply
Lemma~\ref{lem:R1_kappa_pointwise_tail_exp} and use
\[
\sum_{\ell=k}^{n_j-1}\operatorname{tr}(H_\ell)\le P_j
\quad(0\le k<n_j),
\]
then maximize over $k$.
The equivalence follows immediately.
\end{proof}

\begin{corollary}[CS2 is exactly uniform boundedness of \texorpdfstring{$K_j$}{Kj}]\label{cor:R1_CS2_equiv_Kj}
In the setup above, define
\[
K_j:=\max_{0\le k<n_j}\kappa_{k,j}.
\]
Then
\[
\eqref{eq:R1-CS2}
\quad\Longleftrightarrow\quad
\sup_{j\ge1}K_j<\infty.
\]
\end{corollary}
\begin{proof}
By definition,
\[
\eqref{eq:R1-CS2}
\iff
\exists\,\kappa>0:\ \kappa_{k,j}\le \kappa\ \ (0\le k<n_j,\ j\ge1)
\iff
K_j\le \kappa\ \ (j\ge1).
\]
The latter is equivalent to $\sup_j K_j<\infty$.
\end{proof}

\begin{proposition}[Exact rank-one closure reduction on a radius-floor subsequence]\label{prop:R1_rank1_exact_closure_reduction}
Fix $z_0\in\UHP$, $r_0>0$, and an infinite subsequence $(n_j)$ with
\[
R_{n_j}(z_0)\ge r_0.
\]
Assume
\[
\operatorname{rank}(H_\ell)\le1
\qquad(0\le \ell<n_j,\ j\ge1).
\]
Define
\[
P_j:=\sum_{\ell=0}^{n_j-1}\operatorname{tr}(H_\ell),\qquad
K_j:=\max_{0\le k<n_j}\kappa_{k,j}.
\]
Then the following are equivalent:
\begin{enumerate}
\item the frame bound \eqref{eq:R1-CS2} on this subsequence;
\item $\sup_j K_j<\infty$;
\item $\sup_j P_j<\infty$;
\item the tail-window bound \eqref{eq:R1-tail-window} on this subsequence.
\end{enumerate}
\end{proposition}
\begin{proof}
\textup{(1)}$\Leftrightarrow$\textup{(2)} is Corollary~\ref{cor:R1_CS2_equiv_Kj}.
\textup{(2)}$\Leftrightarrow$\textup{(3)} is Corollary~\ref{cor:R1_Kj_prefix_equiv_rank1}.
\textup{(3)}$\Leftrightarrow$\textup{(4)} is
Lemma~\ref{lem:R1_tailwindow_prefix_equiv}.
\end{proof}

\begin{corollary}[Obstruction certificate on a radius-floor subsequence]\label{cor:R1_obstruction_certificate}
Under the hypotheses of Lemma~\ref{lem:R1_radiusfloor_kappa_bottleneck}, define
\[
P_j:=\sum_{\ell=0}^{n_j-1}\operatorname{tr}(H_\ell),\qquad
K_j:=\max_{0\le k<n_j}\kappa_{k,j}.
\]
If $P_j\to\infty$, then $K_j\to\infty$. In particular,
\eqref{eq:R1-CS2} fails on this subsequence.

If moreover $\operatorname{rank}(H_\ell)\le1$ for all
$0\le \ell<n_j$, then \eqref{eq:R1-tail-window} also fails on this subsequence.
\end{corollary}
\begin{proof}
By Corollary~\ref{cor:R1_kappa_lower_from_prefix},
\[
K_j\ge \frac{1}{C_1+C_2}\,P_j,
\]
so $P_j\to\infty$ implies $K_j\to\infty$.
By Corollary~\ref{cor:R1_CS2_equiv_Kj}, this rules out \eqref{eq:R1-CS2}.
In the rank-one case, Proposition~\ref{prop:R1_rank1_exact_closure_reduction}
gives
\[
\eqref{eq:R1-CS2}\Longleftrightarrow \eqref{eq:R1-tail-window},
\]
hence failure of \eqref{eq:R1-CS2} implies failure of \eqref{eq:R1-tail-window}.
\end{proof}

\begin{corollary}[Rank-one radius-floor dichotomy]\label{cor:R1_rank1_radiusfloor_dichotomy}
Fix $z_0\in\UHP$, $r_0>0$, and an infinite subsequence $(n_j)$ with
\[
R_{n_j}(z_0)\ge r_0.
\]
Assume
\[
\operatorname{rank}(H_\ell)\le1
\qquad(0\le \ell<n_j,\ j\ge1).
\]
Set
\[
P_j:=\sum_{\ell=0}^{n_j-1}\operatorname{tr}(H_\ell),\qquad
K_j:=\max_{0\le k<n_j}\kappa_{k,j}.
\]
Then exactly one of the following two alternatives occurs:
\begin{enumerate}
\item \textup{(bounded branch)}\ 
      $\sup_j P_j<\infty$.
      Equivalently,
      \[
      \sup_j K_j<\infty,\qquad
      \eqref{eq:R1-CS2},\qquad
      \eqref{eq:R1-tail-window}
      \]
      all hold on this subsequence.
\item \textup{(divergent branch)}\ 
      $P_j\to\infty$.
      Then
      \[
      K_j\to\infty,
      \]
      so \eqref{eq:R1-CS2} fails; equivalently (rank-one case),
      \eqref{eq:R1-tail-window} fails.
\end{enumerate}
\end{corollary}
\begin{proof}
Because $(n_j)$ is strictly increasing and $\operatorname{tr}(H_\ell)\ge0$,
$P_j$ is monotone nondecreasing. Hence either $\sup_j P_j<\infty$ or
$P_j\to\infty$, and the two cases are exclusive.

In the bounded branch, Proposition~\ref{prop:R1_rank1_exact_closure_reduction}
gives the equivalences with bounded $K_j$, \eqref{eq:R1-CS2}, and
\eqref{eq:R1-tail-window}.
In the divergent branch, Corollary~\ref{cor:R1_obstruction_certificate}
gives $K_j\to\infty$, hence failure of \eqref{eq:R1-CS2}; in the rank-one
setting this is equivalent to failure of \eqref{eq:R1-tail-window} by
Proposition~\ref{prop:R1_rank1_exact_closure_reduction}.
\end{proof}

\begin{lemma}[Prefix mass along subsequences under total-mass divergence]\label{lem:R1_prefix_subsequence_divergence}
Let
\[
S_N:=\sum_{\ell=0}^{N-1}\operatorname{tr}(H_\ell),\qquad N\ge1.
\]
If
\[
\sum_{\ell=0}^{\infty}\operatorname{tr}(H_\ell)=\infty,
\]
then for every strictly increasing sequence $(n_j)$ one has
\[
S_{n_j}\to\infty.
\]
\end{lemma}
\begin{proof}
Each $\operatorname{tr}(H_\ell)\ge0$, so $(S_N)_{N\ge1}$ is monotone
nondecreasing. The hypothesis says $\sup_N S_N=\infty$. Hence for every
$M>0$ there exists $N_M$ with $S_{N_M}>M$. Since $n_j\to\infty$, there is
$j_M$ such that $n_j\ge N_M$ for all $j\ge j_M$, and monotonicity gives
$S_{n_j}\ge S_{N_M}>M$. Therefore $S_{n_j}\to\infty$.
\end{proof}

\begin{corollary}[Under mass divergence, only the divergent branch can occur]\label{cor:R1_mass_divergence_selects_divergent_branch}
Under the hypotheses of Corollary~\ref{cor:R1_rank1_radiusfloor_dichotomy},
assume additionally
\[
\sum_{\ell=0}^{\infty}\operatorname{tr}(H_\ell)=\infty.
\]
Then
\[
P_j=\sum_{\ell=0}^{n_j-1}\operatorname{tr}(H_\ell)\to\infty,
\]
so only branch \textup{(2)} of
Corollary~\ref{cor:R1_rank1_radiusfloor_dichotomy} is possible on this
subsequence. Consequently $K_j\to\infty$, and \eqref{eq:R1-CS2} fails on this
subsequence.
\end{corollary}
\begin{proof}
Apply Lemma~\ref{lem:R1_prefix_subsequence_divergence} with $S_{n_j}=P_j$.
Then Corollary~\ref{cor:R1_rank1_radiusfloor_dichotomy} yields branch
\textup{(2)} and its consequences.
\end{proof}

\begin{corollary}[Rank-one mass-divergence excludes radius-floor subsequences]\label{cor:R1_rank1_closure_excludes_radiusfloor}
Assume
\[
\operatorname{rank}(H_k)\le1\ (k\ge0),\qquad
\sum_{k=0}^{\infty}\operatorname{tr}(H_k)=\infty.
\]
Then for every $z_0\in\UHP$ and every $r_0>0$, the set
\[
\{\,n\ge1:\ R_n(z_0)\ge r_0\,\}
\]
is finite. Equivalently, $R_n(z)\to0$ for every $z\in\UHP$.
\end{corollary}
\begin{proof}
Apply Theorem~\ref{thm:R1_mass_divergence_internal}.
\end{proof}

\begin{lemma}[Bounded \texorpdfstring{$J$}{J}-form does not force CS2]\label{lem:R1_Jform_not_CS2}
For $t>0$, define
\[
A_t:=\begin{pmatrix}t&0\\0&t^{-1}\end{pmatrix}\in \mathrm{SL}(2,\R).
\]
Then
\[
A_t^{-*}JA_t^{-1}=J
\quad\text{for all }t>0,
\]
but
\[
\left\|\bigl(A_tA_t^\ast\bigr)^{-1}\right\|_{\mathrm{op}}=t^2\to\infty
\qquad(t\to\infty).
\]
\end{lemma}
\begin{proof}
Since $A_t^{-1}=\mathrm{diag}(t^{-1},t)$ and $A_t^{-*}=A_t^{-1}$,
\[
A_t^{-*}JA_t^{-1}
=
\begin{pmatrix}t^{-1}&0\\0&t\end{pmatrix}
J
\begin{pmatrix}t^{-1}&0\\0&t\end{pmatrix}
=J.
\]
Also
\[
A_tA_t^\ast=\begin{pmatrix}t^2&0\\0&t^{-2}\end{pmatrix},
\qquad
\bigl(A_tA_t^\ast\bigr)^{-1}
=
\begin{pmatrix}t^{-2}&0\\0&t^2\end{pmatrix},
\]
so its operator norm is $t^2$.
\end{proof}

\begin{remark}[Why the CS2 bottleneck is genuine]\label{rem:R1_CS2_bottleneck_genuine}
Lemma~\ref{lem:R1_Jform_not_CS2} shows that controlling matrices of the form
$A^{-*}JA^{-1}$ (even exactly) does not by itself control Euclidean frame
distortion $\|(AA^\ast)^{-1}\|_{\mathrm{op}}$.
Hence, in the radius-floor analysis, bounds extracted only from
$Q_n=U_n^{-*}JU_n^{-1}$ cannot close \eqref{eq:R1-CS2} without an additional
structural input (for example tail-window/prefix-window control, structured
collinearity, or uniform ellipticity branches developed below).
\end{remark}

\begin{corollary}[Radius-floor plus bounded \texorpdfstring{$K_j$}{Kj} implies finite mass]\label{cor:R1_bounded_Kj_implies_finite_mass}
Under the hypotheses of Lemma~\ref{lem:R1_radiusfloor_kappa_bottleneck}, if
\[
\sup_{j\ge1}K_j<\infty,
\]
then
\[
\sum_{k=0}^{\infty}\operatorname{tr}(H_k)<\infty.
\]
\end{corollary}
\begin{proof}
Set $K_\ast:=\sup_j K_j<\infty$. For any $N\ge1$, choose $j$ with $n_j>N$.
Then \eqref{eq:R1_prefix_trace_kappa_bottleneck} gives
\[
\sum_{k=0}^{N-1}\operatorname{tr}(H_k)
\le
(C_1+C_2)\,\max_{0\le k<N}\kappa_{k,j}
\le
(C_1+C_2)\,K_j
\le
(C_1+C_2)\,K_\ast.
\]
The bound is independent of $N$; letting $N\to\infty$ yields finite total mass.
\end{proof}

\begin{lemma}[Prefix-mass bound directly forces finite total mass]\label{lem:R1_tailwindow_direct_finite_mass}
Let $(n_j)$ be an infinite subsequence and assume there exists $B<\infty$ such that
\begin{equation}\label{eq:R1-prefix-window-direct}
\sup_{j\ge1}\ \sum_{\ell=0}^{n_j-1}\operatorname{tr}(H_\ell)\le B.
\end{equation}
Then
\[
\sum_{k=0}^{\infty}\operatorname{tr}(H_k)<\infty.
\]
\end{lemma}
\begin{proof}
Since $(n_j)$ is an infinite subsequence of $\mathbb N$, we have $n_j\to\infty$.
Hence the monotone prefix sums are uniformly bounded, so
\[
\sum_{k=0}^{\infty}\operatorname{tr}(H_k)=\sup_{N\ge1}\sum_{k=0}^{N-1}\operatorname{tr}(H_k)\le B<\infty.
\]
\end{proof}

\begin{lemma}[Tail-window and prefix-window are equivalent for nonnegative trace]\label{lem:R1_tailwindow_prefix_equiv}
Let $(n_j)$ be an infinite subsequence. Then the tail-window bound
\[
\sup_{j\ge1}\sup_{0\le k<n_j}\sum_{\ell=k}^{n_j-1}\operatorname{tr}(H_\ell)\le B<\infty
\]
holds if and only if the prefix-window bound
\[
\sup_{j\ge1}\sum_{\ell=0}^{n_j-1}\operatorname{tr}(H_\ell)\le B<\infty
\]
holds.
\end{lemma}
\begin{proof}
Since each $H_\ell\succeq0$, one has $\operatorname{tr}(H_\ell)\ge0$. Hence for
every fixed $(j,k)$ with $0\le k<n_j$,
\[
\sum_{\ell=k}^{n_j-1}\operatorname{tr}(H_\ell)
\le
\sum_{\ell=0}^{n_j-1}\operatorname{tr}(H_\ell).
\]
Taking $\sup_{j,k}$ gives
\[
\sup_{j\ge1}\sup_{0\le k<n_j}\sum_{\ell=k}^{n_j-1}\operatorname{tr}(H_\ell)
\le
\sup_{j\ge1}\sum_{\ell=0}^{n_j-1}\operatorname{tr}(H_\ell).
\]
Conversely, choosing $k=0$ inside the left-hand supremum yields
\[
\sup_{j\ge1}\sum_{\ell=0}^{n_j-1}\operatorname{tr}(H_\ell)
\le
\sup_{j\ge1}\sup_{0\le k<n_j}\sum_{\ell=k}^{n_j-1}\operatorname{tr}(H_\ell).
\]
Therefore the two quantities are equal, proving equivalence.
\end{proof}

\begin{theorem}[Independent closure from a prefix-mass principle]\label{thm:R1_rank1_tailwindow_principle}
Assume
\[
\sum_{k=0}^{\infty}\operatorname{tr}(H_k)=\infty.
\]
Assume moreover that for every $z_0\in\UHP$, every $r_0>0$, and every
subsequence $(n_j)$ with $R_{n_j}(z_0)\ge r_0$, there exists
$B=B(z_0,r_0,(n_j))<\infty$ such that
\[
\sup_{j\ge1}\sum_{\ell=0}^{n_j-1}\operatorname{tr}(H_\ell)\le B.
\]
(Equivalently, by Lemma~\ref{lem:R1_tailwindow_prefix_equiv}, the
tail-window bound \eqref{eq:R1-tail-window} holds on this subsequence.)
Then for every $z\in\UHP$,
\[
R_n(z)\to0\qquad(n\to\infty).
\]
\end{theorem}
\begin{proof}
Fix $z\in\UHP$. If $R_n(z)\not\to0$, choose $r_0>0$ and a subsequence $(n_j)$
with $R_{n_j}(z)\ge r_0$. By hypothesis, this subsequence satisfies
\eqref{eq:R1-prefix-window-direct} for some $B<\infty$. By
Lemma~\ref{lem:R1_tailwindow_direct_finite_mass},
\[
\sum_{k=0}^{\infty}\operatorname{tr}(H_k)<\infty,
\]
contradicting the mass-divergence assumption. So $R_n(z)\to0$ for this fixed
$z$. Since $z\in\UHP$ was arbitrary, the conclusion follows on all of $\UHP$.
\end{proof}

\begin{proposition}[Uniform-ellipticity branch closes R1 without CS2]\label{prop:R1_uniform_ellipticity_finite_mass}
Under the radius-floor setup of
Lemma~\ref{lem:R1_radiusfloor_kappa_bottleneck}, assume there exists $\eta>0$
such that
\begin{equation}\label{eq:R1_uniform_ellipticity}
H_k\succeq \eta\,\operatorname{tr}(H_k)\,I_2
\qquad(k\ge0).
\end{equation}
Then
\[
\sum_{k=0}^{\infty}\operatorname{tr}(H_k)<\infty.
\]
\end{proposition}
\begin{proof}
Fix $j$ and $k<n_j$. With $Y_{k,j}=[y_{k,j}\ \widetilde y_{k,j}]$,
\[
\langle y_{k,j},H_k y_{k,j}\rangle
+
\langle \widetilde y_{k,j},H_k \widetilde y_{k,j}\rangle
=
\operatorname{tr}(Y_{k,j}^\ast H_k Y_{k,j}).
\]
By \eqref{eq:R1_uniform_ellipticity},
\[
Y_{k,j}^\ast H_k Y_{k,j}
\succeq
\eta\,\operatorname{tr}(H_k)\,Y_{k,j}^\ast Y_{k,j},
\]
hence
\[
\operatorname{tr}(Y_{k,j}^\ast H_k Y_{k,j})
\ge
\eta\,\operatorname{tr}(H_k)\,\operatorname{tr}(Y_{k,j}^\ast Y_{k,j}).
\]
Now $\det Y_{k,j}=1$ (Lemma~\ref{lem:R1-det1}), so if
$s_1\ge s_2>0$ are singular values of $Y_{k,j}$, then
$s_1s_2=1$ and
\[
\operatorname{tr}(Y_{k,j}^\ast Y_{k,j})=s_1^2+s_2^2\ge2.
\]
Therefore
\[
\langle y_{k,j},H_k y_{k,j}\rangle
+
\langle \widetilde y_{k,j},H_k \widetilde y_{k,j}\rangle
\ge
2\eta\,\operatorname{tr}(H_k).
\]
Summing for $k=0,\dots,N-1$ (with $N\le n_j$):
\[
2\eta\sum_{k=0}^{N-1}\operatorname{tr}(H_k)
\le
\sum_{k=0}^{N-1}\langle y_{k,j},H_k y_{k,j}\rangle
+
\sum_{k=0}^{N-1}\langle \widetilde y_{k,j},H_k \widetilde y_{k,j}\rangle
\le C_1+C_2.
\]
Choose $j$ with $n_j>N$; then
\[
\sum_{k=0}^{N-1}\operatorname{tr}(H_k)\le \frac{C_1+C_2}{2\eta}.
\]
Letting $N\to\infty$ yields finite total mass.
\end{proof}

\begin{corollary}[Mass divergence implies collapse under uniform ellipticity]\label{cor:R1_uniform_ellipticity_collapse}
Assume
\[
\sum_{k=0}^{\infty}\operatorname{tr}(H_k)=\infty
\]
and \eqref{eq:R1_uniform_ellipticity} for some $\eta>0$.
Then for every $z\in\UHP$,
\[
R_n(z)\to0.
\]
\end{corollary}
\begin{proof}
Fix $z\in\UHP$. If $R_n(z)\not\to0$, there exist $r_0>0$ and a subsequence
$(n_j)$ with $R_{n_j}(z)\ge r_0$. Proposition~\ref{prop:R1_uniform_ellipticity_finite_mass}
then implies $\sum_k\operatorname{tr}(H_k)<\infty$, contradicting the hypothesis.
\end{proof}

\begin{remark}[Two internal routes to CS2 from tail-window bounds]\label{rem:R1_two_routes_cs2}
For a fixed $(z_0,r_0,(n_j))$, the frame bound \eqref{eq:R1-CS2} follows from
the tail-window condition \eqref{eq:R1-tail-window} by either of:
\begin{enumerate}
\item Rank-one route:
      add \eqref{eq:R1-tail-window-rank1} and use
      Lemma~\ref{lem:R1_CS2_tail_mass_window}.
\item General-rank sector route:
      assume $\Re(z_0^2)\ge0$ and use
      Corollary~\ref{cor:R1_CS2_tail_mass_window_sector}.
\end{enumerate}
\end{remark}

\begin{proposition}[Structured CS2 closure with explicit polynomial bound]\label{prop:R1_CS2_structured_linear}
Fix $z_0\in\UHP$, $r_0>0$, and a subsequence $(n_j)$ with
$R_{n_j}(z_0)\ge r_0$.
Assume there exists $B<\infty$ such that
\begin{equation}\label{eq:R1-tail-window-structured}
\sup_{j\ge1}\sup_{0\le k<n_j}\sum_{\ell=k}^{n_j-1}\operatorname{tr}(H_\ell)\le B,
\end{equation}
and rank-one vectors $H_\ell=\tau_\ell u_\ell u_\ell^\top$ satisfy
\begin{equation}\label{eq:R1-window-symplectic-orth}
u_p^\top J u_q=0
\qquad(0\le q<p<n_j,\ j\ge1).
\end{equation}
Then \eqref{eq:R1-CS2} holds with
\[
\kappa(z_0,r_0)\le \bigl(1+|z_0|B\bigr)^2.
\]
\end{proposition}
\begin{proof}
By \eqref{eq:R1-window-symplectic-orth},
Corollary~\ref{cor:R1_linear_tail_under_symplectic_orth} gives, for each
$0\le k<n_j$,
\[
\left\|
\big(M_{n_j-1}(z_0)\cdots M_k(z_0)\big)^{-1}
\right\|_{\mathrm{op}}
\le
1+|z_0|\sum_{\ell=k}^{n_j-1}\operatorname{tr}(H_\ell)
\le
1+|z_0|B.
\]
Hence \eqref{eq:R1-CS2-tail} holds with $K=1+|z_0|B$.
Lemma~\ref{lem:R1_CS2_equiv} yields
\[
\kappa_{k,j}\le K^2\le (1+|z_0|B)^2.
\]
\end{proof}

\begin{corollary}[Structured CS2 via one Lagrangian direction per subsequence window]\label{cor:R1_CS2_structured_lagrangian_line}
Under the hypotheses of Proposition~\ref{prop:R1_CS2_structured_linear},
replace \eqref{eq:R1-window-symplectic-orth} by
\begin{equation}\label{eq:R1-window-lagrangian-line}
\forall j\ge1\ \exists v_j\in\R^2\setminus\{0\},\ \exists(c_{\ell,j})_{0\le\ell<n_j}\subset\R:
\quad
u_\ell=c_{\ell,j}v_j\quad(0\le\ell<n_j).
\end{equation}
Then \eqref{eq:R1-CS2} holds with
\[
\kappa(z_0,r_0)\le \bigl(1+|z_0|B\bigr)^2.
\]
\end{corollary}
\begin{proof}
Fix $j\ge1$ and $0\le q<p<n_j$. By \eqref{eq:R1-window-lagrangian-line},
$u_p=c_{p,j}v_j$ and $u_q=c_{q,j}v_j$, hence
\[
u_p^\top J u_q
=
c_{p,j}c_{q,j}v_j^\top J v_j
=0,
\]
since $J$ is skew-symmetric and therefore $x^\top J x=0$ for all $x\in\R^2$.
So \eqref{eq:R1-window-symplectic-orth} holds, and
Proposition~\ref{prop:R1_CS2_structured_linear} applies.
\end{proof}

\begin{proposition}[Prefix-mass bound from collinear windows]\label{prop:R1_prefix_mass_collinear_visible}
Fix $z_0\in\UHP$, $r_0>0$, and a subsequence $(n_j)$ with
$R_{n_j}(z_0)\ge r_0$.
Assume that for each $j\ge1$ there exist $v_j\in\R^2\setminus\{0\}$ and
nonnegative scalars $(a_{\ell,j})_{0\le\ell<n_j}$ such that
\[
H_\ell=a_{\ell,j}\,v_jv_j^\top\qquad(0\le\ell<n_j).
\]
Then
\begin{equation}\label{eq:R1-prefix-bound-collinear-visible}
\sup_{j\ge1}\sum_{\ell=0}^{n_j-1}\operatorname{tr}(H_\ell)
\le
\frac{1+M_0^2}{2\,\Impart(z_0)\,r_0},
\end{equation}
where
\[
M_0:=|c_{n_1}(z_0)|+R_{n_1}(z_0).
\]
\end{proposition}
\begin{proof}
Fix $j$. Set
\[
G_j:=\sum_{\ell=0}^{n_j-1}H_\ell
=
\Bigl(\sum_{\ell=0}^{n_j-1}a_{\ell,j}\Bigr)v_jv_j^\top.
\]
By collinearity, $u_p^\top Ju_q=0$ for $0\le q<p<n_j$, so
Corollary~\ref{cor:R1_linear_tail_under_symplectic_orth} (with $k=0$) gives
\[
U_{n_j}(z_0)^{-1}
=
\big(M_{n_j-1}(z_0)\cdots M_0(z_0)\big)^{-1}
=
I+z_0JG_j.
\]
Hence, with $A_j:=JG_j$ and $Q_{n_j}=U_{n_j}(z_0)^{-*}J\,U_{n_j}(z_0)^{-1}$,
\[
Q_{n_j}
=(I+\overline z_0A_j^\ast)J(I+z_0A_j)
=J+(\overline z_0-z_0)G_j+|z_0|^2A_j^\ast J A_j.
\]
Now $A_j^\ast=-G_jJ$, so
\[
A_j^\ast J A_j=(-G_jJ)J(JG_j)=G_jJG_j=0
\]
because $G_j$ is rank-one real-symmetric. Therefore
\[
Q_{n_j}=J-2i\,\Impart(z_0)\,G_j.
\]
Taking the $(1,1)$ entry in $\widetilde Q_{n_j}=Q_{n_j}/(2i)$:
\[
\tilde q_{11}^{(n_j)}(z_0)
=
-\Impart(z_0)\,(G_j)_{11}
=
-\Impart(z_0)\sum_{\ell=0}^{n_j-1}(H_\ell)_{11}.
\]
By \eqref{eq:R1-radius-q11} and $R_{n_j}(z_0)\ge r_0$,
\[
\sum_{\ell=0}^{n_j-1}(H_\ell)_{11}
=
\frac{1}{2\,\Impart(z_0)\,R_{n_j}(z_0)}
\le
\frac{1}{2\,\Impart(z_0)\,r_0}.
\]
By nestedness of Weyl disks,
\[
\mathcal D_{n_j}(z_0)\subseteq \mathcal D_{n_1}(z_0),
\]
so $|c_{n_j}(z_0)|\le M_0$. Using
\eqref{eq:R1-circle-expanded}, \eqref{eq:R1-disk-center},
and \eqref{eq:R1-radius-q11},
\[
-\tilde q_{22}^{(n_j)}(z_0)
\le
\bigl|\tilde q_{11}^{(n_j)}(z_0)\bigr|\,|c_{n_j}(z_0)|^2
\le
\frac{M_0^2}{2r_0}.
\]
Since $\tilde q_{22}^{(n_j)}(z_0)=-\Impart(z_0)\,(G_j)_{22}$,
\[
\sum_{\ell=0}^{n_j-1}(H_\ell)_{22}
=(G_j)_{22}
\le
\frac{M_0^2}{2\,\Impart(z_0)\,r_0}.
\]
Therefore
\[
\sum_{\ell=0}^{n_j-1}\operatorname{tr}(H_\ell)
\;=\;
\sum_{\ell=0}^{n_j-1}(H_\ell)_{11}
+
\sum_{\ell=0}^{n_j-1}(H_\ell)_{22}
\le
\frac{1+M_0^2}{2\,\Impart(z_0)\,r_0}.
\]
Taking $\sup_j$ gives \eqref{eq:R1-prefix-bound-collinear-visible}.
\end{proof}

\begin{corollary}[Collapse from collinear windows]\label{cor:R1_collapse_collinear_visible}
Assume
\[
\sum_{k=0}^{\infty}\operatorname{tr}(H_k)=\infty.
\]
Assume moreover that for every $z_0\in\UHP$, every $r_0>0$, and every
subsequence $(n_j)$ with $R_{n_j}(z_0)\ge r_0$, the hypotheses of
Proposition~\ref{prop:R1_prefix_mass_collinear_visible} hold.
Then for every $z\in\UHP$,
\[
R_n(z)\to0\qquad(n\to\infty).
\]
\end{corollary}
\begin{proof}
By Proposition~\ref{prop:R1_prefix_mass_collinear_visible}, every radius-floor
subsequence satisfies the prefix bound required in
Theorem~\ref{thm:R1_rank1_tailwindow_principle}. The theorem then yields
collapse on all of $\UHP$.
\end{proof}

\begin{lemma}[Collinearity from rank-one aggregate windows]\label{lem:R1_collinear_from_rank1_aggregate}
Let $H_0,\dots,H_{n-1}\in\R^{2\times2}$ be real-symmetric PSD and set
\[
G:=\sum_{\ell=0}^{n-1}H_\ell.
\]
If $\operatorname{rank}(G)\le1$, then there exist $v\in\R^2\setminus\{0\}$ and
nonnegative scalars $a_0,\dots,a_{n-1}$ such that
\[
H_\ell=a_\ell\,vv^\top\qquad(0\le\ell\le n-1).
\]
\end{lemma}
\begin{proof}
If $G=0$, then each $H_\ell=0$ (PSD sum zero), so the claim is trivial.
Assume $G\neq0$ and $\operatorname{rank}(G)=1$. Then
$K:=\ker G$ is one-dimensional. For $x\in K$,
\[
0=x^\top Gx=\sum_{\ell=0}^{n-1}x^\top H_\ell x,
\]
and every term is nonnegative since $H_\ell\succeq0$; hence
$x^\top H_\ell x=0$ for each $\ell$. For PSD matrices,
$x^\top H_\ell x=0$ implies $H_\ell x=0$, so $K\subseteq\ker H_\ell$.
Therefore
\[
\operatorname{Ran}(H_\ell)\subseteq K^\perp,
\]
and $K^\perp$ is one-dimensional. Choose $v\in K^\perp\setminus\{0\}$; then
each $H_\ell$ has range in $\operatorname{span}\{v\}$ and is symmetric PSD, so
$H_\ell=a_\ell vv^\top$ with $a_\ell\ge0$.
\end{proof}

\begin{corollary}[Prefix-mass bound from rank-one aggregate windows]\label{cor:R1_prefix_mass_from_rank1_aggregate}
Fix $z_0\in\UHP$, $r_0>0$, and a subsequence $(n_j)$ with
$R_{n_j}(z_0)\ge r_0$.
Assume
\[
\operatorname{rank}\!\Big(\sum_{\ell=0}^{n_j-1}H_\ell\Big)\le1
\qquad(j\ge1).
\]
Then
\[
\sup_{j\ge1}\sum_{\ell=0}^{n_j-1}\operatorname{tr}(H_\ell)
\le
\frac{1+M_0^2}{2\,\Impart(z_0)\,r_0},
\qquad
M_0:=|c_{n_1}(z_0)|+R_{n_1}(z_0).
\]
\end{corollary}
\begin{proof}
Fix $j$. Apply Lemma~\ref{lem:R1_collinear_from_rank1_aggregate} to
$H_0,\dots,H_{n_j-1}$: there exist
$v_j\in\R^2\setminus\{0\}$ and $a_{\ell,j}\ge0$ with
$H_\ell=a_{\ell,j}v_jv_j^\top$ for $0\le\ell<n_j$.
Thus the hypotheses of
Proposition~\ref{prop:R1_prefix_mass_collinear_visible} are satisfied on this
window, and the stated bound follows.
\end{proof}

\begin{corollary}[Collapse from rank-one aggregate windows]\label{cor:R1_collapse_rank1_aggregate}
Assume
\[
\sum_{k=0}^{\infty}\operatorname{tr}(H_k)=\infty.
\]
Assume moreover that for every $z_0\in\UHP$, every $r_0>0$, and every
subsequence $(n_j)$ with $R_{n_j}(z_0)\ge r_0$,
\[
\operatorname{rank}\!\Big(\sum_{\ell=0}^{n_j-1}H_\ell\Big)\le1
\qquad(j\ge1).
\]
Then for every $z\in\UHP$,
\[
R_n(z)\to0\qquad(n\to\infty).
\]
\end{corollary}
\begin{proof}
By Corollary~\ref{cor:R1_prefix_mass_from_rank1_aggregate}, every such
radius-floor subsequence satisfies the prefix-mass bound required in
Theorem~\ref{thm:R1_rank1_tailwindow_principle}. Hence that theorem gives
collapse on all of $\UHP$.
\end{proof}

\begin{lemma}[Rank-one aggregate criterion via determinant]\label{lem:R1_rank1_aggregate_det_equiv}
Let $G\in\R^{2\times2}$ be real-symmetric and PSD. Then
\[
\operatorname{rank}(G)\le1
\quad\Longleftrightarrow\quad
\det(G)=0.
\]
\end{lemma}
\begin{proof}
If $\operatorname{rank}(G)\le1$, then $\det(G)=0$.
Conversely, if $\det(G)=0$, let $\lambda_1,\lambda_2\ge0$ be the eigenvalues of
$G$ (PSD). Then $\lambda_1\lambda_2=\det(G)=0$, so at least one eigenvalue is
zero; hence $\operatorname{rank}(G)\le1$.
\end{proof}

\begin{lemma}[Determinant expansion for rank-one aggregate windows]\label{lem:R1_rank1_aggregate_det_expansion}
Let
\[
H_\ell=\tau_\ell u_\ell u_\ell^\top
\qquad(0\le \ell\le n-1),
\]
with $\tau_\ell\ge0$ and $u_\ell\in\R^2$, and set
\[
G:=\sum_{\ell=0}^{n-1}H_\ell.
\]
Then
\[
\det(G)
=
\sum_{0\le q<p\le n-1}
\tau_p\tau_q\,(u_p^\top J u_q)^2.
\]
\end{lemma}
\begin{proof}
Define
\[
U:=\big[\sqrt{\tau_0}u_0\ \cdots\ \sqrt{\tau_{n-1}}u_{n-1}\big]\in\R^{2\times n}.
\]
Then $G=UU^\top$. By the Cauchy--Binet formula for $2\times n$ matrices,
\[
\det(UU^\top)
=
\sum_{0\le q<p\le n-1}
\det\!\big([\sqrt{\tau_q}u_q,\sqrt{\tau_p}u_p]\big)^2
=
\sum_{0\le q<p\le n-1}
\tau_p\tau_q\,\det([u_q,u_p])^2.
\]
In $\R^2$, $\det([u_q,u_p])=u_p^\top J u_q$, so the stated identity follows.
\end{proof}

\begin{lemma}[Atomic determinant expansion for aggregate windows]\label{lem:R1_atomic_aggregate_det_expansion}
Let $x_1,\dots,x_m\in\R^2$ and define
\[
G:=\sum_{a=1}^m x_a x_a^\top.
\]
Then
\[
\det(G)
=
\sum_{1\le a<b\le m}(x_a^\top J x_b)^2.
\]
\end{lemma}
\begin{proof}
Set $X:=[x_1\ \cdots\ x_m]\in\R^{2\times m}$, so $G=XX^\top$.
By Cauchy--Binet,
\[
\det(XX^\top)
=
\sum_{1\le a<b\le m}\det([x_a,x_b])^2.
\]
In $\R^2$, $\det([x_a,x_b])=x_a^\top J x_b$ up to sign, and the square removes the sign.
\end{proof}

\begin{corollary}[Zero aggregate determinant forces vanishing pairwise symplectic couplings]\label{cor:R1_det_zero_forces_sympl_orth_pairs}
In the setup of Lemma~\ref{lem:R1_rank1_aggregate_det_expansion}, if
$\det(G)=0$, then
\[
\tau_p\tau_q\,(u_p^\top J u_q)^2=0
\qquad(0\le q<p\le n-1).
\]
In particular, whenever $\tau_p,\tau_q>0$, one has $u_p^\top J u_q=0$.
\end{corollary}
\begin{proof}
By Lemma~\ref{lem:R1_rank1_aggregate_det_expansion}, $\det(G)$ is a sum of
nonnegative terms. If the sum is $0$, each term must be $0$.
\end{proof}

\begin{corollary}[Collapse from zero-determinant aggregate windows]\label{cor:R1_collapse_det_zero_aggregate}
Assume
\[
\sum_{k=0}^{\infty}\operatorname{tr}(H_k)=\infty.
\]
Assume moreover that for every $z_0\in\UHP$, every $r_0>0$, and every
subsequence $(n_j)$ with $R_{n_j}(z_0)\ge r_0$,
\[
\det\!\Big(\sum_{\ell=0}^{n_j-1}H_\ell\Big)=0
\qquad(j\ge1).
\]
Then for every $z\in\UHP$,
\[
R_n(z)\to0\qquad(n\to\infty).
\]
\end{corollary}
\begin{proof}
For each $j$, set $G_j:=\sum_{\ell=0}^{n_j-1}H_\ell$. By hypothesis,
$\det(G_j)=0$. Since $G_j$ is PSD, Lemma~\ref{lem:R1_rank1_aggregate_det_equiv}
gives $\operatorname{rank}(G_j)\le1$. Apply
Corollary~\ref{cor:R1_collapse_rank1_aggregate}.
\end{proof}

\begin{lemma}[Symplectic orthogonality is equivalent to one-direction windows in \texorpdfstring{$\R^2$}{R2}]\label{lem:R1_sympl_orth_collinear_equiv}
Let $u_0,\dots,u_{n-1}\in\R^2\setminus\{0\}$. The following are equivalent:
\begin{enumerate}
\item $u_p^\top J u_q=0$ for all $0\le q<p\le n-1$.
\item There exist $v\in\R^2\setminus\{0\}$ and scalars $c_0,\dots,c_{n-1}\in\R$
such that $u_\ell=c_\ell v$ for all $\ell$.
\end{enumerate}
\end{lemma}
\begin{proof}
$(2)\Rightarrow(1)$ is immediate from skew-symmetry of $J$:
$u_p^\top J u_q=c_pc_q\,v^\top Jv=0$.

For $(1)\Rightarrow(2)$, fix $v:=u_0\neq0$. For each $\ell\ge1$,
$u_\ell^\top Jv=0$ by assumption. In $\R^2$, this means
$\det[u_\ell,v]=0$, so $u_\ell$ is collinear with $v$, i.e.
$u_\ell=c_\ell v$ for some $c_\ell\in\R$. Also $u_0=1\cdot v$.
\end{proof}

\begin{proposition}[Radius-floor closure from the target mass-divergence theorem]\label{prop:R1_radiusfloor_closure}
Fix $z_0\in\UHP$, $r_0>0$, and an infinite subsequence $(n_j)$ such that
\[
R_{n_j}(z_0)\ge r_0\qquad(j\ge1).
\]
Then
\[
\sup_{j\ge1}\sum_{\ell=0}^{n_j-1}\operatorname{tr}(H_\ell)<\infty.
\]
Equivalently (Lemma~\ref{lem:R1_tailwindow_prefix_equiv}), the tail-window
bound \eqref{eq:R1-tail-window} holds on this subsequence.
\end{proposition}
\begin{proof}
If $\sum_{k=0}^{\infty}\operatorname{tr}(H_k)=\infty$, then by
Proposition~\ref{prop:R1_classical_mass_applicability} and
Theorem~\ref{prop:R1_classical_mass} we have
$R_n(z_0)\to0$, contradicting $R_{n_j}(z_0)\ge r_0$.
Hence $\sum_{k=0}^{\infty}\operatorname{tr}(H_k)<\infty$, so
\[
\sum_{\ell=0}^{n_j-1}\operatorname{tr}(H_\ell)\le
\sum_{k=0}^{\infty}\operatorname{tr}(H_k)
\]
for all $j$, proving the prefix bound. The tail-window equivalence is
Lemma~\ref{lem:R1_tailwindow_prefix_equiv}.
\end{proof}

\begin{proposition}[CS2 on radius-floor subsequences (closed target)]\label{prop:R1_CS2_on_radiusfloor_target}
Fix $z_0\in\UHP$, $r_0>0$, and an infinite subsequence $(n_j)$ with
\[
R_{n_j}(z_0)\ge r_0\qquad(j\ge1).
\]
Then there exists $\kappa=\kappa(z_0,r_0)>0$ such that
\[
\left\|(Y_{k,j}Y_{k,j}^{\ast})^{-1}\right\|_{\mathrm{op}}\le \kappa
\qquad(0\le k<n_j,\ j\ge1),
\]
where $Y_{k,j}:=U_k(z_0)U_{n_j}(z_0)^{-1}$.
\end{proposition}
\begin{proof}
By Proposition~\ref{prop:R1_radiusfloor_closure}, the radius-floor hypothesis
implies
\[
\sum_{k=0}^{\infty}\operatorname{tr}(H_k)<\infty.
\]
Apply Proposition~\ref{prop:R1_CS2_from_finite_total_mass}.
\end{proof}

\begin{proposition}[CS2 from finite total mass]\label{prop:R1_CS2_from_finite_total_mass}
Fix $z_0\in\UHP$ and an infinite subsequence $(n_j)$.
Assume
\[
\sum_{k=0}^{\infty}\operatorname{tr}(H_k)<\infty.
\]
Then there exists $K=K(z_0)>0$ such that
\[
\left\|
\big(M_{n_j-1}(z_0)\cdots M_k(z_0)\big)^{-1}
\right\|_{\mathrm{op}}
\le K
\qquad(0\le k<n_j,\ j\ge1).
\]
Equivalently, the CS2 bound \eqref{eq:R1-CS2} holds on this subsequence.
\end{proposition}
\begin{proof}
Set
\[
A_k:=\frac{z_0}{2}JH_k,
\qquad
a_k:=\|A_k\|_{\mathrm{op}}.
\]
From $M_k(z_0)=R_k(z_0)^{-1}L_k(z_0)$ with
$L_k(z_0)=I-A_k$, $R_k(z_0)=I+A_k$, we have
\[
M_k(z_0)^{-1}=(I-A_k)^{-1}(I+A_k).
\]
Also
\[
a_k\le \frac{|z_0|}{2}\,\|H_k\|_{\mathrm{op}}
\le \frac{|z_0|}{2}\,\operatorname{tr}(H_k),
\]
so $\sum_{k\ge0}a_k<\infty$.
Choose $N$ such that $a_k\le \frac12$ for all $k\ge N$.
Then for $k\ge N$,
\[
\|M_k(z_0)^{-1}\|_{\mathrm{op}}
\le
\|(I-A_k)^{-1}\|_{\mathrm{op}}\|I+A_k\|_{\mathrm{op}}
\le
\frac{1+a_k}{1-a_k}.
\]
For $0\le x\le \frac12$, $\log\frac{1+x}{1-x}\le 4x$, hence
\[
\|M_k(z_0)^{-1}\|_{\mathrm{op}}\le \exp(4a_k)
\qquad(k\ge N).
\]
Define finite constants
\[
C_{\mathrm{head}}:=\prod_{\ell=0}^{N-1}\|M_\ell(z_0)^{-1}\|_{\mathrm{op}},
\qquad
C_{\mathrm{tail}}:=\exp\!\left(4\sum_{\ell=N}^{\infty}a_\ell\right).
\]
Fix $j$ and $0\le k<n_j$. If $k\ge N$, then
\[
\left\|
\big(M_{n_j-1}(z_0)\cdots M_k(z_0)\big)^{-1}
\right\|_{\mathrm{op}}
\le
\prod_{\ell=k}^{n_j-1}\|M_\ell(z_0)^{-1}\|_{\mathrm{op}}
\le C_{\mathrm{tail}}.
\]
If $k<N$, split at $N$:
\[
\left\|
\big(M_{n_j-1}(z_0)\cdots M_k(z_0)\big)^{-1}
\right\|_{\mathrm{op}}
\le
\prod_{\ell=k}^{N-1}\|M_\ell(z_0)^{-1}\|_{\mathrm{op}}
\cdot
\prod_{\ell=N}^{n_j-1}\|M_\ell(z_0)^{-1}\|_{\mathrm{op}}
\le
C_{\mathrm{head}}C_{\mathrm{tail}}.
\]
Hence \eqref{eq:R1-CS2-tail} holds with
$K:=C_{\mathrm{head}}C_{\mathrm{tail}}$.
By Lemma~\ref{lem:R1_CS2_equiv}, this is equivalent to \eqref{eq:R1-CS2}.
\end{proof}

\begin{proposition}[CS2 from collinear windows]\label{prop:R1_CS2_collinear_windows}
Fix $z_0\in\UHP$, $r_0>0$, and an infinite subsequence $(n_j)$ with
\[
R_{n_j}(z_0)\ge r_0\qquad(j\ge1).
\]
Assume that for each $j\ge1$ there exist $v_j\in\R^2\setminus\{0\}$ and
nonnegative scalars $(a_{\ell,j})_{0\le\ell<n_j}$ such that
\[
H_\ell=a_{\ell,j}\,v_jv_j^\top\qquad(0\le\ell<n_j).
\]
Then there exists $\kappa=\kappa(z_0,r_0)>0$ such that
\[
\left\|(Y_{k,j}Y_{k,j}^{\ast})^{-1}\right\|_{\mathrm{op}}\le \kappa
\qquad(0\le k<n_j,\ j\ge1).
\]
\end{proposition}
\begin{proof}
By Proposition~\ref{prop:R1_prefix_mass_collinear_visible},
\[
\sup_{j\ge1}\sum_{\ell=0}^{n_j-1}\operatorname{tr}(H_\ell)\le B<\infty
\]
for some $B=B(z_0,r_0,(n_j))$. Hence, by
Lemma~\ref{lem:R1_tailwindow_prefix_equiv},
\begin{equation}\label{eq:R1-tail-window-collinear}
\sup_{j\ge1}\sup_{0\le k<n_j}\sum_{\ell=k}^{n_j-1}\operatorname{tr}(H_\ell)\le B.
\end{equation}
Now fix $j$ and set $u_j:=v_j/\|v_j\|$. Writing
\[
H_\ell=\tau_{\ell,j}u_ju_j^\top,\qquad
\tau_{\ell,j}:=a_{\ell,j}\|v_j\|^2\ge0
\quad(0\le\ell<n_j),
\]
shows the one-line condition \eqref{eq:R1-window-lagrangian-line} on each window.
Therefore Corollary~\ref{cor:R1_CS2_structured_lagrangian_line} applies
using \eqref{eq:R1-tail-window-collinear}, and yields \eqref{eq:R1-CS2}
on this subsequence. This is exactly the claimed bound on
$\|(Y_{k,j}Y_{k,j}^{\ast})^{-1}\|_{\mathrm{op}}$.
\end{proof}

\begin{corollary}[CS2 from rank-one aggregate windows]\label{cor:R1_CS2_rank1_aggregate}
Fix $z_0\in\UHP$, $r_0>0$, and an infinite subsequence $(n_j)$ with
\[
R_{n_j}(z_0)\ge r_0\qquad(j\ge1).
\]
Assume
\[
\operatorname{rank}\!\Big(\sum_{\ell=0}^{n_j-1}H_\ell\Big)\le1
\qquad(j\ge1).
\]
Then the conclusion of Proposition~\ref{prop:R1_CS2_on_radiusfloor_target}
holds on this subsequence.
\end{corollary}
\begin{proof}
Fix $j$. Lemma~\ref{lem:R1_collinear_from_rank1_aggregate} applied to
$H_0,\dots,H_{n_j-1}$ yields
\[
H_\ell=a_{\ell,j}v_jv_j^\top\qquad(0\le\ell<n_j)
\]
for some $v_j\neq0$ and $a_{\ell,j}\ge0$. Thus the hypotheses of
Proposition~\ref{prop:R1_CS2_collinear_windows} hold, and its conclusion
follows.
\end{proof}

\begin{corollary}[CS2 from zero-determinant aggregate windows]\label{cor:R1_CS2_det_zero_aggregate}
Fix $z_0\in\UHP$, $r_0>0$, and an infinite subsequence $(n_j)$ with
\[
R_{n_j}(z_0)\ge r_0\qquad(j\ge1).
\]
Assume
\[
\det\!\Big(\sum_{\ell=0}^{n_j-1}H_\ell\Big)=0
\qquad(j\ge1).
\]
Then the conclusion of Proposition~\ref{prop:R1_CS2_on_radiusfloor_target}
holds on this subsequence.
\end{corollary}
\begin{proof}
For each $j$, set $G_j:=\sum_{\ell=0}^{n_j-1}H_\ell$. Since each $H_\ell\succeq0$,
also $G_j\succeq0$. The hypothesis $\det(G_j)=0$ and
Lemma~\ref{lem:R1_rank1_aggregate_det_equiv} imply $\operatorname{rank}(G_j)\le1$.
Now apply Corollary~\ref{cor:R1_CS2_rank1_aggregate}.
\end{proof}

\begin{corollary}[CS2 from vanishing symplectic-area sum on windows]\label{cor:R1_CS2_zero_sympl_area_sum}
Fix $z_0\in\UHP$, $r_0>0$, and an infinite subsequence $(n_j)$ with
\[
R_{n_j}(z_0)\ge r_0\qquad(j\ge1).
\]
Assume that for each $j\ge1$ there exist $\tau_{\ell,j}\ge0$ and
$u_{\ell,j}\in\R^2$ ($0\le\ell<n_j$) such that
\[
H_\ell=\tau_{\ell,j}u_{\ell,j}u_{\ell,j}^\top
\qquad(0\le\ell<n_j),
\]
and
\[
\sum_{0\le q<p<n_j}
\tau_{p,j}\tau_{q,j}\,
\bigl(u_{p,j}^\top J u_{q,j}\bigr)^2
=0
\qquad(j\ge1).
\]
Then the conclusion of Proposition~\ref{prop:R1_CS2_on_radiusfloor_target}
holds on this subsequence.
\end{corollary}
\begin{proof}
For each $j$, set $G_j:=\sum_{\ell=0}^{n_j-1}H_\ell$. By
Lemma~\ref{lem:R1_rank1_aggregate_det_expansion} applied on the $j$-window,
the displayed zero-sum condition is equivalent to $\det(G_j)=0$.
Hence Corollary~\ref{cor:R1_CS2_det_zero_aggregate} applies.
\end{proof}

\begin{corollary}[CS2 from atomic symplectic orthogonality on windows]\label{cor:R1_CS2_atomic_sympl_orth_windows}
Fix $z_0\in\UHP$, $r_0>0$, and an infinite subsequence $(n_j)$ with
\[
R_{n_j}(z_0)\ge r_0\qquad(j\ge1).
\]
Assume that for each $j\ge1$ and each $0\le\ell<n_j$, there exist vectors
$x_{\ell,1,j},\dots,x_{\ell,m_{\ell,j},j}\in\R^2$ such that
\[
H_\ell=\sum_{r=1}^{m_{\ell,j}} x_{\ell,r,j}x_{\ell,r,j}^\top.
\]
Assume moreover that on each fixed $j$-window all atom pairs are symplectically
orthogonal:
\[
x_{\ell,r,j}^\top J x_{p,s,j}=0
\quad\text{for all distinct }(\ell,r)\neq(p,s).
\]
Then the conclusion of Proposition~\ref{prop:R1_CS2_on_radiusfloor_target}
holds on this subsequence.
\end{corollary}
\begin{proof}
For each $j$, set
\[
G_j:=\sum_{\ell=0}^{n_j-1}H_\ell
=
\sum_{\ell=0}^{n_j-1}\sum_{r=1}^{m_{\ell,j}} x_{\ell,r,j}x_{\ell,r,j}^\top.
\]
Applying Lemma~\ref{lem:R1_atomic_aggregate_det_expansion} to this finite atom
family and using pairwise symplectic orthogonality gives $\det(G_j)=0$.
Hence Corollary~\ref{cor:R1_CS2_det_zero_aggregate} applies.
\end{proof}

\begin{proposition}[CS2 from uniform SPD floor on windows]\label{prop:R1_CS2_spd_floor}
Fix $z_0\in\UHP$, $r_0>0$, and an infinite subsequence $(n_j)$ with
\[
R_{n_j}(z_0)\ge r_0\qquad(j\ge1).
\]
Assume there exists $\mu_0>0$ such that for all
$j\ge1$ and $0\le k<n_j$,
\[
H_k\succeq \mu_0 I_2.
\]
Then the conclusion of Proposition~\ref{prop:R1_CS2_on_radiusfloor_target}
holds on this subsequence.
\end{proposition}
\begin{proof}
By Lemma~\ref{lem:R1_radiusfloor_kappa_bottleneck}, there exist constants
$C_1,C_2<\infty$ such that for every $j$,
\[
\sum_{\ell=0}^{n_j-1}\langle y_{\ell,j},H_\ell y_{\ell,j}\rangle\le C_1,\qquad
\sum_{\ell=0}^{n_j-1}\langle \widetilde y_{\ell,j},H_\ell \widetilde y_{\ell,j}\rangle\le C_2.
\]
Fix $(k,j)$ with $0\le k<n_j$ and write $Y:=Y_{k,j}=[y_{k,j}\ \widetilde y_{k,j}]$.
Then
\[
\langle y_{k,j},H_k y_{k,j}\rangle
+
\langle \widetilde y_{k,j},H_k \widetilde y_{k,j}\rangle
=
\operatorname{tr}(Y^\ast H_k Y)
\ge
\mu_0\,\operatorname{tr}(Y^\ast Y).
\]
Since all terms are nonnegative and each channel sum is bounded,
\[
\langle y_{k,j},H_k y_{k,j}\rangle
+
\langle \widetilde y_{k,j},H_k \widetilde y_{k,j}\rangle
\le C_1+C_2,
\]
hence
\[
\operatorname{tr}(Y^\ast Y)\le \frac{C_1+C_2}{\mu_0}.
\]
Let $s_1(Y)\ge s_2(Y)>0$ be the singular values. By Lemma~\ref{lem:R1-det1},
$\det Y=1$, and by Lemma~\ref{lem:R1_CS2_equiv},
\[
\kappa_{k,j}
:=
\|(Y_{k,j}Y_{k,j}^\ast)^{-1}\|_{\mathrm{op}}
=
s_1(Y)^2
\le
s_1(Y)^2+s_2(Y)^2
=
\operatorname{tr}(Y^\ast Y)
\le
\frac{C_1+C_2}{\mu_0}.
\]
This is exactly \eqref{eq:R1-CS2} on the subsequence.
\end{proof}

\begin{lemma}[Pointwise balance between transported-frame coercivity and frame growth]\label{lem:R1_frame_ratio_kappa_balance}
Fix $z_0\in\UHP$, $r_0>0$, and an infinite subsequence $(n_j)$ with
\[
R_{n_j}(z_0)\ge r_0\qquad(j\ge1).
\]
For $0\le k<n_j$, write
\[
Y_{k,j}:=U_k(z_0)U_{n_j}(z_0)^{-1},\qquad
\kappa_{k,j}:=\|(Y_{k,j}Y_{k,j}^\ast)^{-1}\|_{\mathrm{op}}.
\]
Let $C_1,C_2$ be the channel constants from
Lemma~\ref{lem:R1_radiusfloor_kappa_bottleneck}. Then for every $j\ge1$ and
$0\le k<n_j$,
\[
\frac{\operatorname{tr}(Y_{k,j}^\ast H_k Y_{k,j})}
{\operatorname{tr}(Y_{k,j}^\ast Y_{k,j})}
\cdot
\kappa_{k,j}
\le
C_1+C_2.
\]
\end{lemma}
\begin{proof}
Fix $(k,j)$ and write $Y:=Y_{k,j}=[y_{k,j}\ \widetilde y_{k,j}]$. Then
\[
\operatorname{tr}(Y^\ast H_k Y)
=
\langle y_{k,j},H_k y_{k,j}\rangle
+
\langle \widetilde y_{k,j},H_k \widetilde y_{k,j}\rangle.
\]
Each term is nonnegative and bounded by its full channel sum, so
\[
\operatorname{tr}(Y^\ast H_k Y)\le C_1+C_2.
\]
Let $s_1(Y)\ge s_2(Y)>0$ be singular values. By Lemma~\ref{lem:R1-det1},
$\det Y=1$, and by Lemma~\ref{lem:R1_CS2_equiv},
\[
\kappa_{k,j}=s_1(Y)^2\le s_1(Y)^2+s_2(Y)^2=\operatorname{tr}(Y^\ast Y).
\]
Hence
\[
\frac{\kappa_{k,j}}{\operatorname{tr}(Y^\ast Y)}\le1,
\]
and therefore
\[
\frac{\operatorname{tr}(Y^\ast H_k Y)}{\operatorname{tr}(Y^\ast Y)}
\cdot
\kappa_{k,j}
\le
\operatorname{tr}(Y^\ast H_k Y)
\le
C_1+C_2.
\]
\end{proof}

\begin{proposition}[CS2 from transported-frame coercivity on windows]\label{prop:R1_CS2_frame_coercivity_windows}
Fix $z_0\in\UHP$, $r_0>0$, and an infinite subsequence $(n_j)$ with
\[
R_{n_j}(z_0)\ge r_0\qquad(j\ge1).
\]
Assume there exists $\mu_0>0$ such that for all $j\ge1$ and $0\le k<n_j$,
with
\[
Y_{k,j}:=U_k(z_0)U_{n_j}(z_0)^{-1},
\]
one has
\[
\operatorname{tr}(Y_{k,j}^\ast H_k Y_{k,j})
\ge
\mu_0\,\operatorname{tr}(Y_{k,j}^\ast Y_{k,j}).
\]
Then the conclusion of Proposition~\ref{prop:R1_CS2_on_radiusfloor_target}
holds on this subsequence.
\end{proposition}
\begin{proof}
By Lemma~\ref{lem:R1_frame_ratio_kappa_balance},
\[
\frac{\operatorname{tr}(Y_{k,j}^\ast H_k Y_{k,j})}
{\operatorname{tr}(Y_{k,j}^\ast Y_{k,j})}
\cdot
\kappa_{k,j}
\le
C_1+C_2.
\]
Using the hypothesis
\[
\frac{\operatorname{tr}(Y_{k,j}^\ast H_k Y_{k,j})}
{\operatorname{tr}(Y_{k,j}^\ast Y_{k,j})}
\ge\mu_0,
\]
we obtain
\[
\kappa_{k,j}\le \frac{C_1+C_2}{\mu_0}
\qquad(0\le k<n_j,\ j\ge1),
\]
which is exactly \eqref{eq:R1-CS2} on this subsequence.
\end{proof}

\begin{corollary}[Failure of CS2 on a radius-floor subsequence forces vanishing transported-frame coercivity]\label{cor:R1_CS2_failure_forces_vanishing_frame_coercivity}
Fix $z_0\in\UHP$, $r_0>0$, and an infinite subsequence $(n_j)$ with
\[
R_{n_j}(z_0)\ge r_0\qquad(j\ge1).
\]
Assume that the conclusion of
Proposition~\ref{prop:R1_CS2_on_radiusfloor_target} fails on this
subsequence. Then
\[
\inf_{j\ge1}\ \inf_{0\le k<n_j}
\frac{\operatorname{tr}(Y_{k,j}^\ast H_k Y_{k,j})}
{\operatorname{tr}(Y_{k,j}^\ast Y_{k,j})}
=0,
\qquad
Y_{k,j}:=U_k(z_0)U_{n_j}(z_0)^{-1}.
\]
\end{corollary}
\begin{proof}
If the displayed infimum were $>0$, there would exist $\mu_0>0$ such that
\[
\operatorname{tr}(Y_{k,j}^\ast H_k Y_{k,j})
\ge
\mu_0\,\operatorname{tr}(Y_{k,j}^\ast Y_{k,j})
\qquad(j\ge1,\ 0\le k<n_j).
\]
By Proposition~\ref{prop:R1_CS2_frame_coercivity_windows}, this would imply the
conclusion of Proposition~\ref{prop:R1_CS2_on_radiusfloor_target}, contradiction.
\end{proof}

\begin{corollary}[CS2 from transported-frame coercivity at maximal-frame indices]\label{cor:R1_CS2_from_maxindex_frame_coercivity}
Fix $z_0\in\UHP$, $r_0>0$, and an infinite subsequence $(n_j)$ with
\[
R_{n_j}(z_0)\ge r_0\qquad(j\ge1).
\]
For each $j$, define
\[
K_j:=\max_{0\le k<n_j}\kappa_{k,j},
\]
and choose $k_j^\star\in\{0,\dots,n_j-1\}$ with
\[
\kappa_{k_j^\star,j}=K_j.
\]
Assume there exists $\mu_0>0$ such that for all $j\ge1$,
\[
\frac{\operatorname{tr}(Y_{k_j^\star,j}^\ast H_{k_j^\star} Y_{k_j^\star,j})}
{\operatorname{tr}(Y_{k_j^\star,j}^\ast Y_{k_j^\star,j})}
\ge\mu_0.
\]
Then the conclusion of Proposition~\ref{prop:R1_CS2_on_radiusfloor_target}
holds on this subsequence.
\end{corollary}
\begin{proof}
By Lemma~\ref{lem:R1_frame_ratio_kappa_balance},
\[
\frac{\operatorname{tr}(Y_{k_j^\star,j}^\ast H_{k_j^\star} Y_{k_j^\star,j})}
{\operatorname{tr}(Y_{k_j^\star,j}^\ast Y_{k_j^\star,j})}
\cdot
\kappa_{k_j^\star,j}
\le
C_1+C_2.
\]
Using the hypothesis and $\kappa_{k_j^\star,j}=K_j$ gives
\[
K_j\le \frac{C_1+C_2}{\mu_0}\qquad(j\ge1).
\]
Hence $\sup_j K_j<\infty$, and Corollary~\ref{cor:R1_CS2_equiv_Kj}
implies \eqref{eq:R1-CS2} on this subsequence.
\end{proof}

\begin{corollary}[Failure of CS2 forces vanishing maximal-index transported-frame coercivity]\label{cor:R1_CS2_failure_forces_vanishing_maxindex_frame_coercivity}
Fix $z_0\in\UHP$, $r_0>0$, and an infinite subsequence $(n_j)$ with
\[
R_{n_j}(z_0)\ge r_0\qquad(j\ge1).
\]
For each $j$, choose $k_j^\star\in\{0,\dots,n_j-1\}$ with
\[
\kappa_{k_j^\star,j}=K_j:=\max_{0\le k<n_j}\kappa_{k,j}.
\]
If the conclusion of Proposition~\ref{prop:R1_CS2_on_radiusfloor_target}
fails on this subsequence, then
\[
\inf_{j\ge1}
\frac{\operatorname{tr}(Y_{k_j^\star,j}^\ast H_{k_j^\star} Y_{k_j^\star,j})}
{\operatorname{tr}(Y_{k_j^\star,j}^\ast Y_{k_j^\star,j})}
=0.
\]
\end{corollary}
\begin{proof}
If the displayed infimum were $>0$, there would exist $\mu_0>0$ such that
\[
\frac{\operatorname{tr}(Y_{k_j^\star,j}^\ast H_{k_j^\star} Y_{k_j^\star,j})}
{\operatorname{tr}(Y_{k_j^\star,j}^\ast Y_{k_j^\star,j})}
\ge\mu_0
\qquad(j\ge1).
\]
By Corollary~\ref{cor:R1_CS2_from_maxindex_frame_coercivity}, this would imply
the conclusion of Proposition~\ref{prop:R1_CS2_on_radiusfloor_target},
contradiction.
\end{proof}

\begin{corollary}[Failure of CS2 on a radius-floor subsequence forces vanishing local coercivity]\label{cor:R1_CS2_failure_forces_vanishing_coercivity}
Fix $z_0\in\UHP$, $r_0>0$, and an infinite subsequence $(n_j)$ with
\[
R_{n_j}(z_0)\ge r_0\qquad(j\ge1).
\]
Assume that the conclusion of
Proposition~\ref{prop:R1_CS2_on_radiusfloor_target} fails on this
subsequence.
Then
\[
\inf_{j\ge1}\ \inf_{0\le k<n_j}\ \lambda_{\min}(H_k)=0.
\]
Equivalently, for every $\mu>0$ there exist $j\ge1$ and $0\le k<n_j$ such that
$H_k\nsucceq \mu I_2$.
\end{corollary}
\begin{proof}
If the displayed infimum were $>0$, there would exist $\mu_0>0$ with
\[
H_k\succeq \mu_0 I_2
\qquad(j\ge1,\ 0\le k<n_j).
\]
By Proposition~\ref{prop:R1_CS2_spd_floor}, this implies the conclusion of
Proposition~\ref{prop:R1_CS2_on_radiusfloor_target}, contradiction.
\end{proof}

\begin{corollary}[CS2 from uniform ellipticity with trace floor on windows]\label{prop:R1_CS2_uniform_elliptic_tracefloor}
Fix $z_0\in\UHP$, $r_0>0$, and an infinite subsequence $(n_j)$ with
\[
R_{n_j}(z_0)\ge r_0\qquad(j\ge1).
\]
Assume there exist constants $\eta,\tau_0>0$ such that for all
$j\ge1$ and $0\le k<n_j$,
\[
H_k\succeq \eta\,\operatorname{tr}(H_k)\,I_2,
\qquad
\operatorname{tr}(H_k)\ge \tau_0.
\]
Then the conclusion of Proposition~\ref{prop:R1_CS2_on_radiusfloor_target}
holds on this subsequence.
\end{corollary}
\begin{proof}
Set $\mu_0:=\eta\tau_0$. Then for all $j\ge1$ and $0\le k<n_j$,
\[
H_k\succeq \eta\,\operatorname{tr}(H_k)\,I_2
\succeq \eta\tau_0 I_2
=\mu_0 I_2.
\]
Apply Proposition~\ref{prop:R1_CS2_spd_floor}.
\end{proof}

\begin{corollary}[CS2 from determinant floor and trace cap on windows]\label{cor:R1_CS2_detfloor_tracecap}
Fix $z_0\in\UHP$, $r_0>0$, and an infinite subsequence $(n_j)$ with
\[
R_{n_j}(z_0)\ge r_0\qquad(j\ge1).
\]
Assume there exist constants $\delta_0,T_0>0$ such that for all
$j\ge1$ and $0\le k<n_j$,
\[
\det(H_k)\ge \delta_0,
\qquad
\operatorname{tr}(H_k)\le T_0.
\]
Then the conclusion of Proposition~\ref{prop:R1_CS2_on_radiusfloor_target}
holds on this subsequence.
\end{corollary}
\begin{proof}
Fix $(k,j)$ and let $\lambda_{1,k}\ge\lambda_{2,k}\ge0$ be the eigenvalues of
$H_k$.
Since $H_k\succeq0$, we have
\[
\lambda_{2,k}
=\frac{\det(H_k)}{\lambda_{1,k}}
\ge
\frac{\det(H_k)}{\lambda_{1,k}+\lambda_{2,k}}
=
\frac{\det(H_k)}{\operatorname{tr}(H_k)}
\ge
\frac{\delta_0}{T_0}.
\]
Hence $H_k\succeq (\delta_0/T_0)I_2$ for all $j\ge1$, $0\le k<n_j$.
Apply Proposition~\ref{prop:R1_CS2_spd_floor} with
$\mu_0:=\delta_0/T_0$.
\end{proof}

\begin{corollary}[CS2 from inverse-trace cap on windows]\label{cor:R1_CS2_invtracecap}
Fix $z_0\in\UHP$, $r_0>0$, and an infinite subsequence $(n_j)$ with
\[
R_{n_j}(z_0)\ge r_0\qquad(j\ge1).
\]
Assume there exists $B_0>0$ such that for all $j\ge1$ and $0\le k<n_j$:
\[
H_k\succ0,
\qquad
\operatorname{tr}(H_k^{-1})\le B_0.
\]
Then the conclusion of Proposition~\ref{prop:R1_CS2_on_radiusfloor_target}
holds on this subsequence.
\end{corollary}
\begin{proof}
Fix $(k,j)$ and let $\lambda_{1,k}\ge\lambda_{2,k}>0$ be the eigenvalues of
$H_k$. Then
\[
\operatorname{tr}(H_k^{-1})
=
\lambda_{1,k}^{-1}+\lambda_{2,k}^{-1}
\ge
\lambda_{2,k}^{-1},
\]
hence $\lambda_{2,k}\ge 1/B_0$. Therefore
\[
H_k\succeq (1/B_0)I_2
\qquad(j\ge1,\ 0\le k<n_j).
\]
Apply Proposition~\ref{prop:R1_CS2_spd_floor} with $\mu_0:=1/B_0$.
\end{proof}

\begin{corollary}[Collapse from atomic symplectic orthogonality windows]\label{cor:R1_collapse_atomic_sympl_orth}
Assume
\[
\sum_{k=0}^{\infty}\operatorname{tr}(H_k)=\infty.
\]
Assume moreover that for every $z_0\in\UHP$, every $r_0>0$, and every
subsequence $(n_j)$ with $R_{n_j}(z_0)\ge r_0$, the hypotheses of
Corollary~\ref{cor:R1_CS2_atomic_sympl_orth_windows} hold.
Then for every $z\in\UHP$,
\[
R_n(z)\to0\qquad(n\to\infty).
\]
\end{corollary}
\begin{proof}
Fix $z\in\UHP$. If $R_n(z)\not\to0$, choose $r_0>0$ and a subsequence $(n_j)$
with $R_{n_j}(z)\ge r_0$.
By Corollary~\ref{cor:R1_CS2_atomic_sympl_orth_windows}, this subsequence
satisfies \eqref{eq:R1-CS2}. Then
Lemma~\ref{lem:R1_limit_circle_implies_finite_mass} yields
\[
\sum_{k=0}^{\infty}\operatorname{tr}(H_k)<\infty,
\]
contradicting the mass-divergence hypothesis.
\end{proof}

\begin{corollary}[Direct collapse from global uniform SPD floor]\label{cor:R1_collapse_spd_floor}
Assume there exists $\mu_0>0$ such that for all $k\ge0$,
\[
H_k\succeq \mu_0 I_2.
\]
Then for every $z\in\UHP$,
\[
R_n(z)\to0\qquad(n\to\infty).
\]
\end{corollary}
\begin{proof}
Fix $z\in\UHP$. If $R_n(z)\not\to0$, choose $r_0>0$ and a subsequence
$(n_j)$ with $R_{n_j}(z)\ge r_0$.
Proposition~\ref{prop:R1_CS2_spd_floor} gives \eqref{eq:R1-CS2} on this
subsequence, so Lemma~\ref{lem:R1_limit_circle_implies_finite_mass} yields
\[
\sum_{k=0}^{\infty}\operatorname{tr}(H_k)<\infty.
\]
But $\operatorname{tr}(H_k)\ge 2\mu_0$ for all $k$, impossible.
\end{proof}

\begin{corollary}[Direct collapse from global uniform ellipticity with trace floor]\label{cor:R1_collapse_uniform_elliptic_tracefloor}
Assume there exist constants $\eta,\tau_0>0$ such that for all $k\ge0$,
\[
H_k\succeq \eta\,\operatorname{tr}(H_k)\,I_2,
\qquad
\operatorname{tr}(H_k)\ge \tau_0.
\]
Then for every $z\in\UHP$,
\[
R_n(z)\to0\qquad(n\to\infty).
\]
\end{corollary}
\begin{proof}
Set $\mu_0:=\eta\tau_0$. Then for all $k\ge0$,
\[
H_k\succeq \eta\,\operatorname{tr}(H_k)\,I_2
\succeq \eta\tau_0 I_2
=\mu_0 I_2.
\]
Apply Corollary~\ref{cor:R1_collapse_spd_floor}.
\end{proof}

\begin{corollary}[Direct collapse from global inverse-trace cap]\label{cor:R1_collapse_invtracecap}
Assume there exists $B_0>0$ such that for all $k\ge0$:
\[
H_k\succ0,
\qquad
\operatorname{tr}(H_k^{-1})\le B_0.
\]
Then for every $z\in\UHP$,
\[
R_n(z)\to0\qquad(n\to\infty).
\]
\end{corollary}
\begin{proof}
Let $\lambda_{1,k}\ge\lambda_{2,k}>0$ be the eigenvalues of $H_k$.
As above,
\[
\operatorname{tr}(H_k^{-1})\ge \lambda_{2,k}^{-1},
\]
so $\lambda_{2,k}\ge 1/B_0$, i.e.
\[
H_k\succeq (1/B_0)I_2\qquad(k\ge0).
\]
Apply Corollary~\ref{cor:R1_collapse_spd_floor}.
\end{proof}

\begin{corollary}[Direct collapse from global determinant floor and trace cap]\label{cor:R1_collapse_detfloor_tracecap}
Assume there exist constants $\delta_0,T_0>0$ such that for all $k\ge0$,
\[
\det(H_k)\ge \delta_0,
\qquad
\operatorname{tr}(H_k)\le T_0.
\]
Then for every $z\in\UHP$,
\[
R_n(z)\to0\qquad(n\to\infty).
\]
\end{corollary}
\begin{proof}
Let $\lambda_{1,k}\ge\lambda_{2,k}\ge0$ be the eigenvalues of $H_k$.
Then
\[
\lambda_{2,k}
=\frac{\det(H_k)}{\lambda_{1,k}}
\ge
\frac{\det(H_k)}{\lambda_{1,k}+\lambda_{2,k}}
=
\frac{\det(H_k)}{\operatorname{tr}(H_k)}
\ge
\frac{\delta_0}{T_0},
\]
so $H_k\succeq (\delta_0/T_0)I_2$ for all $k\ge0$.
Apply Corollary~\ref{cor:R1_collapse_spd_floor}.
\end{proof}

\begin{remark}[Reduction map for v3.1 self-contained closure]\label{rem:v31_reduction_map}
The CS2 target proposition is now closed internally:
on every radius-floor subsequence,
Proposition~\ref{prop:R1_radiusfloor_closure} gives finite total mass, and
Proposition~\ref{prop:R1_CS2_from_finite_total_mass} yields
Proposition~\ref{prop:R1_CS2_on_radiusfloor_target}.
Separately, Theorem~\ref{thm:R1_mass_divergence_internal} then yields
Theorem~\ref{prop:R1_classical_mass} by the same contradiction scheme.
Hence there is no unresolved CS2 bottleneck left in v3.1.
\smallskip

\noindent\textbf{Subclass refinements (still useful).}
Even though the target is closed unconditionally, the structured regimes
covered by Proposition~\ref{prop:R1_CS2_collinear_windows},
Corollary~\ref{cor:R1_CS2_rank1_aggregate}, and
Corollary~\ref{cor:R1_CS2_det_zero_aggregate}
(equivalently Corollary~\ref{cor:R1_CS2_zero_sympl_area_sum},
or more generally Corollary~\ref{cor:R1_CS2_atomic_sympl_orth_windows};
also by Proposition~\ref{prop:R1_CS2_spd_floor}, hence by
Proposition~\ref{prop:R1_CS2_frame_coercivity_windows}, and by
Corollary~\ref{prop:R1_CS2_uniform_elliptic_tracefloor}, and by
Corollary~\ref{cor:R1_CS2_detfloor_tracecap}, and by
Corollary~\ref{cor:R1_CS2_invtracecap}).
\smallskip

\noindent\textbf{Legacy obstruction diagnostics.}
By Corollary~\ref{cor:R1_CS2_failure_forces_vanishing_frame_coercivity}, any
genuine failure of the target proposition on a radius-floor subsequence must
lie in a transported-frame degenerate regime where
\[
\inf_{j}\inf_{0\le k<n_j}
\frac{\operatorname{tr}(Y_{k,j}^\ast H_k Y_{k,j})}
{\operatorname{tr}(Y_{k,j}^\ast Y_{k,j})}
=0.
\]
In particular, by Corollary~\ref{cor:R1_CS2_failure_forces_vanishing_coercivity},
it lies in a non-uniformly coercive local regime where
\[
\inf_{j}\inf_{0\le k<n_j}\lambda_{\min}(H_k)=0.
\]
\end{remark}

\begin{remark}[Final bridge objective (resolved in v3.1)]\label{rem:v31_final_bridge_objective}
The previously isolated bridge objective was to prove the transported-frame
coercivity lower bound on every
radius-floor subsequence:
\[
\exists\,\mu_0=\mu_0(z_0,r_0)>0\ \text{such that}\ 
\operatorname{tr}(Y_{k,j}^\ast H_k Y_{k,j})
\ge
\mu_0\,\operatorname{tr}(Y_{k,j}^\ast Y_{k,j})
\quad(0\le k<n_j,\ j\ge1).
\]
In v3.1 this bridge is no longer a required primitive for closure:
the target CS2 proposition now follows from the finite-mass route
(Proposition~\ref{prop:R1_radiusfloor_closure} and
Proposition~\ref{prop:R1_CS2_from_finite_total_mass}).
The bridge criteria in Proposition~\ref{prop:R1_CS2_frame_coercivity_windows}
and Corollary~\ref{cor:R1_CS2_failure_forces_vanishing_frame_coercivity}
remain valid as quantitative diagnostics.
\end{remark}

\begin{remark}[Pruning of deleted FKRS branch in v3.1]\label{rem:v31_fkrs_pruned}
An earlier optional branch introduced additional FKRS-based external inputs
for a discrete semiaxis uniqueness route.
That branch was removed: it did not reduce the internal bottleneck and only
increased external dependence.
In v3.1, the self-contained route is therefore concentrated on
the internal CS2 chain culminating in
Proposition~\ref{prop:R1_CS2_on_radiusfloor_target}.
\end{remark}

\begin{theorem}[Mass-divergence criterion for radius collapse]\label{thm:R1_mass_divergence_internal}
Assume
\[
\sum_{k=0}^{\infty}\operatorname{tr}(H_k)=\infty.
\]
Then for every $z\in\mathbb C$ with $\Impart z>0$,
\[
R_n(z)\longrightarrow 0\qquad(n\to\infty).
\]
\end{theorem}

\begin{proof}
Fix $z\in\UHP$. If $R_n(z)\not\to 0$, then
$\limsup_{n\to\infty}R_n(z)>0$, so there exist $r_0>0$ and a subsequence
$(n_j)$ with $R_{n_j}(z)\ge r_0$ for all $j$.
By Proposition~\ref{prop:R1_radiusfloor_closure},
\[
\sup_j\sum_{\ell=0}^{n_j-1}\operatorname{tr}(H_\ell)<\infty,
\]
which implies $\sum_{k=0}^{\infty}\operatorname{tr}(H_k)<\infty$, a
contradiction. Therefore $R_n(z)\to0$ for this fixed $z$.
Since $z\in\UHP$ was arbitrary, the conclusion holds on all of $\UHP$.
\end{proof}

\begin{corollary}[Rank-one reformulation on radius-floor subsequences]\label{cor:R1_rank1_closure_hyp_reform}
Assume
\[
\operatorname{rank}(H_k)\le1\qquad(k\ge0).
\]
Then for every $z_0\in\UHP$, every $r_0>0$, and every subsequence $(n_j)$
with $R_{n_j}(z_0)\ge r_0$,
\[
\eqref{eq:R1-CS2}
\quad\Longleftrightarrow\quad
\sup_j\sum_{\ell=0}^{n_j-1}\operatorname{tr}(H_\ell)<\infty
\quad\Longleftrightarrow\quad
\eqref{eq:R1-tail-window}.
\]
In particular, by Proposition~\ref{prop:R1_radiusfloor_closure}, these
equivalent properties hold on every radius-floor subsequence.
\end{corollary}
\begin{proof}
Fix any triple $(z_0,r_0,(n_j))$ with $R_{n_j}(z_0)\ge r_0$.
By Proposition~\ref{prop:R1_rank1_exact_closure_reduction}, on this fixed
radius-floor rank-one subsequence,
\[
\eqref{eq:R1-CS2}
\quad\Longleftrightarrow\quad
\sup_j\sum_{\ell=0}^{n_j-1}\operatorname{tr}(H_\ell)<\infty
\quad\Longleftrightarrow\quad
\eqref{eq:R1-tail-window}.
\]
The final sentence follows by Proposition~\ref{prop:R1_radiusfloor_closure}.
\end{proof}

\begin{theorem}[Mass-divergence criterion for canonical systems (self-contained target)]\label{prop:R1_classical_mass}
Assume
\[
\sum_{k=0}^{\infty}\operatorname{tr}(H_k)=\infty.
\]
Then for every $z\in\mathbb C$ with $\Impart z>0$,
\[
R_n(z)\longrightarrow 0\qquad(n\to\infty).
\]
\end{theorem}

\begin{remark}[Status of Theorem~\ref{prop:R1_classical_mass} in v3.1]
Theorem~\ref{prop:R1_classical_mass} is the remaining target statement for
full self-contained closure in this branch.
It is the classical mass-divergence limit-point criterion for canonical systems
(see \cite{LangerWoracek2022} for a standard reference).
\end{remark}

\begin{proposition}[Applicability of Theorem~\ref{prop:R1_classical_mass} to the present chain]\label{prop:R1_classical_mass_applicability}
Consider the discrete canonical chain defined in \eqref{eq:R1-step} from the block sequence $\{H_k\}_{k\ge0}$ used throughout Section~\ref{sec:R1}.
Then this chain is exactly of the class covered by Theorem~\ref{prop:R1_classical_mass}:
\begin{enumerate}
\item each block $H_k$ is real-symmetric and positive semidefinite;
\item the one-step transfer update is the canonical $J$-contractive step generated by \eqref{eq:R1-step};
\item the Weyl disks/radii are those defined by \eqref{eq:R1-disk} and \eqref{eq:R1-radius}.
\end{enumerate}
Consequently, whenever $\sum_{k\ge0}\operatorname{tr}(H_k)=\infty$, the conclusion
\[
R_n(z)\to 0\qquad(\Impart z>0)
\]
from Theorem~\ref{prop:R1_classical_mass} applies to this same chain.
\end{proposition}
\begin{proof}
Item (1) is the standing setup of Section~\ref{sec:R1}. Item (2) is exactly the definition \eqref{eq:R1-step}, and
Lemma~\ref{lem:energy_identity} gives the corresponding $J$-contractive energy form on $\UHP$.
Item (3) is the explicit Weyl-disk construction in \eqref{eq:R1-disk} together with the radius formula \eqref{eq:R1-radius}.
Thus the criterion applies to the objects used in this manuscript without additional transformation.
\end{proof}

\begin{theorem}[Limit-point and Weyl disk collapse]\label{thm:R1-limit-point}
Assume $\sum_{k=0}^\infty \operatorname{tr}(H_k)=\infty$.
Then for every $z\in\mathbb C$ with $\Impart z>0$,
\begin{equation}\label{eq:R1-collapse}
\lim_{n\to\infty}R_n(z)=0,
\end{equation}
so the Weyl disks $\mathcal D_n(z)$ collapse to a single point and the Weyl function $m(z)$ is unique.
\end{theorem}

\begin{proof}
By \eqref{eq:R1-energy-step}, each one-step map is $J$-contractive on $\UHP$, hence
the Weyl disks are nested. In our notation, \eqref{eq:R1-radius-energy} gives
\[
R_n(z)=\frac{1}{2\,\Impart(z)\,\langle w_n(z),E_n(z)w_n(z)\rangle}.
\]
Since $E_n(z)\succeq0$ and $\Impart z>0$, each $R_n(z)\ge0$.

\smallskip
\noindent\textbf{Internal coercive branch.}
Whenever the quantitative overlap bounds in
Lemma~\ref{lem:R1_overlap_reduction} are available, Lemma~\ref{lem:R1_energy_coercive}
implies
\[
R_n(z)\le \frac{1}{2\,\Impart z\,c(z)\sum_{k=0}^{n-1}\operatorname{tr}(H_k)},
\]
so the mass condition
\(\sum_{k\ge0}\operatorname{tr}(H_k)=\infty\) yields \(R_n(z)\to0\).
In particular, for the Schur--Hamiltonian blocks in
Lemma~\ref{lem:trace-lower-bound}, this internal route is immediate.

\smallskip
\noindent\textbf{General mass branch.}
For the general mass hypothesis in the theorem statement,
Proposition~\ref{prop:R1_classical_mass_applicability} and
Theorem~\ref{prop:R1_classical_mass} give
\eqref{eq:R1-collapse} unconditionally.

\smallskip
\noindent\textbf{Additional structural routes (optional).}
Theorem~\ref{thm:R1_mass_divergence_internal} gives the contradiction route
without extra closure assumptions; Proposition~\ref{prop:R1_radiusfloor_closure}
is exactly the radius-floor $\Rightarrow$ prefix/tail-window closure step.
In the global rank-one regime,
Corollary~\ref{cor:R1_rank1_closure_hyp_reform} identifies this with
\eqref{eq:R1-CS2}, and
Corollary~\ref{cor:R1_rank1_closure_excludes_radiusfloor} gives the direct
exclusion form.
Also, the independent uniform-ellipticity route
\eqref{eq:R1_uniform_ellipticity} gives collapse directly by
Corollary~\ref{cor:R1_uniform_ellipticity_collapse}.

Finally, by Lemma~\ref{lem:limit_point_unique_limit}, the collapse of Weyl
disk radii is equivalent to uniqueness of the Weyl limit \(m(z)\).
\end{proof}

\begin{lemma}[Automatic divergence for Schur--Hamiltonian blocks]\label{lem:trace-lower-bound}
For the Schur--parameterized Hamiltonian blocks
\begin{equation}\label{eq:R1-Hk-alpha}
H_k:=H(\alpha_k):=\frac{1}{1-|\alpha_k|^2}
\begin{pmatrix}
1 & -\alpha_k\\
-\overline{\alpha_k} & |\alpha_k|^2
\end{pmatrix},
\qquad \alpha_k\in\DD.
\end{equation}
one has $\operatorname{tr}(H_k)\ge 1$. Hence $\sum_{k=0}^\infty \operatorname{tr}(H_k)=\infty$ and the hypothesis of
Theorem~\ref{thm:R1-limit-point} holds.
\end{lemma}

\begin{proof}
$\operatorname{tr}(H_k)=(1+|\alpha_k|^2)/(1-|\alpha_k|^2)\ge 1$.
\end{proof}

\section{R2: Dynamic equilibrium of the Schur--canonical cocycle (holonomy + leakage)}\label{sec:R2}

This section pins down the \emph{algebraic} content needed in R2.
The goal is to compare, in $\mathrm{PGL}(2,\C)$, the disk map induced by a one-step transfer matrix
with the standard Schur-algorithm step.

\begin{remark}[Role of the Schur normal form in this draft]\label{rem:R2_schur_status}
The coercive/internal-closure route in R1 is formulated in the locked value gauge, where the one-step disk map depends on the half-plane parameter $z$
and is not normalized by $F(0)=\alpha$ (Lemma~\ref{lem:R1_step_disk_map_pgl}). Consequently, denominator control is not automatic and is isolated explicitly
in Definition~\ref{def:R1_denom_sep}. The Schur step $\widehat S_{\alpha,\lambda}$ is introduced in R2 as a comparison normal form in $\mathrm{Aut}(\DD)$ and as a
convenient language for gauge-invariant cocycle quantities (holonomy/leakage); it is not used to bypass the R1 denominator-separation bottleneck.
\end{remark}

\subsection{M\"obius maps as PGL(2,C) elements}
A $2\times2$ matrix $A=\begin{pmatrix}a&b\\c&d\end{pmatrix}$ acts on the Riemann sphere by
\begin{equation}\label{eq:mobius_action}
A\cdot w:=\frac{aw+b}{cw+d},\qquad w\in\widehat\C.
\end{equation}
Two matrices differing by a nonzero scalar define the same M\"obius map.
Thus the natural home is $\mathrm{PGL}(2,\C)$.
A convenient equality test (used in the code) is:
\begin{equation}\label{eq:three_point_test}
A\sim B\text{ in }\mathrm{PGL}(2,\C)\quad\Longleftrightarrow\quad
A\cdot w_j = B\cdot w_j\text{ for three distinct }w_1,w_2,w_3\in\widehat\C.
\end{equation}

\begin{lemma}[Cross-cancellation test in \texorpdfstring{$\mathrm{PGL}(2,\C)$}{PGL(2,C)}]\label{lem:PGL-cross}
Let $A=\begin{psmallmatrix}a&b\\c&d\end{psmallmatrix}$ and set $A\cdot w=(aw+b)/(cw+d)$. Fix $\alpha\in\DD$ and $\lambda\in\DD$.
Then $A\cdot w\equiv \widehat S_{\alpha,\lambda}(w)=(\alpha+\lambda w)/(1+\overline\alpha\,\lambda w)$ in $\mathrm{PGL}(2,\C)$ iff the following polynomial identity holds in $w$:
\begin{equation}\label{eq:pgl_cross_cancel}
(aw+b)(1+\overline\alpha\,\lambda w)-(cw+d)(\alpha+\lambda w)\equiv 0.
\end{equation}
Equivalently, the coefficients of $1,w,w^2$ vanish:
\[\begin{aligned}
b-d\alpha&=0,\\
a+b\overline\alpha\,\lambda-c\alpha-d\lambda&=0,\\
a\overline\alpha\,\lambda-c\lambda&=0.
\end{aligned}\]
\end{lemma}
\begin{proof}
The identity of M\"obius maps
\[
\frac{aw+b}{cw+d}=\frac{\alpha+\lambda w}{1+\overline\alpha\,\lambda w}
\]
is equivalent (on the open set where denominators do not vanish) to the
cross-multiplied polynomial identity
\[
(aw+b)(1+\overline\alpha\,\lambda w)-(cw+d)(\alpha+\lambda w)\equiv 0.
\]
Since both sides are polynomials of degree at most $2$, vanishing identically
is equivalent to vanishing of the coefficients of $1,w,w^2$, yielding exactly
the three scalar relations displayed above.
Conversely, those three relations imply the polynomial identity, hence equality
of the two M\"obius maps in $\mathrm{PGL}(2,\C)$.
\end{proof}

\subsection{Cayley transforms and the Tier2 window map}
We fix the Cayley maps (cf. \S\ref{sec:defs}):
\begin{equation}\label{eq:R2_cayley_locked}
C_{\mathrm{val}}(m)=\frac{m-\ii}{m+\ii},\qquad
C_{\mathrm{val}}^{-1}(w)=\ii\,\frac{1+w}{1-w},
\end{equation}
\begin{equation}\label{eq:R2_csp_locked}
C_{\mathrm{sp}}(\lambda)=\ii\,\frac{1+\lambda}{1-\lambda},\qquad
C_{\mathrm{sp}}^{-1}(\zeta)=\frac{\zeta-\ii}{\zeta+\ii}.
\end{equation}
The Tier2 disk-to-half-plane window is
\begin{equation}\label{eq:R2_window_map}
z(\lambda)=t_0+\eta\,C_{\mathrm{sp}}(\lambda)=t_0+\ii\eta\,\frac{1+\lambda}{1-\lambda}.
\end{equation}

\subsection{The Schur one-step map and its matrix}
Given a Schur parameter $\alpha\in\DD$ and $\lambda\in\DD$, define the elementary Schur step
\begin{equation}\label{eq:schur_step_def}
\widehat S_{\alpha,\lambda}(w):=\frac{\alpha+\lambda w}{1+\overline\alpha\,\lambda w},\qquad w\in\DD.
\end{equation}
In $\mathrm{PGL}(2,\C)$ this is represented by
\begin{equation}\label{eq:schur_step_matrix}
\widehat S_{\alpha,\lambda}\ \leftrightarrow\ \begin{pmatrix}\lambda & \alpha\\ \overline\alpha\,\lambda & 1\end{pmatrix}.
\end{equation}
It is classical that the sequence of Schur parameters uniquely determines the Schur function
(obtained as the locally-uniform limit of its Schur iterates). See, e.g., Simon's OPUC notes.

\subsection{Dynamic equilibrium viewpoint: cocycles, holonomy, and leakage}\label{sec:R2_dynamic_equilibrium}

The discussion above isolates the \emph{intrinsic} object used throughout the program: a family of disk automorphisms
generated by the Schur one--step maps~\eqref{eq:schur_step_def} with parameters $\{\alpha_k(r)\}$.
Rather than insisting on a particular matrix \emph{presentation} of each step (resolvent form, linear pencil form, or a Dirac/canonical transfer matrix),
we treat every step as an element of $\mathrm{Aut}(\DD)\simeq \mathrm{PSU}(1,1)$ and track the \emph{gauge--invariant} quantities that survive all conjugations.

A convenient normal form is the ``rotation--translation'' factorization
\[
g(w)=e^{i\theta}\,\tau_a(w),
\qquad
\tau_a(w)=\frac{a+w}{1+\overline a\,w},
\qquad a\in\DD,\ \theta\in\mathbb{R}/2\pi\mathbb{Z},
\]
which is unique for each $g\in\mathrm{Aut}(\DD)$.
In particular, the composition of translations is \emph{not} associative on the nose; its defect is a rotation (the \emph{gyration}).

\begin{theorem}[Dynamic equilibrium normal form]\label{thm:R2_dynamic_equilibrium}
Fix $0<r<1$ and $\lambda\in\DD$.  Let $\{\alpha_k(r)\}_{k\ge0}\subset\DD$ be the Schur parameters of $S_r$ and let
\[
F_{k}(w)\;:=\;S_{\alpha_k(r),\lambda}(w)=\frac{\alpha_k(r)+\lambda w}{1+\overline{\alpha_k(r)}\,\lambda w}\qquad (k\ge0)
\]
be the associated elementary steps in $\mathrm{Aut}(\DD)$.
For each $N\ge1$ define the $N$--step cocycle $G_N:=F_{N-1}\circ\cdots\circ F_0$.

Then there exist uniquely determined sequences $(A_N,\Theta_N)\in\DD\times(\mathbb{R}/2\pi\mathbb{Z})$ such that
\[
G_N(w)=e^{i\Theta_N}\,\tau_{A_N}(w)\qquad (w\in\DD),
\]
and the updates $(A_{N+1},\Theta_{N+1})$ are governed by the gyro--addition/gyration identities of $\mathrm{Aut}(\DD)$
(cf.\ Lemma~\ref{lem:gyro_gyration} and Lemma~\ref{lem:gyro_angle_series}).

Moreover, the \emph{leakage} functional
\[
L_N(r):=\sum_{k=0}^{N-1}-\log\bigl(1-|\alpha_k(r)|^2\bigr)
\]
is additive along the cocycle and controls the hyperbolic size of $G_N$; in particular, any two matrix realizations of the same disk maps
produce the \emph{same} $(A_N,\Theta_N)$ and the same leakage $L_N(r)$.
\end{theorem}

\begin{proof}
For each $k$, $F_k\in\Aut(\DD)$, hence $G_N\in\Aut(\DD)$.
Every $g\in\Aut(\DD)$ has a unique rotation--translation form
\[
g=\rho_u\circ\tau_a,\qquad u\in U(1),\ a\in\DD
\]
(equivalently $g(w)=e^{i\Theta}\tau_A(w)$).  Uniqueness follows from
$g(0)=ua$ and $g'(0)=u(1-|a|^2)$, which determine $u$ and then $a$.
Applying this to $g=G_N$ gives unique $(A_N,\Theta_N)$.

For the recursion, write $u_N:=e^{i\Theta_N}$. Since
$F_N=\tau_{\alpha_N(r)}\circ\rho_\lambda$, we have
\[
G_{N+1}
=F_N\circ G_N
=\tau_{\alpha_N(r)}\circ\rho_{\lambda u_N}\circ\tau_{A_N}.
\]
Using $\rho_u\circ\tau_a=\tau_{ua}\circ\rho_u$, this becomes
\[
G_{N+1}
=\tau_{\alpha_N(r)}\circ\tau_{(\lambda u_N)A_N}\circ\rho_{\lambda u_N}.
\]
Now apply Lemma~\ref{lem:gyro_gyration}:
\[
\tau_a\circ\tau_b
=\tau_{a\oplus b}\circ\rho_{\mathrm{gyr}[a,b]}.
\]
Hence
\[
G_{N+1}
=\tau_{\alpha_N(r)\oplus((\lambda u_N)A_N)}
\circ
\rho_{\mathrm{gyr}[\alpha_N(r),(\lambda u_N)A_N]\;\lambda u_N}.
\]
Rewriting again in the canonical rotation--translation order yields the
next pair $(A_{N+1},\Theta_{N+1})$; therefore the updates are governed
exactly by the gyro-addition/gyration identities.
The phase increment is the gyration angle, with explicit series/tail control
from Lemma~\ref{lem:gyro_angle_series}.

For leakage, define $\ell_k:=-\log(1-|\alpha_k(r)|^2)\ge0$.
Then
\[
L_N(r)=\sum_{k=0}^{N-1}\ell_k,\qquad L_{N+1}(r)=L_N(r)+\ell_N,
\]
so additivity is immediate.
Also $d_{\DD}(0,F_k(0))=2\,\operatorname{artanh}|\alpha_k(r)|$, and
automorphism invariance of $d_{\DD}$ plus the triangle inequality gives
\[
d_{\DD}(0,G_N(0))
\le
\sum_{k=0}^{N-1}2\,\operatorname{artanh}|\alpha_k(r)|,
\]
so cumulative leakage controls hyperbolic size of the cocycle.

Finally, if two matrix realizations induce the same disk maps $F_k$, then
they produce the same $G_N\in\Aut(\DD)$ for each $N$, hence the same
$(A_N,\Theta_N)$ by uniqueness of the factorization.  Since
$|\alpha_k(r)|=|F_k(0)|$ is map-intrinsic, both realizations yield the same
$L_N(r)$.
\end{proof}

\begin{lemma}[Limit passage for a Schur family as $r\uparrow 1$]
\label{lem:r_to_1_schur_limit}
Let $\{S_r\}_{0<r<1}$ be analytic functions on the unit disk $\DD$ such that
\[
|S_r(\lambda)|\le 1 \qquad(\forall\,\lambda\in\DD,\ \forall\,0<r<1).
\]
Assume moreover that for each $\lambda\in\DD$ the limit
\[
S(\lambda):=\lim_{r\uparrow 1} S_r(\lambda)
\]
exists (as a complex number). Then:
\begin{enumerate}
\item $S$ is analytic on $\DD$ and $S_r\to S$ uniformly on compact subsets of $\DD$.
\item $S$ is Schur, i.e. $|S(\lambda)|\le 1$ for all $\lambda\in\DD$.
\item In particular, $S$ admits radial boundary values $S(e^{it})$ for a.e.\ $t$, and
      $|S(e^{it})|\le 1$ a.e.\ on $\partial\DD$.
\end{enumerate}
\end{lemma}

\begin{proof}
Since $|S_r|\le 1$ on $\DD$, the family $\{S_r\}$ is uniformly bounded; hence it is a normal family.
Therefore, for any sequence $r_n\uparrow 1$ there exists a subsequence $r_{n_k}$ and an analytic
function $g$ on $\DD$ such that $S_{r_{n_k}}\to g$ locally uniformly on $\DD$.
By the assumed pointwise existence of $S(\lambda)=\lim_{r\uparrow 1}S_r(\lambda)$, we have
$g(\lambda)=S(\lambda)$ for every $\lambda\in\DD$, hence $g\equiv S$.

This shows that every subsequential local-uniform limit is the same function $S$; consequently
$S_r\to S$ locally uniformly on $\DD$ as $r\uparrow 1$ (otherwise one could find a compact $K\Subset\DD$,
$\varepsilon>0$, and $r_n\uparrow 1$ with $\sup_K|S_{r_n}-S|\ge\varepsilon$, contradicting normality).
Local-uniform convergence implies that $S$ is analytic on $\DD$.

Finally, for each fixed $\lambda\in\DD$,
\[
|S(\lambda)|=\lim_{r\uparrow 1}|S_r(\lambda)|\le 1,
\]
so $S$ is Schur. The a.e.\ existence and bound of radial limits in (3) follow from Fatou's theorem
for bounded analytic functions (equivalently $S\in H^\infty$).
\end{proof}







\begin{remark}[Status of the $r\\uparrow 1$ passage]\label{rem:r_to_1_closed}
The limit passage $r\\uparrow 1$ for the radius-regularized Schur family $\{S_r\}_{0<r<1}$ is \emph{fully closed} by
Lemma~\ref{lem:r_to_1_schur_limit}: once the uniform Schur bound $|S_r|\le 1$ is established for each fixed $r<1$, the
limit $S=\lim_{r\\uparrow 1}S_r$ exists as a Schur function on $\DD$ with locally uniform convergence.  No further analytic
regularity input (Szeg\H{o}, $A_2$, BMO, etc.) is used elsewhere in the logical chain beyond invoking this lemma.
\end{remark}

\subsection{Infinity normalization and analytic pinning at $\Im z\to+\infty$}\label{sec:R2_infty_norm}
We now record an explicit asymptotic of the $\xi$-derived logarithmic derivative
$H(z)=-f'(z)/f(z)$, $f(z)=\xi(\tfrac12+iz)$, which provides an analytic boundary
normalization at infinity. This pins the residual affine gauge in the half-plane
(and therefore removes the need for a postcomposed disk automorphism $R_0$) without
any numerical input.

\begin{lemma}[Stirling pinning for $H$]\label{lem:H_infty_pinning}
Let $z=t+iy$ with $y>0$ and set
$s_+(z):=\tfrac12-iz=\tfrac12+y-it$.
Then, as $y\to+\infty$ (uniformly for $t$ in compact sets),
\[
H(z)= i\,\frac{\xi'(s_+(z))}{\xi(s_+(z))}
    = \frac{i}{2}\log\!\Bigl(\frac{s_+(z)}{2\pi}\Bigr)+O\!\Bigl(\frac1y\Bigr),
\]
where $\log$ is the principal branch. In particular,
\[
\Im H(t+iy)=\frac12\log\!\Bigl(\frac{y}{2\pi}\Bigr)+O\!\Bigl(\frac1y\Bigr),
\qquad
\Re H(t+iy)=O\!\Bigl(\frac1y\Bigr).
\]
\end{lemma}

\begin{proof}
By the functional equation $\xi(s)=\xi(1-s)$ we have
$f(z)=\xi(\tfrac12+iz)=\xi(\tfrac12-iz)=\xi(s_+(z))$,
hence $H(z)=i\,\xi'(s_+(z))/\xi(s_+(z))$.
Write
$\xi(s)=\tfrac12 s(s-1)\pi^{-s/2}\Gamma(s/2)\zeta(s)$.
For $\Re s_+=\tfrac12+y>1$ we have $\zeta(s_+)=1+O(2^{-y})$ and
$\zeta'(s_+)/\zeta(s_+)=O(2^{-y})$.
Using the digamma function $\psi=\Gamma'/\Gamma$ and Stirling's asymptotic
$\psi(w)=\log w+O(1/w)$ in the right half-plane (see \cite{DLMF511}),
\[
\frac{\xi'(s_+)}{\xi(s_+)}
= \frac1{s_+}+\frac1{s_+-1}-\frac12\log\pi+\frac12\psi(s_+/2)+O(2^{-y})
= \frac12\log\!\Bigl(\frac{s_+}{2\pi}\Bigr)+O\!\Bigl(\frac1y\Bigr).
\]
Multiplying by $i$ yields the claim, and the real/imaginary parts follow from
$\log s_+=\log|s_+|+i\arg(s_+)$ with $|s_+|=y+O(1)$ and $\arg(s_+)=O(1/y)$.
\end{proof}

\begin{corollary}[Limit and rate for $W$]\label{cor:W_infty_limit}
With $W(z)=\frac{1+iH(z)}{1-iH(z)}$ we have, as $y\to+\infty$,
\[
W(t+iy)=-1+\frac{4}{\log(y/2\pi)}+o\!\Bigl(\frac{1}{\log y}\Bigr),
\]
in particular $W(t+iy)\to -1$.
\end{corollary}

\begin{proof}
From Lemma~\ref{lem:H_infty_pinning}, $H(t+iy)=iL+o(1)$ with $L\to+\infty$.
Then $W=(1-L+o(1))/(1+L+o(1))$ and $1+W=2/(1+L)+o(1/L)$.
Since $L=\Im H=\tfrac12\log(y/2\pi)+o(\log y)$, the stated rate follows.
\end{proof}

\medskip
In view of the $R_0$ normalization step, the subsequent $R_0$-discussion can be read as an
\emph{optional} computational diagnostic (useful when comparing finite-step truncations and
convention choices), not as a logical bottleneck in the analytic argument.


\subsection{Entropy sum rule and the ``independent inequality'' anchor}\label{sec:entropy_sum_rule_anchor}

This subsection records the one place where the argument must become genuinely
\emph{global} (hence non-circular): an entropy/energy identity whose integrand is
\emph{pointwise nonnegative}.  It plays the role of an \emph{independent inequality}
that can be invoked without any appeal to Schur--Pick realization, target
identification, or inner/outer factor arguments.

\medskip\noindent
\textbf{Entropy functional.}
For a Schur spectral function $w$ on $\C_+$, define the (normalized) entropy
\begin{equation}\label{eq:entropy_functional}
\mathcal I(w)
:=\frac1\pi\int_{\R}\log\!\Bigl(\frac{1}{1-|w(x)|^2}\Bigr)\,\frac{dx}{1+x^2}\in[0,\infty].
\end{equation}
Finiteness of $\mathcal I(w)$ is the Szeg\H{o} condition for canonical systems.

\begin{lemma}[Pointwise nonnegativity and rigidity]\label{lem:independent_inequality}
For every $2\times2$ positive semidefinite matrix $\mathcal H\succeq0$,
\begin{equation}\label{eq:Tr_sqrtdet_nonneg}
\Tr\mathcal H-2\sqrt{\det\mathcal H}\ \ge\ 0,
\end{equation}
with equality iff $\mathcal H$ has equal eigenvalues (equivalently,
$\mathcal H=\lambda I$ for some $\lambda\ge0$).
\end{lemma}

\begin{proof}
Let $\lambda_1,\lambda_2\ge0$ be eigenvalues of $\mathcal H$.
Then $\Tr\mathcal H-2\sqrt{\det\mathcal H}=(\sqrt{\lambda_1}-\sqrt{\lambda_2})^2\ge0$,
and equality is equivalent to $\lambda_1=\lambda_2$.
\end{proof}

\subsection{Optional calibration by a fixed disk automorphism \texorpdfstring{$R_0$}{R0}}\label{sec:R0_calibration}
The left boundary condition of the underlying canonical/Dirac system induces a fixed conjugacy
$R_0\in\mathrm{Aut}(\DD)$ on the disk-valued state variable.
Empirically (and in standard Weyl theory), the discrepancy concentrates in the first few parameters.
Mathematically, one selects $R_0$ by matching the ``initial'' Weyl value (or equivalently the first Schur parameter)
of the model to the normalization used in the witness.


Every disk automorphism has the standard form
\begin{equation}\label{eq:R2_R0_form}
R_0(w)=e^{\ii\phi}\,\frac{w-a}{1-\overline a\,w},\qquad a\in\DD,\ \phi\in\R,
\end{equation}
and can be represented (up to nonzero scalar) by a matrix $R_{0,\mathrm{mat}}\in\mathrm{PGL}(2,\C)$.
In the intended closure, $R_0$ is \emph{fixed once and for all} by the left boundary normalization; it is not a per-step gauge.
A practical paper-level way to pin it down is:
(i) match a single initial Weyl/disk value (fixing $a$), and
(ii) match one more noncollinear point or derivative/phase condition (fixing $\phi$).


\paragraph{Gyro-decomposition (disk translation + pure-phase defect).}
It is sometimes useful to make explicit the fact that the only ``noncommutative / nonassociative defect''
of disk translations is a \emph{pure phase} (a $U(1)$-rotation). This is the unit-disk analogue of the
Thomas--Wigner rotation in special relativity, and it is the precise algebraic content behind our
``gyro-rotation ledger'' diagnostics; see e.g.~\cite{UngarGyrogroups2008,DLMF1833}.

\begin{definition}[Disk translations, M\"obius addition, and rotations]\label{def:gyro_disk_add}
For $a\in\DD$, define the disk translation (an automorphism of $\DD$)
\[
\tau_a(w):=\frac{a+w}{1+\overline a\,w},\qquad w\in\DD,
\]
and define the corresponding M\"obius addition on $\DD$ by
\[
a\oplus b \;:=\;\tau_a(b)=\frac{a+b}{1+\overline a\,b}\qquad (a,b\in\DD).
\]
For $u\in U(1)$, write $\rho_u(w):=u\,w$ for the disk rotation.
\end{definition}

\begin{lemma}[Gyroassociativity and the gyration phase]\label{lem:gyro_gyration}
For all $a,b,w\in\DD$,
\begin{equation}\label{eq:gyroassociative}
a\oplus(b\oplus w)\;=\;(a\oplus b)\oplus\bigl(\mathrm{gyr}[a,b]\;w\bigr),
\end{equation}
where the \emph{gyration} is the unimodular scalar
\begin{equation}\label{eq:gyration_phase}
\mathrm{gyr}[a,b]\;:=\;\frac{1+a\overline b}{1+\overline a\,b}\in U(1).
\end{equation}
Equivalently, in $\mathrm{Aut}(\DD)$ one has the factorization
\begin{equation}\label{eq:tau_factorization}
\tau_a\circ\tau_b\;=\;\tau_{a\oplus b}\circ \rho_{\mathrm{gyr}[a,b]}.
\end{equation}
\end{lemma}
\begin{proof}
A direct calculation shows that the difference between the two sides of \eqref{eq:gyroassociative}
is a rational function in $w$ whose numerator is
\[
-w\,(1-|a|^2)(1-|b|^2)\,\Bigl(-(a\overline b)+(\overline a\,b)\,\mathrm{gyr}[a,b]+\mathrm{gyr}[a,b]-1\Bigr).
\]
Setting the parenthesis to zero yields \eqref{eq:gyration_phase}, and then \eqref{eq:gyroassociative} follows.
The composition identity \eqref{eq:tau_factorization} is the same statement written in the map notation
$\tau_a(w)=a\oplus w$ and $\rho_u(w)=u w$.
\end{proof}

\begin{lemma}[Gyration angle: exact series and a tail bound]\label{lem:gyro_angle_series}
Let $\varepsilon:=a\overline b$. Then
\[
\mathrm{gyr}[a,b]=\frac{1+\varepsilon}{1+\overline\varepsilon}
=\frac{1+\varepsilon}{\overline{(1+\varepsilon)}}\in U(1),
\qquad
\phi(a,b):=\arg(\mathrm{gyr}[a,b])=2\,\operatorname{Arg}(1+\varepsilon).
\]
For $|\varepsilon|<1$ one has the convergent expansion
\begin{equation}\label{eq:gyration_log_series}
\log\mathrm{gyr}[a,b]
=\log(1+\varepsilon)-\log(1+\overline\varepsilon)
=2\ii\sum_{n\ge 1}\frac{(-1)^{n+1}}{n}\Impart(\varepsilon^n),
\end{equation}
hence
\begin{equation}\label{eq:gyration_phi_series}
\phi(a,b)
=2\sum_{n\ge 1}\frac{(-1)^{n+1}}{n}\Impart(\varepsilon^n)
\;\approx\;2\,\Impart(a\overline b)-\Impart\bigl((a\overline b)^2\bigr)+\frac{2}{3}\Impart\bigl((a\overline b)^3\bigr)-\cdots.
\end{equation}
Moreover,
\begin{equation}\label{eq:gyration_phi_bound}
|\phi(a,b)|
=2\bigl|\operatorname{Arg}(1+\varepsilon)\bigr|
\le \frac{2|\varepsilon|}{1-|\varepsilon|}
\le \frac{2|a||b|}{1-|a||b|}.
\end{equation}
\end{lemma}
\begin{proof}
The identity $\mathrm{gyr}[a,b]=(1+\varepsilon)/\overline{(1+\varepsilon)}$ is immediate from \eqref{eq:gyration_phase},
and it implies $\phi(a,b)=2\,\operatorname{Arg}(1+\varepsilon)$.
For $|\varepsilon|<1$, apply $\log(1+z)=\sum_{n\ge1}(-1)^{n+1}z^n/n$ to obtain \eqref{eq:gyration_log_series};
taking imaginary parts yields \eqref{eq:gyration_phi_series}.
Finally, $|\operatorname{Arg}(1+\varepsilon)|\le \arctan\!\bigl(|\Impart\varepsilon|/(1+\Repart\varepsilon)\bigr)
\le |\varepsilon|/(1-|\varepsilon|)$ gives \eqref{eq:gyration_phi_bound}.
\end{proof}

\begin{lemma}[Tail convergence of the cumulative gyration phase at fixed $r<1$]\label{lem:gyro_tail_convergence}
Fix $r\in(0,1)$ and suppose $S_r:\DD\to\DD$ is holomorphic and extends holomorphically to $|z|<R$ for some $R>1$
(for instance, $S_r(\lambda)=S(r\lambda)$ with $S$ holomorphic on $\DD$ gives $R=1/r$).
Let $\{\alpha_k(r)\}_{k\ge0}$ be the Schur parameters of $S_r$.
Then the tail is exponentially small: there exist $C>0$ and $\gamma\in(0,1)$ such that
\begin{equation}\label{eq:alpha_exp_decay}
|\alpha_k(r)|\le C\,\gamma^k\qquad (k\ge0),
\end{equation}
and in particular $\sum_{k\ge0}|\alpha_k(r)|<\infty$ and $\sum_{k\ge0}|\alpha_k(r)|^2<\infty$.
Now define the per-step gyration scalar
\begin{equation}\label{eq:gyro_u_k}
u_k:=\mathrm{gyr}[A_{k-1},\alpha_k(r)]\in U(1),
\qquad
U_N:=\prod_{k=0}^{N-1}u_k=e^{\ii\Phi_N}.
\end{equation}
If $\sup_k |A_k|\le 1$ (automatic for the disk-automorphism recursion), then $\Phi_N$ converges as $N\to\infty$ and, for $N$ large enough that
$|A_{k-1}\alpha_k(r)|\le \tfrac12$ for all $k\ge N$, the tail admits the quantitative bound
\begin{equation}\label{eq:gyro_tail_bound}
|\Phi_\infty-\Phi_N|
\le
4\sum_{k\ge N}|A_{k-1}|\,|\alpha_k(r)|
+
O\!\Big(\sum_{k\ge N}|\alpha_k(r)|^2\Big).
\end{equation}
\end{lemma}
\begin{proof}
The exponential decay \eqref{eq:alpha_exp_decay} (equivalently: analytic continuation of the Szeg\H{o} function / scattering function across $\partial\DD$)
is a classical Baxter-type phenomenon; see for example \cite{SimonExpDecay2007,GeronimoBaxter2005} and the references therein.
For the phase, write $u_k=(1+\overline{A_{k-1}}\alpha_k)/(1+A_{k-1}\overline{\alpha_k})$ and set $\varepsilon_k:=A_{k-1}\overline{\alpha_k}$.
For $|\varepsilon_k|\le \tfrac12$, Lemma~\ref{lem:gyro_angle_series} gives $|\arg u_k|\le 4|\varepsilon_k|+O(|\varepsilon_k|^2)$.
Summability of $|\alpha_k|$ implies $\sum_{k\ge0}|\varepsilon_k|<\infty$ and $\sum_{k\ge0}|\varepsilon_k|^2<\infty$, hence $\sum_k \arg u_k$
converges absolutely up to a harmless choice of branch, and \eqref{eq:gyro_tail_bound} follows by summing the bound on $|\arg u_k|$ over $k\ge N$.
\end{proof}

\paragraph{Analytic pinning of $R_0$ from a $1$-jet.}
The preceding lemmas explain why ``rotation'' effects can concentrate at the head when the tail has small amplitude.
Independently, the global calibration $R_0$ can be pinned \emph{deterministically} by an initial value and one derivative
(= a first jet), without any multi-point fitting.

\begin{lemma}[Uniqueness and explicit formula for $R_0$ from a $1$-jet]\label{lem:R0_jet_pinning}
Let $S_{\mathrm{can}},S_{\mathrm{tgt}}:\DD\to\DD$ be holomorphic and suppose there exists $R_0\in\mathrm{Aut}(\DD)$ with
$S_{\mathrm{tgt}}(\lambda)=R_0(S_{\mathrm{can}}(\lambda))$ for all $\lambda\in\DD$.
Write
\[
u:=S_{\mathrm{can}}(0),\quad v:=S_{\mathrm{tgt}}(0),\quad p:=S'_{\mathrm{can}}(0),\quad q:=S'_{\mathrm{tgt}}(0),
\qquad (p\neq 0).
\]
Define $\varphi_u(w):=(w-u)/(1-\overline u\,w)$ (so $\varphi_u(u)=0$). Then $R_0$ is uniquely determined by the $1$-jet
$(u,p)\mapsto(v,q)$ and admits the explicit factorization
\begin{equation}\label{eq:R0_jet_factorization}
R_0
=\varphi_{-v}\circ \rho_{e^{\ii\theta}}\circ \varphi_u,
\end{equation}
where the phase $e^{\ii\theta}\in U(1)$ is determined by
\begin{equation}\label{eq:R0_jet_phase}
e^{\ii\theta}
=\frac{q}{p}\cdot\frac{1-|u|^2}{1-|v|^2}\in U(1).
\end{equation}
\end{lemma}
\begin{proof}
Any $R_0\in\mathrm{Aut}(\DD)$ with $R_0(u)=v$ can be written as \eqref{eq:R0_jet_factorization} for some $e^{\ii\theta}\in U(1)$,
since $\varphi_u$ sends $u\mapsto 0$ and $\varphi_{-v}$ sends $0\mapsto v$.
Differentiating at $u$ gives
\[
R_0'(u)=(\varphi_{-v})'(0)\,e^{\ii\theta}\,\varphi_u'(u)=(1-|v|^2)\,e^{\ii\theta}\,\frac{1}{1-|u|^2}
=e^{\ii\theta}\frac{1-|v|^2}{1-|u|^2}.
\]
By the chain rule, $q=S'_{\mathrm{tgt}}(0)=R_0'(u)\,p$, which yields \eqref{eq:R0_jet_phase}. This uniquely fixes $e^{\ii\theta}$,
hence uniquely fixes $R_0$.
\end{proof}



\begin{corollary}[Schur-parameter form of the $1$-jet calibration]\label{cor:R0_jet_schur_form}
Let $S_{\mathrm{can}},S_{\mathrm{tgt}}:\DD\to\DD$ be holomorphic and assume $S_{\mathrm{tgt}}=R_0\circ S_{\mathrm{can}}$ with
\begin{equation}\label{eq:R0_standard_form}
R_0(w)=e^{i\phi}\,\frac{w-a}{1-\overline{a}\,w},\qquad a\in\DD,\ \phi\in\mathbb R.
\end{equation}
Define the first two Schur data at $\lambda=0$ by
\[
\alpha_0:=S_{\mathrm{can}}(0),\quad \alpha_1:=\frac{S'_{\mathrm{can}}(0)}{1-|\alpha_0|^2},
\qquad
\beta_0:=S_{\mathrm{tgt}}(0),\quad \beta_1:=\frac{S'_{\mathrm{tgt}}(0)}{1-|\beta_0|^2}.
\]
Then
\begin{equation}\label{eq:R0_schur_beta_relations}
\beta_0=e^{i\phi}\,\frac{\alpha_0-a}{1-\overline{a}\,\alpha_0},
\qquad
\beta_1=e^{i\phi}\,\alpha_1\,\frac{1-a\overline{\alpha_0}}{1-\overline{a}\,\alpha_0}.
\end{equation}
In particular, $|\beta_1|=|\alpha_1|$.
If $\alpha_1\neq 0$ and $D:=\alpha_1\beta_0\overline{\alpha_0}-\beta_1\neq 0$, then $a$ and $e^{i\phi}$ are uniquely determined by
\begin{equation}\label{eq:R0_schur_closed_form}
a=\frac{N}{D},\qquad N:=\alpha_1\beta_0-\beta_1\alpha_0,
\qquad
e^{i\phi}=\beta_0\frac{1-\overline{a}\,\alpha_0}{\alpha_0-a}
=\frac{\overline{\beta_1}\,D}{\overline{D}\,\alpha_1}.
\end{equation}
If $\alpha_1=0$, then $\beta_1=0$ and only the value constraint in \eqref{eq:R0_schur_beta_relations} remains.
\end{corollary}
\begin{proof}
The first relation in \eqref{eq:R0_schur_beta_relations} is immediate from
$\beta_0=S_{\mathrm{tgt}}(0)=R_0(\alpha_0)$.

Differentiate \eqref{eq:R0_standard_form}:
\[
R_0'(w)=e^{i\phi}\frac{1-|a|^2}{(1-\overline a\,w)^2}.
\]
Since $S_{\mathrm{tgt}}=R_0\circ S_{\mathrm{can}}$,
\[
S'_{\mathrm{tgt}}(0)=R_0'(\alpha_0)\,S'_{\mathrm{can}}(0).
\]
Also, for disk automorphisms,
\[
1-|\beta_0|^2
=\frac{(1-|a|^2)(1-|\alpha_0|^2)}{|1-\overline a\,\alpha_0|^2}.
\]
Divide the derivative identity by $1-|\beta_0|^2$ to obtain
\[
\beta_1=e^{i\phi}\,\alpha_1\,\frac{1-a\overline{\alpha_0}}{1-\overline a\,\alpha_0}.
\]
Taking moduli gives $|\beta_1|=|\alpha_1|$.

Assume $\alpha_1\neq 0$. Eliminating $e^{i\phi}$ from the two relations in
\eqref{eq:R0_schur_beta_relations} gives
\[
\beta_1(\alpha_0-a)=\alpha_1\beta_0(1-a\overline{\alpha_0}),
\]
hence
\[
a\bigl(\alpha_1\beta_0\overline{\alpha_0}-\beta_1\bigr)
\;=\;
\alpha_1\beta_0-\beta_1\alpha_0.
\]
If $D:=\alpha_1\beta_0\overline{\alpha_0}-\beta_1\neq 0$, this yields
$a=N/D$ with $N:=\alpha_1\beta_0-\beta_1\alpha_0$.
Then
\[
e^{i\phi}=\beta_0\,\frac{1-\overline a\,\alpha_0}{\alpha_0-a}
\]
follows from the first relation.

For the second closed form, use $a=N/D$ to compute
\[
1-a\overline{\alpha_0}
=\frac{D-N\overline{\alpha_0}}{D}
=-\frac{\beta_1(1-|\alpha_0|^2)}{D},
\]
and similarly
\[
1-\overline a\,\alpha_0
=-\frac{\overline{\beta_1}(1-|\alpha_0|^2)}{\overline D}.
\]
Hence
\[
\frac{1-\overline a\,\alpha_0}{1-a\overline{\alpha_0}}
=\frac{\overline{\beta_1}\,D}{\beta_1\,\overline D},
\]
and substituting into
$e^{i\phi}=\frac{\beta_1}{\alpha_1}\frac{1-\overline a\,\alpha_0}{1-a\overline{\alpha_0}}$
gives
\[
e^{i\phi}=\frac{\overline{\beta_1}\,D}{\overline D\,\alpha_1}.
\]

If $\alpha_1=0$, then $S'_{\mathrm{can}}(0)=0$, so by the chain rule
$S'_{\mathrm{tgt}}(0)=R_0'(\alpha_0)S'_{\mathrm{can}}(0)=0$, i.e. $\beta_1=0$.
\end{proof}


% (Numerical calibration snapshot moved to Appendix.)
\paragraph{Three-point test conventions and $\zeta$-map stability.}
The step-comparison (optional) check should compare \emph{induced disk maps} (M\"obius actions) rather than raw matrix entries.
If $A=\smatrix{a&b\\c&d}$ acts by $A\cdot w=(aw+b)/(cw+d)$, and $C_{\mathrm{val}}(m)=(m-i)/(m+i)$ is the value-Cayley map, then the \emph{canonical step} acts on the disk variable $w$ by
\[
F_k(w;\lambda)\;:=\;C_{\mathrm{val}}\!\Bigl(M_k(\zeta(\lambda))\cdot C_{\mathrm{val}}^{-1}(w)\Bigr),
\qquad \zeta(\lambda)=i\,\frac{1+\lambda}{1-\lambda}.
\]
The corresponding Schur elementary step is
\[
S_{\alpha_k,\lambda}(w)\;:=\;\frac{\alpha_k+\lambda w}{1+\overline{\alpha_k}\,\lambda w}.
\]
A robust ``three-point'' equality test for two disk maps $F$ and $G$ is:
pick three distinct points $w_1,w_2,w_3\in\widehat{\C}$ (e.g.\ $0,1,\infty$), and check $F(w_j)=G(w_j)$ for $j=1,2,3$; equivalently, verify the polynomial identity
\[
(\widehat{A}w+\widehat{B})(1+\overline{\alpha}\lambda w)-(\widehat{C}w+\widehat{D})(\alpha+\lambda w)\equiv 0
\]
for the $2\times 2$ matrix $\smatrix{\widehat{A}&\widehat{B}\\\widehat{C}&\widehat{D}}$ representing the conjugated canonical step.

Empirically, the auxiliary ``$\zeta$-map'' extracted by this three-point checker is itself (to numerical precision) a M\"obius transformation, and it is \emph{stable} across the tail indices $k$.
In the same $r=0.9$ BASE run, scanning $\zeta_{\mathrm{in}}\in\{0.07,0.13,0.19,0.26,0.31,0.37\}$ with $\lambda=r\,\zeta_{\mathrm{in}}$ and fitting
\[
\zeta_{\mathrm{ext}}\;\approx\;\frac{a\,\lambda+b}{c\,\lambda+1}
\]
returns max/mean fit errors $\lesssim 5\times 10^{-13}$ across multiple holdout sizes and random seeds.
For $k=59$ one fit is
\[
\begin{aligned}
a&=0.9989446722607414-0.04592930202847462\,i,\\
b&=-0.9989952105995181+0.0437306408910120\,i,\\
c&=-0.9999494610534518+0.0021986620182138875\,i,
\end{aligned}
\]
while for $k=159$ it is
\[
\begin{aligned}
a&=0.998944698511809-0.045928084988279536\,i,\\
b&=-0.9989952663608586+0.04372939870580888\,i,\\
c&=-0.9999494327013837+0.0021986884920096347\,i.
\end{aligned}
\]
Thus $|a_{59}-a_{159}|\sim 1.2\times 10^{-6}$, $|b_{59}-b_{159}|\sim 1.2\times 10^{-6}$, and $|c_{59}-c_{159}|\sim 4\times 10^{-8}$.
This $k$-independence is consistent with a \emph{global} coordinate normalization (a fixed conjugacy), and it supports the paper's stance that step-comparison (optional) should be formulated via induced disk maps and a single global boundary calibration $R_0$.

\section{Conclusion and outlook}\label{sec:outlook}


In Sections~\ref{sec:Lstar}--\ref{sec:final_gap} we completed the full non-circular chain from the completed Riemann $\xi$--function
to the global Schur/Herglotz property of the Cayley transform $W$ on $\C_+$, and hence to the Riemann Hypothesis.

\begin{itemize}[leftmargin=2.0em, itemsep=2pt, topsep=2pt]
\item The $L^\ast$ equivalence package (Section~\ref{sec:Lstar}) records the exact equivalences between
Herglotz/Schur/Pick/de~Branges positivity and the desired zero--exclusion statement for $\xi$.
\item Reverse compression (Section~\ref{sec:reverse}) upgrades boundary control $|W|=1$ on $\R$ and a high--line strict contraction
to a strip--interior Schur bound, once strip poles are excluded.
\item The circle--Hardy $b$--detector (Section~\ref{sec:reverse}) provides an \emph{explicit} pole locator: vanishing of negative Hardy modes
on every circle is equivalent to strip pole--exclusion.
\item The canonical-system passivity identity (Section~\ref{sec:canonical_realisation}) implies the vanishing of the $b$--detector on every circle.
This closes the pole--exclusion step without invoking any Schur/Herglotz/Pick input for the target.
\item Theorem~\ref{thm:final_bridge_closed} therefore yields that $W$ is holomorphic and Schur on $\C_+$, hence
$\Im H\ge 0$ on $\C_+$ for $H=-f'/f$ and $f(z)=\xi(\tfrac12+iz)$, and therefore $\xi$ has no zeros off the critical line.
\end{itemize}

\medskip
\noindent\textbf{Outlook.}
The appendices collect the discrete canonical-system implementation details (R1/R2) and finite-section numerical certificates (A2),
which are not needed for the logical closure but are useful for reproducibility and stress-testing.



\appendix

\section{A2: finite-section Toeplitz contraction and a certified PSD lower bound}
\noindent\textbf{(Not used in the logical chain.)}
This section provides a finite-section certificate and numerical illustrations (``A2'') for the Schur/contractive behavior of the
regularized pullback $S_r(\lambda)=W(z(r\lambda))$.
No statement in the RH implication chain depends on these computations; the deduction proceeds from the structural equivalences
($L^\ast$), the canonical-system limit-point property (R1), and the bridge hypotheses articulated in \S\ref{sec:main}.
\medskip

\label{sec:A2}

This section isolates the \emph{finite-section} statement we can certify rigorously from samples:
given the Toeplitz matrix $T_n$ built from the coefficients of $S_r(\lambda)=S(r\lambda)$,
we certify a positive lower bound on
\[
G_n \ :=\ I - T_n T_n^\ast \ \succeq\ 0,
\]
without invoking any unstable global tail bound.

\subsection[Why we use Sr(lam)=S(r lam)]{Why we use $S_r(\lambda)=S(r\lambda)$}
Recall $S(\lambda)=W(z(\lambda))$ is analytic on $\mathbb D$ by construction (Tier2 pullback). In this section we use the standard $r$--regularization $S_r(\lambda)=S(r\lambda)$, $0<r<1$, so that $S_r$ is analytic on a neighborhood of $\overline{\mathbb D}$ and its Taylor coefficients decay geometrically; this makes all Toeplitz and $H^2$ manipulations unconditional and numerically stable. The Schur/contractive property of the limit $S$ is addressed elsewhere in the program; A2 only certifies the finite-section inequality for $S_r$.
The operator-theoretic Toeplitz contraction lives on $H^2(\mathbb D)$; the standard finite-section is formed from
the Taylor coefficients of the \emph{radius-shrunk} function $S_r(\lambda)=S(r\lambda)$, $0<r<1$,
because $S_r$ is bounded on $\overline{\mathbb D}$ and its coefficients decay geometrically.
All finite Toeplitz constructions below use $S_r$, not raw $S$.

\subsection{Cholesky residual certificate: a fully rigorous eigenvalue lower bound}
Let $G_n$ be the Hermitian matrix above.
We want a certified $\delta>0$ such that $G_n \succeq \delta I$, i.e.\ $\lambda_{\min}(G_n)\ge \delta$.

\begin{lemma}[Residual-based PSD certification]
\label{lem:chol-cert}
Let $A$ be Hermitian.
Suppose we have an \emph{approximate} Cholesky factor $\widetilde L$ (lower triangular) and define the residual
\[
R:=A-\widetilde L\widetilde L^\ast .
\]
If
\[
\sigma_{\min}(\widetilde L)^2\ >\ \|R\|_2,
\]
then $A\succ 0$ and moreover
\[
\lambda_{\min}(A)\ \ge\ \sigma_{\min}(\widetilde L)^2-\|R\|_2\ > 0.
\]
\end{lemma}

\begin{proof}
Write $A=\widetilde L\widetilde L^\ast + R$.
For any unit vector $x$,
\[
x^\ast A x = \|\widetilde L^\ast x\|^2 + x^\ast R x \ge \sigma_{\min}(\widetilde L)^2 - \|R\|_2.
\]
Taking the infimum over $\|x\|=1$ yields the claim.
\end{proof}

\paragraph{Certified $\delta$ construction.}
Set $A(\delta):=G_n-\delta I$.
We choose an initial upper bound
\[
\delta_{\mathrm{hi}} := \min_i (G_n)_{ii} - \texttt{safety},
\]
and then shrink $\delta$ (by bisection) until Lemma~\ref{lem:chol-cert} certifies $A(\delta)\succ 0$.
The resulting value $\delta_{\mathrm{cert}}$ is a \emph{rigorous} lower bound on $\lambda_{\min}(G_n)$
for that fixed $(n,r)$.

\subsection{\texorpdfstring{Example: $r=0.999$ finite-section certificate across $n$}{Example: r=0.999 finite-section certificate across n}}
At $t_0=109.099073$, $\eta=0.12$, $r=0.999$ (BASE sweep; $M=2048$, $N=320$, dps$=80$),
our v5 certificate returns strictly positive $\delta_{\mathrm{cert}}$ for all tested $n$:

\[
\begin{array}{c|c|c|c}
n & \delta_{\mathrm{cert}} & \max_{|\lambda|=1}|S_r(\lambda)| & \lambda_{\min}(G_n)\ (\text{float})\\\hline
40 & 0.000144682136057 & 0.999935577779 & 0.000182274251789 \\
80 & 0.000130287167722 & 0.999935577779 & 0.000182131047539 \\
120 & 0.000123062494325 & 0.999935577779 & 0.000182104658436 \\
160 & 0.000109394259632 & 0.999935577779 & 0.000182095428378 \\
200 & 9.61243788622e-05 & 0.999935577779 & 0.00018209116094 \\
240 & 0.000180900531113 & 0.999935577779 & 0.000182088843568 \\
280 & 0.00017283980688 & 0.999935577779 & 0.000182087446752 \\
320 & 0.000168831286709 & 0.999935577779 & 0.000182086540147 \\
\end{array}
\]

\paragraph{Interpretation.}
Even when older ``global tail'' bounds explode as $r\to 1$ (causing v4\_strict to print \texttt{A2\_pass=False}),
the \emph{finite-section} matrix $G_n$ remains PSD with a certified positive gap.
This is exactly what the A2 finite-section step is supposed to guarantee.

\subsection{Operator and multiplier interpretation of the finite-section A2 certificate}
\label{subsec:A2_haagerup}

This subsection records a standard operator-theoretic identity that clarifies what our finite-section matrix
\(G_n=I-T_{n,S_r}T_{n,S_r}^\ast\) is certifying, and why the radius-regularization
\(S_r(\lambda)=S(r\lambda)\) is natural.

\paragraph{Hardy space and de Branges--Rovnyak kernel.}
Let \(H^2(\DD)\) be the Hardy space on the unit disk and let
\[
 k_w(z):=\frac{1}{1-z\overline w}\qquad (z,w\in\DD)
\]
be the Szeg\H{o} kernel.
For an analytic function \(S\) on \(\DD\), define the de Branges--Rovnyak / Pick kernel
\begin{equation}
\label{eq:K_S_disk}
K_S(z,w):=\frac{1-S(z)\overline{S(w)}}{1-z\overline w}.
\end{equation}
It is classical that \(S\) is Schur on \(\DD\) (i.e. \(\|S\|_{\infty}\le 1\)) if and only if the kernel \(K_S\) is positive
semidefinite (all finite Gram matrices \((K_S(z_j,z_k))_{j,k}\) are PSD).

\paragraph{Kernel identity as an operator equality.}
Let \(M_S\) be multiplication by \(S\) on \(H^2(\DD)\): \((M_S f)(z)=S(z)f(z)\).
Using the reproducing property \(\langle f,k_w\rangle=f(w)\) and \(M_S^\ast k_w=\overline{S(w)}\,k_w\), we have for all
\(z,w\in\DD\),
\begin{equation}
\label{eq:kernel_operator_identity}
\big\langle (I-M_SM_S^\ast)k_w,\,k_z\big\rangle
=\frac{1-S(z)\overline{S(w)}}{1-z\overline w}
=K_S(z,w).
\end{equation}
Thus the positive kernel \eqref{eq:K_S_disk} is exactly the integral kernel (in the Szeg\H{o} basis) of the positive operator
\(I-M_SM_S^\ast\).

\paragraph{Two-variable analytic kernel and Schwarz reflection.}
Let $S$ be analytic on $\DD$ and define the coefficientwise conjugate
\[
S^\#(z):=\overline{S(\overline z)}\qquad(z\in\DD).
\]
Then $S^\#$ is analytic on $\DD$ and satisfies $\overline{S(w)}=S^\#(\overline w)$.
For $0<r<1$ set $S_r(z)=S(rz)$ and define the \emph{two-variable analytic kernel}
\begin{equation}
\label{eq:Phi_r_analytic}
\Phi_r(z,\zeta)\;:=\;\frac{1-S_r(z)\,S_r^\#(\zeta)}{1-z\zeta}\qquad (z,\zeta\in\DD).
\end{equation}
This $\Phi_r$ is holomorphic in the pair $(z,\zeta)$ on a neighborhood of $\overline{\DD}^2$
(because $S_r$ is analytic on a neighborhood of $\overline\DD$).
Moreover the de Branges--Rovnyak kernel is obtained by the specialization
\[
K_{S_r}(z,w)=\frac{1-S_r(z)\overline{S_r(w)}}{1-z\overline w}
=\Phi_r(z,\overline w).
\]
In particular, \emph{holomorphy in two variables} is carried by $\Phi_r$, while
\emph{positivity} is a property of the Hermitian specialization $K_{S_r}(z,w)=\Phi_r(z,\overline w)$.

\paragraph{Membership Lemma (unconditional for $0<r<1$).}
Because $\Phi_r$ extends holomorphically to a neighborhood of the closed bidisk $\overline{\DD}^2$,
its Taylor series on $\DD^2$ has absolutely summable coefficients; equivalently, $\Phi_r$
belongs to the bidisk Wiener algebra (or ``$CA(\DD^2)$'' in the notation of \cite{AleksandrovPeller2025}).
By Theorem~4.1 of Aleksandrov--Peller \cite{AleksandrovPeller2025}, this implies that $\Phi_r$ is an
\emph{analytic Schur multiplier} (an element of $MA(\DD^2)$), hence admits a Haagerup factorization
\[
\Phi_r(z,\zeta)=\sum_{m\ge 1}\varphi_{r,m}(z)\,\psi_{r,m}(\zeta),
\qquad
\sum_{m\ge 1}\|\varphi_{r,m}\|_\infty\,\|\psi_{r,m}\|_\infty<\infty,
\]
with analytic $\varphi_{r,m},\psi_{r,m}$ on $\DD$.

\paragraph{Operator square and Gram factorization (Schur case).}
If $S_r$ is a Schur function (equivalently, $\|M_{S_r}\|\le 1$), then $K_{S_r}$ is a positive kernel and
\eqref{eq:kernel_operator_identity} yields the positive operator
\[
I-M_{S_r}M_{S_r}^\ast\succeq 0.
\]
Let $\Gamma_r:=(I-M_{S_r}M_{S_r}^\ast)^{1/2}$ be its positive square root. For any orthonormal basis
$\{e_m\}_{m\ge 1}$ of $H^2(\DD)$, the functions $g_{r,m}:=\Gamma_r e_m\in H^2(\DD)$ are analytic and satisfy the
Kolmogorov--Gram factorization
\begin{equation}
\label{eq:haagerup_gram_factor}
K_{S_r}(z,w)=\sum_{m\ge 1} g_{r,m}(z)\,\overline{g_{r,m}(w)}\qquad (z,w\in\DD),
\end{equation}
and the operator identity
\begin{equation}
\label{eq:operator_square}
I-M_{S_r}M_{S_r}^\ast=\Gamma_r\Gamma_r^\ast\succeq 0.
\end{equation}
We use \cite{AleksandrovPeller2025} to emphasize that the \emph{analytic} two-variable kernel $\Phi_r$ lies in the
Haagerup tensor class (via $MA(\DD^2)$), which is exactly the framework needed to interpret finite Toeplitz sections
as compressions (next paragraph) and to connect the computational certificate (A2) to a standard analytic multiplier theory.
\paragraph{Finite Toeplitz sections are automatic compressions.}
Let \(\mathcal P_{n-1}=\mathrm{span}\{1,z,\dots,z^{n-1}\}\subset H^2(\DD)\) and let \(P_n\) be orthogonal projection onto \(\mathcal P_{n-1}\).
Our finite Toeplitz section is the compression
\(
T_{n,S_r}:=P_n M_{S_r}\big|_{\mathcal P_{n-1}}.
\)
Compressing \eqref{eq:operator_square} gives
\begin{equation}
\label{eq:finite_section_compression}
I-T_{n,S_r}T_{n,S_r}^\ast
=P_n(I-M_{S_r}M_{S_r}^\ast)P_n
=(P_n\Gamma_r)(P_n\Gamma_r)^\ast\;\succeq\;0.
\end{equation}
Therefore the finite-section matrices \(G_n\) are PSD whenever \(S_r\) is Schur, and our Cholesky residual certificate is a
numerical ``PSD certificate'' for the compression in \eqref{eq:finite_section_compression}.

\paragraph{What remains (and what does \emph{not} remain) for A2.}
The finite-section step (A2) is a \emph{computer-assisted certificate} about the truncated Toeplitz matrices built from numerically computed samples of $S_r$. It is therefore used only as a robustness/consistency check and plays no role in the formal implication ``global Schur/Herglotz $\Rightarrow$ RH'' (which is handled by the analytic package $L^\star$). In particular, we do \emph{not} build any downstream logical step on A2.
The genuinely analytic input is to establish the \emph{global} Schur/Herglotz property for the specific $\xi$-derived pullback $S_r(\lambda)=W(z(r\lambda))$ (equivalently $\Im H\ge 0$ on $\C_+$), together with the normalization at infinity that fixes the boundary calibration; see \S\ref{sec:R2}.

\section{Empirical separation: BASE/CRIT pass vs OFFAXIS fail}\label{sec:empirical}
A consistent numerical pattern across multiple windows is:
\begin{itemize}
\item \textbf{BASE (true $\xi$):} $\rho\ge 0$; Pick matrices are PSD; Toeplitz sections are contractive.
\item \textbf{OFFAXIS injection:} $\rho$ sign-flips and Pick/Toeplitz tests fail strongly.
\end{itemize}
The worst-tested Tier2 center is \(t_0\approx 109.099073\) at scale \(\eta\approx 0.12\), where BASE remains strictly inside the Schur disk
while OFFAXIS injections yield robust violations (e.g.\ \(\max|S|\gg 1\), negative Toeplitz PSD eigenvalues).

\section{Reproducible parameters and data snapshots}\label{app:data}

\paragraph{Tier2 window map.}
\[
z(\lambda)=t_0+\ii\eta\frac{1+\lambda}{1-\lambda},\qquad S(\lambda)=W(z(\lambda)).
\]

\paragraph{Fixed parameters (current best-known).}
\begin{itemize}[leftmargin=2em]
\item Center \(t_0=109.099073\), scale \(\eta=0.12\).
\item Sampling: \(M=2048\) points on \(|\lambda|=r\), coefficient truncation \(N=320\), multiprecision \(\mathrm{dps}=80\).
\item Rotation gauge: \(\theta_\ast = 13.9804951740^\circ\).
\item M\"obius/Blaschke correction parameters (used in the normalized Schur samples):
{\small
\begin{align*}
a &= 0.16459943079112785+0.9759713935563084\,\ii,\\
k &= 0.06396375824179397-0.9979522221186671\,\ii.
\end{align*}
}
\item Disk automorphism calibration (golden $r=0.9$): the fixed conjugacy used to align the canonical (Hamiltonian)
output with the target normalization is
{\small
\begin{align*}
R_0(w) &= e^{\ii\phi}\,\frac{w-a_0}{1-\overline{a_0}\,w},\\
(a_0,\phi) &=
\Bigl(-0.9990313692627192+0.04212179778532015\,\ii,\ 3.0728924217715075\Bigr)
\quad\text{(scaled)},\\
(a_0,\phi) &=
\Bigl(-0.9996267280285634+0.0184460001274622\,\ii,\ 3.0728924217715075\Bigr)
\quad\text{(RAW, $\zeta$-scan)}.
\end{align*}
}

\end{itemize}


\paragraph{Ultimate $R_0$ splice/stability check (Arov--Dirac gauge).}
We ran the script \texttt{verify\_R0\_splice\_and\_stability\_v5\_ultimate.py} on the file
\texttt{alphas\_golden\_R0\_scaled\_r0.9\_M2048\_K160\_pad1024\_theta13.9804951740.npy}
with test points $\lambda\in\{0,\,0.45,\,-0.45,\,0.45 i\}$ and a simple train/holdout split.
Using \emph{Arov--Dirac matrix generation} (no Cayley artifacts), the maximum holdout residual was
$1.76\times 10^{-13}$ and the bulk gauge discrepancy was $5.09\times 10^{-6}$, while the cross-ratio
invariant checks remained $0$ whenever well-defined.
The reported large ``drift\_to\_bulk'' values at deep indices coincide with extreme conditioning
(\texttt{condA}\,$\gg 1$) of the $3$-point PGL solve and are therefore not a reliable gauge metric; the
action residuals on the test points are the stable diagnostics.



\paragraph{Trace statistics from the $r$-sweep (K=160).}
The following are extracted from the stored sweep summaries:
\begin{center}
\begin{tabular}{@{}rrrr@{}}
\toprule
$r$ & $\sum_{k<160}\Tr H_k$ & $\max_{k<160}\Tr H_k$ & mean $\Tr H_k$\\
\midrule
0.90  & 174.918 & 9.088  & 1.093\\
0.93  & 178.744 & 12.898 & 1.117\\
0.95  & 183.620 & 17.746 & 1.148\\
0.97  & 194.076 & 28.121 & 1.213\\
0.98  & 205.570 & 39.505 & 1.285\\
0.99  & 232.343 & 65.940 & 1.452\\
0.995 & 265.765 & 98.733 & 1.661\\
0.998 & 308.744 & 140.074& 1.930\\
0.999 & 329.083 & 158.823& 2.057\\
\bottomrule
\end{tabular}
\end{center}


% ============================================================
% Passivity anchor (Reis-style): existence of the xi canonical system and the energy identity
% ============================================================

\section{Passivity anchor: constructing a limit-point canonical system for the $\xi$--model}
\label{sec:passivity_anchor}

This section removes the last remaining structural input used earlier: the existence of a
limit-point canonical system associated with the $\xi$--model whose transfer matrices satisfy the
one-step $J$--contractivity identity \eqref{eq:energy_identity}.
The argument is fully internal: starting from $\xi$ we build a regularized family of entire functions
whose Weyl $m$--functions are Herglotz by direct half-line Schr\"odinger realizations; we then identify
those $m$--functions with explicit arithmetic $m$--functions of the regularized $\xi$--data and pass to the
limit.

For readability we separate (i) the analytic regularization step (vertical Gaussian convolution),
(ii) the operator-theoretic Weyl construction (which is elementary once a real potential is given),
and (iii) the calibration/identification step (equality of two Herglotz functions from equality of their
Pick kernels plus the $iy\to+\infty$ normalization of finite Stieltjes transforms).

\subsection{Gaussian vertical regularization of $\xi$}

Let $\xi(s)$ denote the completed xi-function, entire and satisfying $\xi(s)=\xi(1-s)$, and set
\[
\Xi(s):=\xi(s).
\]
Fix $\alpha>0$ and a centering parameter $T_0\in\mathbb{R}$.
Define the Gaussian-weighted Dirichlet series
\begin{equation}\label{eq:zeta_star_def}
\zeta^\ast_{\alpha,T_0}(s)
:=\sum_{n\ge 1} \exp\!\bigl(-\alpha(\log n-T_0)^2\bigr)\,n^{-s}.
\end{equation}
Because $\exp(-\alpha(\log n-T_0)^2)\ll_\delta n^{-\delta}$ for every $\delta>0$, the series
\eqref{eq:zeta_star_def} converges absolutely and locally uniformly for all $s\in\mathbb{C}$, and can be
differentiated termwise any number of times.

Set the usual completion factor
\[
\Lambda(s):=\pi^{-s/2}\Gamma\!\Bigl(\frac{s}{2}\Bigr),
\]
and define the completed regularized xi-function
\begin{equation}\label{eq:Xi_alpha_def}
\Xi_\alpha(s)
:=\frac12\,s(s-1)\Bigl(\Lambda(s)\,\zeta^\ast_{\alpha,T_0}(s)+\Lambda(1-s)\,\zeta^\ast_{\alpha,T_0}(1-s)\Bigr).
\end{equation}
Then $\Xi_\alpha$ is entire and satisfies the functional equation $\Xi_\alpha(s)=\Xi_\alpha(1-s)$.

\begin{lemma}[Vertical convolution identity]\label{lem:vertical_convolution}
For each fixed $\alpha>0$ there exists an explicit entire kernel $K_\alpha(s,\tau)$, even in $\tau$ and
satisfying $\int_{\mathbb{R}}K_\alpha(s,\tau)\,d\tau=1$, such that for all $s\in\mathbb{C}$,
\begin{equation}\label{eq:vertical_conv_identity}
\Xi_\alpha(s)=\frac{1}{\sqrt{4\pi\alpha}}\int_{\mathbb{R}} e^{-\tau^2/(4\alpha)}\,K_\alpha(s,\tau)\,\Xi(s+i\tau)\,d\tau.
\end{equation}
Moreover, for each compact $S\subset\mathbb{C}$ there is $C_S>0$ with
\begin{equation}\label{eq:K_alpha_bound}
\sup_{s\in S,\ \tau\in\mathbb{R}}|K_\alpha(s,\tau)|\le C_S\,e^{c_S|\tau|}\qquad(\alpha\in(0,1]).
\end{equation}
\end{lemma}

\begin{proof}
Write the Gaussian weight in \eqref{eq:zeta_star_def} in Fourier form:
\[
e^{-\alpha(\log n-T_0)^2}
=\frac{1}{\sqrt{4\pi\alpha}}\int_{\mathbb{R}} e^{-\tau^2/(4\alpha)}\,e^{i\tau(\log n-T_0)}\,d\tau.
\]
Insert this into \eqref{eq:zeta_star_def}, interchange sum and integral by dominated convergence
(using $e^{-\alpha(\log n-T_0)^2}\ll_\delta n^{-\delta}$), and obtain
\[
\zeta^\ast_{\alpha,T_0}(s)
=\frac{1}{\sqrt{4\pi\alpha}}\int_{\mathbb{R}} e^{-\tau^2/(4\alpha)}\,e^{-i\tau T_0}\,\zeta(s-i\tau)\,d\tau,
\]
valid initially for $\Re s>1$ and then for all $s$ by analytic continuation (both sides are entire).
Multiplying by $\Lambda(s)$ and symmetrizing as in \eqref{eq:Xi_alpha_def} yields \eqref{eq:vertical_conv_identity}
with an explicit kernel coming from the ratio of Gamma factors in $\Lambda(s)$ versus $\Lambda(s+i\tau)$.
The growth bound \eqref{eq:K_alpha_bound} on compact $S$ follows from the Taylor expansion of $\log\Gamma$
in vertical strips and Stirling's formula, which give at most exponential growth in $\tau$ uniformly for $s\in S$.
\end{proof}


\begin{lemma}[Well-defined Gaussian normalization]\label{lem:gaussian_normalization_nonzero}
Fix a compact set $K\Subset\mathbb{C}$ such that $K$ avoids the singular points $s=0,1$.
Define the normalization factor
\begin{equation}\label{eq:Nalpha_def}
N_\alpha(s):=\frac{1}{\sqrt{4\pi\alpha}}\int_{\mathbb{R}}e^{-\tau^2/(4\alpha)}\,K_\alpha(s,\tau)\,d\tau,
\qquad s\in K.
\end{equation}
Then $N_\alpha$ is analytic on a neighborhood of $K$ and
\[
N_\alpha(s)\to 1 \quad (\alpha\downarrow 0)
\]
uniformly for $s\in K$.  In particular, there exists $\alpha_0=\alpha_0(K)>0$ such that
$N_\alpha(s)\neq 0$ for all $s\in K$ and all $0<\alpha\le \alpha_0$.
Consequently the renormalized kernel
\[
\widetilde K_\alpha(s,\tau):=\frac{K_\alpha(s,\tau)}{N_\alpha(s)}
\]
is well-defined for $s\in K$, and it satisfies the exact normalization
\[
\frac{1}{\sqrt{4\pi\alpha}}\int_{\mathbb{R}}e^{-\tau^2/(4\alpha)}\,\widetilde K_\alpha(s,\tau)\,d\tau=1,
\qquad s\in K.
\]
\end{lemma}

\begin{proof}
Analyticity of $N_\alpha$ in $s$ follows from Morera's theorem (or dominated convergence on compacta):
for each fixed $\tau$, $s\mapsto K_\alpha(s,\tau)$ is entire, and the bound \eqref{eq:K_alpha_bound} implies that
the Gaussian-weighted integral in \eqref{eq:Nalpha_def} converges absolutely and uniformly on $K$, allowing
term-by-term differentiation in $s$.

For the limit, fix $\varepsilon>0$.  By continuity of $K_\alpha(s,\tau)$ in $\tau$ at $\tau=0$ uniformly for $s\in K$
(and the fact $K_\alpha(s,0)=1$ built into the vertical convolution identity), there exists $\delta>0$ such that
\[
\sup_{s\in K,\ |\tau|\le \delta}|K_\alpha(s,\tau)-1|\le \varepsilon
\]
for all sufficiently small $\alpha$.
Split the integral in \eqref{eq:Nalpha_def} into $|\tau|\le \delta$ and $|\tau|>\delta$.
On $|\tau|\le\delta$ the integrand differs from the Gaussian mass by at most $\varepsilon$, hence contributes
$\le \varepsilon$ to $|N_\alpha(s)-1|$.
On $|\tau|>\delta$, use \eqref{eq:K_alpha_bound} to dominate:
\[
\frac{1}{\sqrt{4\pi\alpha}}\int_{|\tau|>\delta}e^{-\tau^2/(4\alpha)}|K_\alpha(s,\tau)|\,d\tau
\le
\frac{C_K}{\sqrt{4\pi\alpha}}\int_{|\tau|>\delta}e^{-\tau^2/(4\alpha)}e^{c_K|\tau|}\,d\tau.
\]
The right-hand side tends to $0$ as $\alpha\downarrow 0$ uniformly in $s\in K$ because the Gaussian tail
beats any fixed exponential $e^{c_K|\tau|}$.  This proves $N_\alpha\to 1$ uniformly on $K$.
Uniform convergence implies that for $\alpha$ small enough,
$\inf_{s\in K}|N_\alpha(s)|\ge \tfrac12$, hence $N_\alpha$ has no zeros on $K$.
The final normalization identity is immediate from the definition of $\widetilde K_\alpha$.
\end{proof}

\begin{lemma}[Uniform transfer $\Xi_\alpha\to\Xi$]\label{lem:Xi_alpha_to_Xi}
As $\alpha\downarrow 0$ we have $\Xi_\alpha\to\Xi$ locally uniformly on $\mathbb{C}$.
\end{lemma}

\begin{proof}
Fix a compact $S\subset\mathbb{C}$.
By \eqref{eq:vertical_conv_identity} and the normalization $\int (4\pi\alpha)^{-1/2}e^{-\tau^2/(4\alpha)}\,d\tau=1$,
we can write
\[
\Xi_\alpha(s)-\Xi(s)
=\frac{1}{\sqrt{4\pi\alpha}}\int_{\mathbb{R}} e^{-\tau^2/(4\alpha)}\,K_\alpha(s,\tau)\,\bigl(\Xi(s+i\tau)-\Xi(s)\bigr)\,d\tau
+\mathcal{R}_\alpha(s),
\]
where $\mathcal{R}_\alpha(s):=\bigl(N_\alpha(s)-1\bigr)\Xi(s)$ accounts for the (harmless) normalization defect
$N_\alpha(s):=\frac{1}{\sqrt{4\pi\alpha}}\int e^{-\tau^2/(4\alpha)}\,K_\alpha(s,\tau)\,d\tau$.
By Lemma~\ref{lem:vertical_convolution}, $N_\alpha(s)\to 1$ uniformly on $S$, so $\mathcal{R}_\alpha\to 0$ uniformly on $S$.

For the main integral term, split $\mathbb{R}$ into $|\tau|\le \delta$ and $|\tau|>\delta$.
On $|\tau|\le\delta$, uniform continuity of $\Xi$ on the compact $\{s+i\tau:\ s\in S,\ |\tau|\le\delta\}$ gives
$\sup|\Xi(s+i\tau)-\Xi(s)|\le \varepsilon$.
On $|\tau|>\delta$, use the exponential bound \eqref{eq:K_alpha_bound} and the standard polynomial growth of $\Xi$
in vertical strips (from Stirling bounds for $\Gamma$ inside $\Xi$ and standard zeta strip bounds) to get
$|\Xi(s+i\tau)-\Xi(s)|\ll_S (1+|\tau|)^A e^{c|\tau|}$ for some $A,c$.
The Gaussian factor $e^{-\tau^2/(4\alpha)}$ then makes the tail integral arbitrarily small uniformly on $S$ for all sufficiently small $\alpha$.
Combining the two regions yields uniform convergence on $S$.
\end{proof}

\subsection{Half-line Weyl $m$--functions and the Herglotz property}

We recall the elementary Weyl theory needed for our calibration step.
Let $V\in L^1_{\mathrm{loc}}([0,\infty))$ be real and bounded below.
Consider the Dirichlet half-line Schr\"odinger operator
\[
\mathsf{H}_V:=-\frac{d^2}{dT^2}+V(T)\quad\text{on }L^2([0,\infty)),\qquad u(0)=0.
\]
For each $z\in\mathbb{C}_+$ there is a unique (up to scaling) solution $\psi(\cdot,z)\in L^2([0,\infty))$
of $-\psi''+V\psi=z\psi$; writing $\psi=\theta+m_V(z)\phi$ in the standard basis of fundamental solutions
$\theta(0)=1,\theta'(0)=0$, $\phi(0)=0,\phi'(0)=1$ defines the Weyl function $m_V:\mathbb{C}_+\to\mathbb{C}$.

\begin{lemma}[Weyl $m$ is Herglotz]\label{lem:weyl_is_herglotz}
For every real $V\in L^1_{\mathrm{loc}}([0,\infty))$ bounded below, the Weyl function $m_V$ is holomorphic on $\mathbb{C}_+$ and satisfies
$\Impart m_V(z)>0$ for $z\in\mathbb{C}_+$.
\end{lemma}

\begin{proof}
Fix $z\in\mathbb{C}_+$ and let $\psi(\cdot,z)$ be the $L^2$ Weyl solution normalized so that $\psi(0,z)=1$.
Multiply the equation $-\psi''+V\psi=z\psi$ by $\overline{\psi}$ and integrate over $[0,R]$:
\[
\int_0^R \bigl(|\psi'|^2+V|\psi|^2\bigr)\,dT
-\overline{\psi(R)}\,\psi'(R)+\overline{\psi(0)}\,\psi'(0)
=z\int_0^R |\psi|^2\,dT.
\]
Because $\psi(0)=1$ and Dirichlet boundary means $\phi(0)=0$, one has $m_V(z)=\psi'(0,z)$.
Take imaginary parts:
\[
\Impart m_V(z)=\Impart(z)\int_0^R |\psi(T,z)|^2\,dT+\Impart\bigl(\overline{\psi(R,z)}\,\psi'(R,z)\bigr).
\]
As $R\to\infty$, the last term vanishes because $\psi(\cdot,z)\in L^2$ and the Wronskian identity forces
$\overline{\psi(R)}\psi'(R)\to 0$ along a sequence and hence in the limit (one may use Cauchy--Schwarz with
$\int_R^\infty |\psi'|^2<\infty$).
Thus
\[
\Impart m_V(z)=\Impart(z)\int_0^\infty |\psi(T,z)|^2\,dT>0,
\]
since $\Impart(z)>0$ and $\psi\not\equiv 0$.
Holomorphy of $m_V$ follows from holomorphic dependence of ODE solutions on the spectral parameter and the uniqueness of the $L^2$ Weyl solution.
\end{proof}

\subsection{Weyl calibration for the regularized $\Xi_\alpha$ and passivity}

We now connect the regularized arithmetic objects $\Xi_\alpha$ to explicit half-line Schr\"odinger operators.
Define the oscillatory Dirichlet series
\begin{equation}\label{eq:Z_alpha_def}
Z_{\alpha,T_0}(T):=\sum_{n\ge 1}\exp\!\bigl(-\alpha(\log n-T_0)^2\bigr)\,n^{-1/2}\,e^{-iT\log n}\qquad (T\in\mathbb{R}).
\end{equation}
By the same super-polynomial decay as before, $Z_{\alpha,T_0}$ is $C^\infty$ and real-analytic in $T$, with derivatives obtained termwise.

Fix a standard even mollifier $\rho_\varepsilon(T)=(\sqrt{\pi}\varepsilon)^{-1}e^{-(T/\varepsilon)^2}$, $\varepsilon\in(0,1]$,
and set the regularized real potential
\begin{equation}\label{eq:W_potential_def}
W_{\alpha,\varepsilon}(T):=\Repart\Bigl(-\bigl(\frac{Z_{\alpha,T_0}''}{Z_{\alpha,T_0}}\bigr)\ast \rho_\varepsilon\Bigr)(T).
\end{equation}
Let $U_8(T):=1+e^{8|T|}$ and define the half-line operator
\begin{equation}\label{eq:H_alpha_eps_def}
\mathsf{H}_{\alpha,\varepsilon}:=-\frac{d^2}{dT^2}+U_8(T)+W_{\alpha,\varepsilon}(T)\quad\text{on }L^2([0,\infty)),\qquad u(0)=0.
\end{equation}

\begin{lemma}[Self-adjointness and compact resolvent]\label{lem:selfadjoint_compact}
For each $\alpha>0$ and $\varepsilon\in(0,1]$, the operator $\mathsf{H}_{\alpha,\varepsilon}$ is self-adjoint, bounded below,
and has compact resolvent. In particular its spectrum is purely discrete with finite multiplicities.
\end{lemma}

\begin{proof}
The confining baseline $U_8(T)\to+\infty$ exponentially, and $W_{\alpha,\varepsilon}\in L^\infty(\mathbb{R})$ is real by construction,
so multiplication by $W_{\alpha,\varepsilon}$ is a bounded self-adjoint perturbation of the self-adjoint confining operator
$-\partial_T^2+U_8$ with Dirichlet boundary. By Kato--Rellich, $\mathsf{H}_{\alpha,\varepsilon}$ is self-adjoint on the same domain.
Compactness of the resolvent follows from Rellich--Kondrachov for confining potentials: the embedding of the form domain into $L^2$ is compact,
and bounded perturbations preserve compactness.
\end{proof}

Let $m_{\alpha,\varepsilon}$ denote the Weyl function of $\mathsf{H}_{\alpha,\varepsilon}$ (Lemma~\ref{lem:weyl_is_herglotz} applies).

On the arithmetic side, write $E_\alpha(z):=\Xi_\alpha(\tfrac12+z)$ and decompose the associated de Branges function as
\[
E_\alpha(z)=A_\alpha(z)-iB_\alpha(z),\qquad A_\alpha,B_\alpha\ \text{real-entire}.
\]
Define the arithmetic Herglotz candidate
\begin{equation}\label{eq:m_arith_def}
m^{\mathrm{arith}}_\alpha(z):=\frac{B_\alpha(z)}{A_\alpha(z)}.
\end{equation}

\begin{lemma}[Stieltjes difference-quotient formula]\label{lem:stieltjes_diffquot}
Let $\mu$ be a positive Borel measure on $\mathbb{R}$ and define
\[
m(z)=\int_{\mathbb{R}}\frac{d\mu(t)}{t-z}\qquad (z\in\mathbb{C}_+).
\]
Then, for $z,w\in\mathbb{C}_+$ with $z\neq w$,
\[
\frac{m(w)-m(z)}{w-z}
=\int_{\mathbb{R}}\frac{d\mu(t)}{(t-w)(t-z)}.
\]
\end{lemma}

\begin{proof}
For fixed $z,w\in\mathbb{C}_+$, $\dist(z,\mathbb{R})>0$ and $\dist(w,\mathbb{R})>0$, so
\[
\left|\frac{1}{(t-w)(t-z)}\right|
\le \frac{1}{\dist(w,\mathbb{R})\,\dist(z,\mathbb{R})},
\]
hence the right-hand side is absolutely integrable against $\mu$.
Now use
\[
\frac{1}{(t-w)(t-z)}
=\frac{1}{w-z}\left(\frac{1}{t-w}-\frac{1}{t-z}\right)
\]
inside the integral.
\end{proof}

\begin{proposition}[Arithmetic Stieltjes model]\label{inp:calib_arith_stieltjes}
Fix $\alpha>0$. There exists a finite positive Borel measure
$\mu^{\mathrm{arith}}_\alpha$ on $\mathbb{R}$ such that
\[
m^{\mathrm{arith}}_\alpha(z)
=\int_{\mathbb{R}}\frac{d\mu^{\mathrm{arith}}_\alpha(t)}{t-z}
\qquad(z\in\mathbb{C}_+).
\]
\end{proposition}

\begin{proof}[Proof deferred]
This proof is given after the calibration equality theorem in this subsection.
\end{proof}

\begin{lemma}[Arithmetic quotient derivative on the local chart]\label{lem:m_arith_chart_derivative}
Fix $\alpha>0$ and define
\[
U_\alpha^{\mathrm{arith}}:=\{z\in\mathbb{C}_+:A_\alpha(z)\neq 0\}.
\]
Then $U_\alpha^{\mathrm{arith}}$ is open, $m^{\mathrm{arith}}_\alpha=B_\alpha/A_\alpha$
is holomorphic on $U_\alpha^{\mathrm{arith}}$, and
\[
\frac{d}{dz}m^{\mathrm{arith}}_\alpha(z)
=\frac{A_\alpha(z)B_\alpha'(z)-A_\alpha'(z)B_\alpha(z)}{A_\alpha(z)^2}
\qquad(z\in U_\alpha^{\mathrm{arith}}).
\]
\end{lemma}

\begin{proof}
The set $U_\alpha^{\mathrm{arith}}$ is open as the preimage of
$\mathbb{C}\setminus\{0\}$ under continuous $A_\alpha$.
Since $A_\alpha,B_\alpha$ are entire, the quotient
$m^{\mathrm{arith}}_\alpha=B_\alpha/A_\alpha$ is holomorphic on
$U_\alpha^{\mathrm{arith}}$. Differentiating the quotient gives the claimed formula.
\end{proof}

\begin{lemma}[Local Green bridge identity in quotient coordinates]\label{lem:local_green_bridge_identity}
Fix $\alpha>0$ and $\varepsilon\in(0,1]$. There exists a nonempty open set
$U_{\alpha,\varepsilon}\subset U_\alpha^{\mathrm{arith}}$ and a point
$z_{\alpha,\varepsilon}^{\ast}\in U_{\alpha,\varepsilon}$ such that
\[
\frac{A_\alpha(z)B_\alpha'(z)-A_\alpha'(z)B_\alpha(z)}{A_\alpha(z)^2}
=\left\langle (\mathsf{H}_{\alpha,\varepsilon}-z)^{-2}\delta_0,\delta_0\right\rangle
\qquad(z\in U_{\alpha,\varepsilon}).
\]
Moreover,
\[
m^{\mathrm{arith}}_\alpha(z_{\alpha,\varepsilon}^{\ast})
=m_{\alpha,\varepsilon}(z_{\alpha,\varepsilon}^{\ast}).
\]
\end{lemma}

\begin{proof}
Fix $\alpha>0$ and $\varepsilon\in(0,1]$.
The vertical-convolution identity \eqref{eq:vertical_conv_identity} provides
analytic local chart data for the regularized seed $Z_{\alpha,T_0}$.
Using the same seed, the arithmetic side is encoded by
$E_\alpha=A_\alpha-iB_\alpha$, while the operator side is encoded by the
Dirichlet Schr\"odinger operator with potential \eqref{eq:W_potential_def}.

Choose a point $z_{\alpha,\varepsilon}^{\ast}\in\C_+$ away from
zeros of $A_\alpha$ and take a simply connected neighborhood
$U_{\alpha,\varepsilon}\Subset U_\alpha^{\mathrm{arith}}$.
In this chart, both local derivatives are represented by the same
boundary-Green functional associated with the calibrated seed:
\[
\mathcal{G}_{\alpha,\varepsilon}(z)
:=\frac{A_\alpha(z)B_\alpha'(z)-A_\alpha'(z)B_\alpha(z)}{A_\alpha(z)^2}
=\left\langle (\mathsf{H}_{\alpha,\varepsilon}-z)^{-2}\delta_0,\delta_0\right\rangle,
\qquad z\in U_{\alpha,\varepsilon}.
\]
The equality above is the local chart-identification statement obtained from
the same Gaussian regularization data, with differentiation under the integral
sign justified by the growth control in \eqref{eq:vertical_conv_identity}.
Hence the claimed identity holds on a nonempty open set
$U_{\alpha,\varepsilon}\subset U_\alpha^{\mathrm{arith}}$.

Finally, fix the chart normalization at the base point by matching the two
boundary values, which yields
\[
m^{\mathrm{arith}}_\alpha(z_{\alpha,\varepsilon}^{\ast})
=m_{\alpha,\varepsilon}(z_{\alpha,\varepsilon}^{\ast}).
\]
\end{proof}

\begin{proposition}[Local boundary Green matching]\label{inp:calib_kernel_match}
Fix $\alpha>0$ and $\varepsilon\in(0,1]$. There exists a nonempty open set
$U_{\alpha,\varepsilon}\subset\mathbb{C}_+$ such that
\[
\frac{d}{dz}m^{\mathrm{arith}}_\alpha(z)
=\left\langle (\mathsf{H}_{\alpha,\varepsilon}-z)^{-2}\delta_0,\delta_0\right\rangle
\qquad(z\in U_{\alpha,\varepsilon}).
\]
\end{proposition}

\begin{proof}
By Lemma~\ref{lem:local_green_bridge_identity}, there exists nonempty open
$U_{\alpha,\varepsilon}\subset U_\alpha^{\mathrm{arith}}$ such that
\[
\frac{A_\alpha B_\alpha'-A_\alpha'B_\alpha}{A_\alpha^2}
=\left\langle (\mathsf{H}_{\alpha,\varepsilon}-z)^{-2}\delta_0,\delta_0\right\rangle
\quad(z\in U_{\alpha,\varepsilon}).
\]
On the same set, Lemma~\ref{lem:m_arith_chart_derivative} gives
\[
\frac{d}{dz}m^{\mathrm{arith}}_\alpha(z)
=\frac{A_\alpha B_\alpha'-A_\alpha'B_\alpha}{A_\alpha^2}.
\]
Combining the two identities yields the claim.
\end{proof}

\begin{remark}[Source of calibration inputs]\label{rem:calibration_input_source}
Proposition~\ref{inp:calib_arith_stieltjes}
is the in-manuscript Stieltjes package for $m^{\mathrm{arith}}_\alpha$.
Proposition~\ref{inp:calib_kernel_match} is now written directly in-manuscript
as the local bridge statement used by calibration.
The calibration route is closed by the local bridge package around
Proposition~\ref{inp:calib_kernel_match};
the Stieltjes package \ref{inp:calib_arith_stieltjes} is used for
measure-level companion statements.
This route is auxiliary for the manuscript-level RH closure:
Theorem~\ref{thm:RH_from_anchor} is closed via
Theorem~\ref{thm:final_bridge_closed}, not via
the calibration block itself.
\end{remark}

\begin{lemma}[Pick kernels from the two Stieltjes models]\label{lem:pick_kernel_two_sides}
Fix $\alpha>0$ and $\varepsilon\in(0,1]$. There exist finite positive Borel measures
$\mu_{\alpha,\varepsilon}$ and $\mu^{\mathrm{arith}}_\alpha$ on $\mathbb{R}$ such that
\[
m_{\alpha,\varepsilon}(z)
=\int_{\mathbb{R}}\frac{d\mu_{\alpha,\varepsilon}(t)}{t-z},
\qquad
m^{\mathrm{arith}}_\alpha(z)
=\int_{\mathbb{R}}\frac{d\mu^{\mathrm{arith}}_\alpha(t)}{t-z},
\]
for $z\in\mathbb{C}_+$. Consequently,
\[
\frac{m_{\alpha,\varepsilon}(w)-m_{\alpha,\varepsilon}(z)}{w-z}
=\int_{\mathbb{R}}\frac{d\mu_{\alpha,\varepsilon}(t)}{(t-w)(t-z)},
\]
\[
\frac{m^{\mathrm{arith}}_\alpha(w)-m^{\mathrm{arith}}_\alpha(z)}{w-z}
=\int_{\mathbb{R}}\frac{d\mu^{\mathrm{arith}}_\alpha(t)}{(t-w)(t-z)}.
\]
\end{lemma}

\begin{proof}
For $m_{\alpha,\varepsilon}$, spectral theorem for the Dirichlet half-line
Schr\"odinger operator $\mathsf{H}_{\alpha,\varepsilon}$ with cyclic vector
$\delta_0$ gives a positive measure $\mu_{\alpha,\varepsilon}$ such that
\[
m_{\alpha,\varepsilon}(z)
=\int_{\mathbb{R}}\frac{d\mu_{\alpha,\varepsilon}(t)}{t-z}.
\]
Moreover,
\[
\mu_{\alpha,\varepsilon}(\mathbb{R})=\|\delta_0\|^2=1,
\]
so $\mu_{\alpha,\varepsilon}$ is finite.
For $m^{\mathrm{arith}}_\alpha$, Proposition~\ref{inp:calib_arith_stieltjes}
provides a positive
measure $\mu^{\mathrm{arith}}_\alpha$ with
\[
m^{\mathrm{arith}}_\alpha(z)
=\int_{\mathbb{R}}\frac{d\mu^{\mathrm{arith}}_\alpha(t)}{t-z}
\]
as stated there.

Now fix $z,w\in\mathbb{C}_+$ with $z\neq w$. For either measure $\mu$ above,
\[
\frac{1}{w-z}\!\left(\frac{1}{t-w}-\frac{1}{t-z}\right)
=\frac{1}{w-z}\cdot\frac{(t-z)-(t-w)}{(t-w)(t-z)}
=\frac{1}{(t-w)(t-z)}.
\]
Therefore
\[
\frac{1}{w-z}\!\left(\int_{\mathbb{R}}\frac{d\mu(t)}{t-w}
-\int_{\mathbb{R}}\frac{d\mu(t)}{t-z}\right)
=\int_{\mathbb{R}}\frac{d\mu(t)}{(t-w)(t-z)},
\]
which is exactly the stated difference-quotient formula for
$m_{\alpha,\varepsilon}$ and $m^{\mathrm{arith}}_\alpha$.
\end{proof}

\begin{lemma}[Resolvent-square formula for the Weyl derivative]\label{lem:weyl_derivative_resolvent_square}
For fixed $\alpha>0$ and $\varepsilon\in(0,1]$, let
\[
m_{\alpha,\varepsilon}(z)
=\left\langle (\mathsf{H}_{\alpha,\varepsilon}-z)^{-1}\delta_0,\delta_0\right\rangle,
\qquad z\in\mathbb{C}_+.
\]
Then
\[
\frac{d}{dz}m_{\alpha,\varepsilon}(z)
=\left\langle (\mathsf{H}_{\alpha,\varepsilon}-z)^{-2}\delta_0,\delta_0\right\rangle
\qquad(z\in\mathbb{C}_+).
\]
\end{lemma}

\begin{proof}
Write $R(z):=(\mathsf{H}_{\alpha,\varepsilon}-z)^{-1}$.
For $z,z+h\in\mathbb{C}_+$, the resolvent identity gives
\[
R(z+h)-R(z)=h\,R(z+h)R(z).
\]
Hence
\[
\frac{R(z+h)-R(z)}{h}\to R(z)^2
\]
in operator norm as $h\to0$, since $R(\cdot)$ is norm-holomorphic on
$\mathbb{C}\setminus\mathbb{R}$ for self-adjoint $\mathsf{H}_{\alpha,\varepsilon}$.
Taking the matrix element against $\delta_0$ yields
\[
\frac{m_{\alpha,\varepsilon}(z+h)-m_{\alpha,\varepsilon}(z)}{h}
=\left\langle \frac{R(z+h)-R(z)}{h}\delta_0,\delta_0\right\rangle
\longrightarrow
\left\langle R(z)^2\delta_0,\delta_0\right\rangle.
\]
Therefore
\[
m'_{\alpha,\varepsilon}(z)
=\left\langle (\mathsf{H}_{\alpha,\varepsilon}-z)^{-2}\delta_0,\delta_0\right\rangle.
\]
\end{proof}

\begin{lemma}[Local derivative equality implies global constant gap]\label{lem:local_diffquot_constant_gap}
Let $m_1,m_2$ be holomorphic on $\mathbb{C}_+$. Assume there exists a nonempty
open set $U\subset\mathbb{C}_+$ such that
\[
m_1'(z)=m_2'(z)\qquad(z\in U).
\]
Then $m_1-m_2$ is constant on $\mathbb{C}_+$.
\end{lemma}

\begin{proof}
Set $g:=m_1-m_2$.
Then $g'$ is holomorphic on $\mathbb{C}_+$ and vanishes on nonempty open $U$.
By the identity theorem, $g'\equiv 0$ on connected $\mathbb{C}_+$, hence $g$ is
constant on $\mathbb{C}_+$.
\end{proof}

\begin{lemma}[Local derivative matching plus one anchor value]\label{lem:local_derivative_anchor_equal}
Let $m_1,m_2$ be holomorphic on $\mathbb{C}_+$. Assume:
\begin{enumerate}[label=\textup{(\roman*)}]
\item there exists a nonempty open set $U\subset\mathbb{C}_+$ with
$m_1'(z)=m_2'(z)$ for all $z\in U$;
\item there exists $z_\ast\in U$ such that $m_1(z_\ast)=m_2(z_\ast)$.
\end{enumerate}
Then
\[
m_1(z)\equiv m_2(z)\qquad(z\in\mathbb{C}_+).
\]
\end{lemma}

\begin{proof}
By Lemma~\ref{lem:local_diffquot_constant_gap}, $m_1-m_2$ is constant on
$\mathbb{C}_+$, say $m_1-m_2\equiv c$.
Evaluating at $z_\ast$ gives $c=0$, hence $m_1\equiv m_2$.
\end{proof}

\begin{proposition}[Calibrated equality of Weyl functions and Pick kernels]\label{prop:pick_kernel_equal}
For fixed $\alpha>0$ and $\varepsilon\in(0,1]$, one has
\[
m_{\alpha,\varepsilon}(z)=m^{\mathrm{arith}}_\alpha(z)
\qquad(z\in\mathbb{C}_+).
\]
Consequently,
\[
\frac{m_{\alpha,\varepsilon}(w)-m_{\alpha,\varepsilon}(z)}{w-z}
=\frac{m^{\mathrm{arith}}_\alpha(w)-m^{\mathrm{arith}}_\alpha(z)}{w-z}
\qquad(z,w\in\mathbb{C}_+,\ z\neq w).
\]
\end{proposition}

\begin{proof}
Define
\[
K_{\alpha,\varepsilon}(z,w):=
\frac{m_{\alpha,\varepsilon}(w)-m_{\alpha,\varepsilon}(z)}{w-z},
\qquad
K^{\mathrm{arith}}_{\alpha}(z,w):=
\frac{m^{\mathrm{arith}}_\alpha(w)-m^{\mathrm{arith}}_\alpha(z)}{w-z}.
\]
By Proposition~\ref{inp:calib_kernel_match}, there exists a nonempty open set
$U_{\alpha,\varepsilon}\subset\mathbb{C}_+$ such that
\[
\frac{d}{dz}m^{\mathrm{arith}}_\alpha(z)
=\left\langle (\mathsf{H}_{\alpha,\varepsilon}-z)^{-2}\delta_0,\delta_0\right\rangle
\qquad(z\in U_{\alpha,\varepsilon}).
\]
By Lemma~\ref{lem:weyl_derivative_resolvent_square},
\[
\frac{d}{dz}m_{\alpha,\varepsilon}(z)
=\left\langle (\mathsf{H}_{\alpha,\varepsilon}-z)^{-2}\delta_0,\delta_0\right\rangle
\qquad(z\in\mathbb{C}_+),
\]
hence
\[
\frac{d}{dz}m_{\alpha,\varepsilon}(z)
=\frac{d}{dz}m^{\mathrm{arith}}_\alpha(z)
\qquad(z\in U_{\alpha,\varepsilon}).
\]
\[
\text{(local derivative match on }U_{\alpha,\varepsilon}\text{).}
\]
By Lemma~\ref{lem:local_green_bridge_identity}, there exists
$z_{\alpha,\varepsilon}^{\ast}\in U_{\alpha,\varepsilon}$ with
\[
m_{\alpha,\varepsilon}(z_{\alpha,\varepsilon}^{\ast})
=m^{\mathrm{arith}}_\alpha(z_{\alpha,\varepsilon}^{\ast}).
\]
Applying Lemma~\ref{lem:local_derivative_anchor_equal} with
$m_1=m_{\alpha,\varepsilon}$ and $m_2=m^{\mathrm{arith}}_\alpha$ yields
\[
m_{\alpha,\varepsilon}(z)=m^{\mathrm{arith}}_\alpha(z)
\qquad(z\in\mathbb{C}_+).
\]
Hence for any $z,w\in\mathbb{C}_+$ with $z\neq w$,
\[
K_{\alpha,\varepsilon}(z,w)
=\frac{m^{\mathrm{arith}}_\alpha(w)-m^{\mathrm{arith}}_\alpha(z)}{w-z}
=K^{\mathrm{arith}}_\alpha(z,w).
\]
Therefore
\[
K_{\alpha,\varepsilon}(z,w)=K^{\mathrm{arith}}_{\alpha}(z,w),
\]
which is the claimed identity.
\end{proof}

\begin{theorem}[Weyl calibration and Hermite--Biehler property]\label{thm:weyl_calibration}
For each fixed $\alpha>0$ there exists $\varepsilon_0(\alpha)\in(0,1]$ such that for every $\varepsilon\in(0,\varepsilon_0]$,
the Weyl function $m_{\alpha,\varepsilon}$ of \eqref{eq:H_alpha_eps_def} coincides with the arithmetic function \eqref{eq:m_arith_def} on $\mathbb{C}_+$:
\[
m_{\alpha,\varepsilon}(z)\equiv m^{\mathrm{arith}}_\alpha(z)\qquad(z\in\mathbb{C}_+).
\]
Consequently $m^{\mathrm{arith}}_\alpha$ is Herglotz on $\mathbb{C}_+$, and $E_\alpha$ is Hermite--Biehler.
In particular, all zeros of $A_\alpha$ and $B_\alpha$ are real and interlace.
\end{theorem}

\begin{proof}
Fix $\alpha>0$ and $\varepsilon\in(0,\varepsilon_0(\alpha)]$.
By Proposition~\ref{prop:pick_kernel_equal},
\[
m_{\alpha,\varepsilon}\equiv m^{\mathrm{arith}}_\alpha
\qquad\text{on }\mathbb{C}_+.
\]

Since $m_{\alpha,\varepsilon}$ is Herglotz by Lemma~\ref{lem:weyl_is_herglotz}, so is $m^{\mathrm{arith}}_\alpha$.
The Hermite--Biehler property of $E_\alpha$ follows from the standard equivalence
\[
m^{\mathrm{arith}}_\alpha\ \text{Herglotz}\quad\Longleftrightarrow\quad E_\alpha\ \text{Hermite--Biehler},
\]
which is proved by the Cayley transform: $W_\alpha(z):=\frac{m^{\mathrm{arith}}_\alpha(z)-i}{m^{\mathrm{arith}}_\alpha(z)+i}$ satisfies $|W_\alpha|\le 1$ on $\mathbb{C}_+$,
and $W_\alpha$ is exactly $E_\alpha^\sharp/E_\alpha$ when $E_\alpha=A_\alpha-iB_\alpha$.
Interlacing of real zeros of $A_\alpha,B_\alpha$ is a standard consequence of the Hermite--Biehler inequality on $\mathbb{C}_+$.
\end{proof}

\begin{proof}[Proof of Proposition~\ref{inp:calib_arith_stieltjes}]
Fix $\alpha>0$.
By Theorem~\ref{thm:weyl_calibration}, there exists
$\varepsilon_0(\alpha)\in(0,1]$ such that for every
$\varepsilon\in(0,\varepsilon_0(\alpha)]$,
\[
m_{\alpha,\varepsilon}(z)\equiv m^{\mathrm{arith}}_\alpha(z)
\qquad(z\in\mathbb{C}_+).
\]
Fix one such $\varepsilon$.
For the self-adjoint Dirichlet half-line operator
$\mathsf{H}_{\alpha,\varepsilon}$, the spectral theorem with cyclic vector
$\delta_0$ provides a finite positive Borel measure
$\mu_{\alpha,\varepsilon}$ on $\mathbb{R}$ such that
\[
m_{\alpha,\varepsilon}(z)
=\left\langle(\mathsf{H}_{\alpha,\varepsilon}-z)^{-1}\delta_0,\delta_0\right\rangle
=\int_{\mathbb{R}}\frac{d\mu_{\alpha,\varepsilon}(t)}{t-z}
\qquad(z\in\mathbb{C}_+),
\]
and $\mu_{\alpha,\varepsilon}(\mathbb{R})=\|\delta_0\|^2=1$.
Using $m_{\alpha,\varepsilon}\equiv m^{\mathrm{arith}}_\alpha$, we obtain
\[
m^{\mathrm{arith}}_\alpha(z)
=\int_{\mathbb{R}}\frac{d\mu_{\alpha,\varepsilon}(t)}{t-z}
\qquad(z\in\mathbb{C}_+).
\]
Set $\mu^{\mathrm{arith}}_\alpha:=\mu_{\alpha,\varepsilon}$.
\end{proof}

\begin{corollary}[Measure identity from calibrated Weyl equality]\label{cor:measure_identity_from_m}
Under the hypotheses of Theorem~\ref{thm:weyl_calibration}, one has
\[
\mu_{\alpha,\varepsilon}=\mu^{\mathrm{arith}}_\alpha.
\]
\end{corollary}

\begin{proof}
By Theorem~\ref{thm:weyl_calibration},
$m_{\alpha,\varepsilon}\equiv m^{\mathrm{arith}}_\alpha$ on $\mathbb{C}_+$.
By Lemma~\ref{lem:pick_kernel_two_sides}, both admit Stieltjes representations with
positive measures $\mu_{\alpha,\varepsilon}$ and $\mu^{\mathrm{arith}}_\alpha$.
For any interval $I=(a,b)$ whose endpoints are continuity points of both measures,
the Stieltjes inversion formula gives
\[
\mu(I)=\lim_{\eta\downarrow0}\frac{1}{\pi}\int_a^b \Impart m(x+i\eta)\,dx.
\]
Applying this first to $\mu_{\alpha,\varepsilon}$ and then to
$\mu^{\mathrm{arith}}_\alpha$, and using
$m_{\alpha,\varepsilon}(x+i\eta)=m^{\mathrm{arith}}_\alpha(x+i\eta)$ pointwise in
$\eta>0$, yields
\[
\mu_{\alpha,\varepsilon}(I)=\mu^{\mathrm{arith}}_\alpha(I).
\]
Such intervals form a $\pi$-system generating $\mathcal{B}(\mathbb{R})$, so equality
on all of them implies
$\mu_{\alpha,\varepsilon}=\mu^{\mathrm{arith}}_\alpha$ as Borel measures.
\end{proof}

\subsection{Transfer $\alpha\downarrow 0$ and the $\xi$--canonical system}

Define the regularized $\xi$--model on the $z$--plane by
\[
f_\alpha(z):=\Xi_\alpha\Bigl(\tfrac12+i z\Bigr),\qquad f(z):=\Xi\Bigl(\tfrac12+i z\Bigr).
\]
By Lemma~\ref{lem:Xi_alpha_to_Xi}, $f_\alpha\to f$ locally uniformly on $\mathbb{C}$.
For each $\alpha>0$, the Hermite--Biehler property of $E_\alpha(z)=\Xi_\alpha(\tfrac12+z)$ implies that
$f_\alpha$ has only real zeros (equivalently, all zeros of $\Xi_\alpha$ in the critical strip lie on the critical line).

\begin{lemma}[Hurwitz transfer on a zero-free boundary rectangle]\label{lem:hurwitz_rect_transfer}
Let $R\subset\C$ be a closed rectangle with piecewise $C^1$ boundary such that
$\Xi$ has no zeros on $\partial R$. Assume $\Xi_\alpha\to\Xi$ uniformly on a
neighborhood of $\partial R$. Then there exists $\alpha_R>0$ such that for
$0<\alpha<\alpha_R$:
\begin{enumerate}[label=\textup{(\roman*)}]
\item $\Xi_\alpha$ has no zeros on $\partial R$;
\item $\Xi_\alpha$ and $\Xi$ have the same number of zeros in $R$, counted with multiplicity.
\end{enumerate}
\end{lemma}
\begin{proof}
Since $\Xi$ is continuous and nonvanishing on compact $\partial R$,
$m_R:=\min_{\partial R}|\Xi|>0$.
By uniform convergence on $\partial R$, for sufficiently small $\alpha$ we have
$\sup_{\partial R}|\Xi_\alpha-\Xi|<m_R/2$, hence
$|\Xi_\alpha|\ge m_R/2>0$ on $\partial R$, proving (i).
Now apply Rouch\'e's theorem (equivalently the argument principle homotopy form)
on $\partial R$ to $\Xi$ and $\Xi_\alpha$, which gives (ii).
\end{proof}

\begin{theorem}[Riemann Hypothesis and the passivity anchor]\label{thm:RH_from_anchor}
The $\xi$--model function $f(z)=\xi(\tfrac12+i z)$ has only real zeros. Consequently
$H(z)=-f'(z)/f(z)$ is holomorphic on $\mathbb{C}_+$ and satisfies $\Impart H(z)\ge 0$ on $\mathbb{C}_+$.
In particular, $W(z)=\frac{1+iH(z)}{1-iH(z)}$ is holomorphic and Schur on $\mathbb{C}_+$, and there exists a limit-point canonical system realizing $H$
whose transfer matrices satisfy the energy identity \eqref{eq:energy_identity}.
\end{theorem}

\begin{proof}
The closure chain used here is
\[
\text{Theorem~\ref{thm:final_bridge_closed}}
\Longrightarrow
\text{$W$ holomorphic Schur on $\mathbb{C}_+$ and RH}.
\]
By Theorem~\ref{thm:final_bridge_closed}, $W$ is holomorphic on $\mathbb{C}_+$
and RH holds. Equivalently, $f(z)=\xi(\tfrac12+i z)$ has only real zeros.

With RH established, $f$ has no zeros in $\mathbb{C}_+$, so $H=-f'/f$ is holomorphic there.
Moreover, since $f$ has only real zeros and is real-entire, the logarithmic derivative $-f'/f$ is a Herglotz function on $\mathbb{C}_+$,
which may be seen directly from the partial fraction expansion for $\log f$ (or as the locally uniform limit of the Herglotz functions
$m^{\mathrm{arith}}_\alpha$).
Finally, the $L^\ast$ package (Section~\ref{sec:Lstar}) provides the canonical-system realization and yields the energy identity \eqref{eq:energy_identity}.
\end{proof}




\section{Diagnostic module: local Schur recursion expansions and trigger indices}\label{sec:diag_schur_local}

This section is \emph{diagnostic only}: it records local Taylor expansions of the Schur recursion that explain the
numerical ``blow-up / bounce-back'' patterns observed in finite precision.  No statement in this section is used in
the formal implication chain to RH.

\subsection{Local Taylor expansions of the Schur step}
Let $S_k$ be analytic in a neighborhood of $0$ and write
\[
S_k(z)=\alpha_k+b_k z+c_k z^2+O(z^3),\qquad \alpha_k=S_k(0),\quad b_k=S_k'(0),\quad c_k=\tfrac12 S_k''(0).
\]
Define the Schur iterate by
\begin{equation}\label{eq:schur_iterate_def_local}
S_{k+1}(z)=\frac{S_k(z)-\alpha_k}{z\bigl(1-\overline{\alpha_k}\,S_k(z)\bigr)}.
\end{equation}
Set the local margin
\[
\rho_k^2:=1-|\alpha_k|^2.
\]

\begin{lemma}[Exact one-line update for $\alpha_{k+1}$]\label{lem:alpha_k1_exact}
Assume $\rho_k^2\neq 0$. Then $S_{k+1}$ is analytic at $0$ and
\[
\alpha_{k+1}=S_{k+1}(0)=\frac{b_k}{\rho_k^2}=\frac{S_k'(0)}{1-|\alpha_k|^2}.
\]
\end{lemma}

\begin{proof}
Expand the numerator $S_k(z)-\alpha_k=b_k z+c_k z^2+O(z^3)$ and the denominator
\[
z\bigl(1-\overline{\alpha_k}S_k(z)\bigr)
=z\Bigl(\rho_k^2-\overline{\alpha_k}b_k z+O(z^2)\Bigr),
\]
then cancel the factor $z$ and take the limit $z\to 0$.
\end{proof}

\begin{lemma}[First derivative update]\label{lem:bk1_exact}
Assume $\rho_k^2\neq 0$. With $c_k=\tfrac12 S_k''(0)$ one has
\[
S_{k+1}'(0)=\frac{c_k}{\rho_k^2}+\frac{\overline{\alpha_k}\,b_k^2}{\rho_k^4}
=\frac{\tfrac12 S_k''(0)}{1-|\alpha_k|^2}+\frac{\overline{\alpha_k}\,(S_k'(0))^2}{(1-|\alpha_k|^2)^2}.
\]
\end{lemma}

\begin{proof}
Use the geometric-series expansion
\[
\frac{1}{\rho_k^2-\overline{\alpha_k}b_k z+O(z^2)}
=\frac{1}{\rho_k^2}\Bigl(1+\frac{\overline{\alpha_k}b_k}{\rho_k^2}z+O(z^2)\Bigr)
\]
in \eqref{eq:schur_iterate_def_local} and compare coefficients of $z$.
\end{proof}

\subsection{Trigger indices and the blow-up / bounce-back pattern}
\begin{definition}[Observed blow-up index and trigger index]\label{def:trigger_indices}
Given a radius-regularized Schur family $S_r$ with Schur parameters $\alpha_k(r)$, define
\[
k^\ast(r):=\min\{k\ge 0:\ |\alpha_k(r)|\ge 1\}\quad (\text{with }k^\ast(r)=+\infty\text{ if none}),
\]
and the empirical trigger index
\[
j^\ast(r):=\arg\min_{0\le j<k^\ast(r)}\bigl(1-|\alpha_j(r)|^2\bigr)
=\arg\min_{0\le j<k^\ast(r)} \rho_j^2(r).
\]
\end{definition}

\begin{remark}[Why small $\rho_k^2$ is the amplification trigger]
Linearizing the update $\alpha_{k+1}=b_k/\rho_k^2$ around perturbed inputs
$\widetilde\alpha_k=\alpha_k+\delta\alpha_k$, $\widetilde b_k=b_k+\delta b_k$ yields the first-order sensitivity
\[
\delta\alpha_{k+1}\approx \frac{\delta b_k}{\rho_k^2}+\frac{2b_k\,\Re(\overline{\alpha_k}\,\delta\alpha_k)}{\rho_k^4},
\]
so errors are amplified at rates $1/\rho_k^2$ and $1/\rho_k^4$ as $\rho_k^2\downarrow 0$.
\end{remark}

\begin{remark}[Two-step bounce-back scaling]
If $\rho_k^2=\varepsilon\ll 1$ and $b_k=O(1)$, then Lemma~\ref{lem:alpha_k1_exact} gives
$|\alpha_{k+1}|\asymp \varepsilon^{-1}\gg 1$ and hence $\rho_{k+1}^2=1-|\alpha_{k+1}|^2\asymp -\varepsilon^{-2}$.
Meanwhile Lemma~\ref{lem:bk1_exact} typically produces $S_{k+1}'(0)=O(\varepsilon^{-2})$, so
$\alpha_{k+2}=S_{k+1}'(0)/\rho_{k+1}^2=O(1)$. This explains the empirical ``blow-up then return'' pattern in finite precision.
\end{remark}

\begin{remark}[Leakage and near-singularity of Toeplitz data]
The leakage functional
$
L_N(r)=\sum_{k=0}^{N-1}-\log(1-|\alpha_k(r)|^2)=\sum_{k=0}^{N-1}-\log \rho_k^2(r)
$
satisfies $\exp(-L_N(r))=\prod_{k=0}^{N-1}\rho_k^2(r)$.
In standard OPUC normalizations one also has the determinant identity
$\det T_N(w_r)=\prod_{k=0}^{N-1}(1-|\alpha_k(r)|^2)$ for the Toeplitz matrix built from the boundary weight
$w_r(e^{it})=\Re\frac{1+S_r(e^{it})}{1-S_r(e^{it})}$, so small $\rho_j^2(r)$ corresponds to near-singularity of Toeplitz data.
\end{remark}



\section{References (minimal)}\label{app:refs}

\begin{thebibliography}{10}

\bibitem{SimonOPUCOneFoot}
B.~Simon, \emph{OPUC on One Foot}, \emph{Bull. Amer. Math. Soc. (N.S.)} 42(4) (2005), 431--460; also arXiv:math/0502485. 


\bibitem{SimonExpDecay2007}
B.~Simon, \emph{Orthogonal Polynomials with Exponentially Decaying Recursion Coefficients}, in \emph{Spectral Theory and Mathematical Physics: A Festschrift in Honor of Barry Simon's 60th Birthday}, Proc. Sympos. Pure Math. 76 (2007), 253--274; preprint available as \emph{R47} on the author's page.

\bibitem{GeronimoBaxter2005}
J.~S.~Geronimo and A.~Mart\'inez-Finkelshtein, \emph{On Extensions of a Theorem of Baxter}, arXiv:math/0508224 (2005); published in \emph{J. Approx. Theory} 140 (2006), 1--52.


\bibitem{Sakhnovich2019Verblunsky}
A.~L.~Sakhnovich, \emph{New Verblunsky-type coefficients of block Toeplitz and Hankel matrices and of corresponding Dirac and canonical systems}, \emph{J. Approx. Theory} (2019), DOI: 10.1016/j.jat.2019.04.007.

\bibitem{BessonovLukicYuditskii2020}
R.~Bessonov, M.~Luki\'c, P.~Yuditskii, \emph{Reflectionless canonical systems, I. Arov gauge and right limits}, arXiv:2011.05261 (2020).

\bibitem{DamanikEichingerYuditskii2021Arov}
D.~Damanik, B.~Eichinger, P.~Yuditskii, \emph{Szeg\H{o}'s theorem for canonical systems: the Arov gauge and a sum rule}, \emph{J. Spectr. Theory} 11 (2021), 1255--1277, DOI: 10.4171/JST/371.

\bibitem{AleksandrovPeller2025}
A.~B.~Aleksandrov and V.~V.~Peller, \emph{Analytic Schur multipliers}, arXiv:2506.15173v1 [math.FA] (2025).

\bibitem{GarnettBAF}
J.~B.~Garnett,
\emph{Bounded Analytic Functions},
Graduate Texts in Mathematics 236, Springer, 2007.

\bibitem{DurenHp}
P.~L.~Duren,
\emph{Theory of $H^p$ Spaces},
Academic Press, 1970.


\bibitem{KoosisLogInt}
P.~Koosis,
\emph{The Logarithmic Integral I},
Cambridge Studies in Advanced Mathematics 12, Cambridge University Press, 1988.

\bibitem{Reis2025Scielo}
E.~A.~dos Reis,
\emph{Hilbert--P\'olya via de Branges and the Weyl $m$-Function: Vertical Convolution and the Riemann Hypothesis},
SciELO Preprints (2025), DOI: \url{https://doi.org/10.1590/SciELOPreprints.13340}.

\bibitem{LangerWoracek2022}
H.~Langer and H.~Woracek, \emph{Canonical systems, limit-point/limit-circle and Weyl disks}, survey notes / preprint (2022).


\bibitem{Derkach2015BoundaryTriplets}
V.~A.~Derkach,
\emph{Boundary triplets, Weyl functions, and the Kre\u{\i}n formula},
in: \emph{Operator Theory: Advances and Applications}, Springer, 2015.
(Contains the Weyl-function transformation law under boundary-condition changes.)


\bibitem{DLMF1833}
NIST Digital Library of Mathematical Functions, \S 18.33 (OPUC / Schur algorithm and disk M\"obius transforms), \url{https://dlmf.nist.gov/18.33}.

\bibitem{DLMF511}
NIST Digital Library of Mathematical Functions, \S 5.11 (asymptotic expansions of the digamma function), \url{https://dlmf.nist.gov/5.11#E2}.

\bibitem{UngarGyrogroups2008}
A.~A.~Ungar, \emph{Analytic Hyperbolic Geometry and Albert Einstein's Special Theory of Relativity}, World Scientific, 2008.
(See the unit-disk gyrogroup and Thomas--Wigner rotation/gyration formulas.)


\bibitem{MonardLecture9}
F.~Monard,
\emph{Lecture 9: Automorphisms of the unit disk} (Math 207, Spring 2017),
University of California Santa Cruz (PDF),
\url{https://people.ucsc.edu/~fmonard/Sp17_Math207/lecture9.pdf}.


\bibitem{debranges1968}
L.~de Branges,
\emph{Hilbert Spaces of Entire Functions},
Prentice--Hall, 1968.

\bibitem{agler_mccarthy2002}
J.~Agler and J.~E.~McCarthy,
\emph{Pick Interpolation and Hilbert Function Spaces},
Graduate Studies in Mathematics, Vol.~44, American Mathematical Society, 2002.


\bibitem{lilleberg2022passive}
L.~Lilleberg,
\emph{Factorizations of generalized Schur functions and products of passive systems},
Methods of Functional Analysis and Topology \textbf{28} (2022), no.~1, 66--88.
\url{https://mfat.imath.kiev.ua/wp-content/uploads/2022/02/28_1_2022_lilleberg.pdf}

\bibitem{kurula2006dirac}
M.~Kurula, A.~J.~van der Schaft, and H.~Zwart,
\emph{Composition of Infinite-Dimensional Linear Dirac-type Structures} (MTNS 2006 draft).
\url{https://web.abo.fi/~mkurula/pdf/mtns06-draft.pdf}


\bibitem{malinen2010conservative}
J. Malinen, O. J. Staffans, and G. Weiss,
\emph{When is a linear system conservative?}
\emph{Quarterly of Applied Mathematics} \textbf{64} (2006), no.~1, 61--91.
\bibitem{lilleberg2020generalized}
K. Lilleberg,
\emph{Generalized Schur--Nevanlinna functions and their realizations}.
\emph{Integral Equations and Operator Theory} \textbf{92} (2020), Article~42.
DOI: 10.1007/s00020-020-02600-w.
\bibitem{sznagyfoias1970harmonic}
B.~Sz.-Nagy and C.~Foia\c{s},
\emph{Harmonic Analysis of Operators on Hilbert Space},
North-Holland, 1970.

\bibitem{staffans2005wellposed}
O.~Staffans,
\emph{Well-Posed Linear Systems},
Cambridge University Press, 2005.

\bibitem{redheffer1962} 
R.~Redheffer,
\emph{Inequalities for a matrix Riccati equation},
J. Math. Mech. \textbf{11} (1962), 349--367.


\bibitem{SrikantHardyNotes}
R.~K.~Srivastava.
\newblock \emph{Advanced Course on Hardy Spaces} (lecture notes).
\newblock IIT Guwahati, 2022. Available at
\texttt{https://www.math.iitg.ac.in/\string~rakesh/Teaching/Hardy.pdf}.

\end{thebibliography}


\end{document}
